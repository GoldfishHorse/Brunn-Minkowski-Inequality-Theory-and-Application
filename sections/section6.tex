\chapter{Blaschke–Santaló 不等式和仿射等周不等式}\label{chapter:6}

 


\section{Blaschke–Santaló 不等式}
Blaschke–Santaló 不等式也是凸分析里的一个十分重要的不等式,它揭示了凸体和其极集的关系,并且可以结合 Brunn-Minkowski 不等式得到一些非常优美的结论。%首先我们补充几个定义。
\subsection{Santaló  点}
\begin{lemma}[Santaló 点]
对于任意凸体 $K\subset\mathbb R^n$,取
\[
(K-x)^\circ=\{y\in\mathbb R^n:\langle y,z-x\rangle\le 1,\;\forall z\in K\}
\]
那么对 $x\in\mathrm{int}K$,存在唯一的点 $x^*$最小化函数 $g(x):= V_n((K-x)^\circ)$,且
\begin{equation}
\int_{\mathbb{S}^{n-1}}\Big(h_K(u)-\langle x^*,u\rangle\Big)^{-(n+1)}u\,\mathrm{d}u=0 \label{eq:6-1}\tag{6-1}
\end{equation}
称点 $x^*$为 \emph{Santaló 点}。特别地, $x^*=0$当且仅当 $K$的极集 $K^\circ$重心在原点。
\end{lemma}

\begin{proof}
由于对 $x\in\mathrm{int}K$有 $h_{K-x}(u)=h_K(u)-\langle x,u\rangle$,用式 ~\eqref{eq:4-2}和引理~\ref{lem:support_radial},
\begin{align*}
g(x)&=V_n((K-x)^\circ)=\frac1n\int_{\mathbb{S}^{n-1}}\rho_{(K-x)^\circ}^n(u)\mathrm{d}u\\
&=\frac1n\int_{\mathbb{S}^{n-1}}h_{K-x}^{-n}(u)\mathrm{d}u=\int_{\mathbb{S}^{n-1}}\big(h_K(u)-\langle x,u\rangle\big)^{-n}\mathrm{d}u
\end{align*}
注意到
\begin{enumerate}
\item $g(x)$这对 $x$是严格凸的 (因为 $t\mapsto t^{-n}$严格凸);
\item 当 $x$趋近于 $K$的边界 $\partial K$时,存在 $u$使得 $h_K(u)-\langle x,u\rangle\to 0^+$,故 $g(x)\to\infty$。
\end{enumerate}
所以必存在一个唯一的 $x^*\in\mathrm{int}K$最小化 $g(x)$。由梯度 $\nabla g(x^*)=0$即可得到式 \eqref{eq:6-1}。

特别地,在 \eqref{eq:6-1}中取 $x^*=0$,有
\begin{align*}
K^\circ\text{ 的重心 }&=\int_{K^\circ}x\,\mathrm{d}x\\
&=\int_{\mathbb S^{n-1}}\int_0^{\rho_{K^\circ}(u)}ru\cdot r^{n-1}\,\mathrm{d}r\mathrm{d}u\\
&=\frac1{n+1}\int_{\mathbb S^{n-1}}\rho_{K^\circ}^{n+1}(u)u\,\mathrm du\\
&=\frac1{n+1}\int_{\mathbb S^{n-1}}h^{-(n+1)}_{K^\circ}(u)u\,\mathrm du\\
&=0 \tag{由式~\eqref{eq:6-1}}
\end{align*}
\end{proof}

\begin{remark}
\begin{itemize}
\item 注意式~\eqref{eq:6-1}的被积分项是向量值函数,右侧的 0 也代表全 0 的 $n$维向量。
\item 该引理参考了 Böröczky, K. J., Figalli, A., \& Ramos, J. P. (2025). \textit{Isoperimetric inequalities, Brunn-Minkowski theory and Minkowski type Monge-Ampère equations on the sphere}.的 Lemma 6.5.1 (p. 211)。
\end{itemize}
\end{remark}

\subsection{BS 不等式的证明}

\begin{theorem}[Blaschke–Santaló 不等式]\label{thm:bs}
设 $K\subset\mathbb R^n$是一个凸体,且其 Santaló 点 在原点,即
\begin{equation}
\int_{\mathbb{S}^{n-1}}h_K^{-(n+1)}(u)u\, \mathrm{d}u=0 \label{eq:6-2}\tag{6-2}
\end{equation}
那么成立不等式
\begin{equation}
V_n(K)V_n(K^\circ)\le\big(V_n(B_n)\big)^2 \label{eq:6-3}\tag{6-3}
\end{equation}
取等当且仅当 $K$是一个中心在原点的椭球。
\end{theorem}

\begin{proof}
由于 $K$有界且 $0\in\mathrm{int}K$,支撑函数 $0<h_K<\infty$恒成立,我们定义一个新的测度
\[
\mathrm{d}\mu:=h_K^{-(n+1)}(u)\,\mathrm{d}u
\]
这是一个球面 $\mathbb {S}^{n-1}$上的、非零、有限的 Borel 测度,而且根据式 \eqref{eq:6-2}有 $\mu$的重心在原点:
\[
\int_{\mathbb{S}^{n-1}}u\,\mathrm{d}\mu(u)=0
\]
同时,由于 $h_K$在球面上恒大于 0,所以 $\mu$肯定不集中在任意一个大圆上。至此, $\mu$满足了 Minkowski 问题的全部条件 (定理~\ref{thm:minkowski_exist}),可以找到凸体 $\Lambda K$使得
\begin{equation}
\mathrm{d}\mu(u)=h_K^{-(n+1)}(u)\,\mathrm{d}u=\mathrm{d}S(\Lambda K,u) \label{eq:6-4}\tag{6-4}
\end{equation}
有的文献中称 $\Lambda K$为"曲率体 (curvature body)",因为上式同时表明 $\Lambda K$的曲率函数满足,$f_{\Lambda K}:=\frac{\mathrm{d}S(\Lambda K,u)}{\mathrm du}=h_K^{-(n+1)}$。

$\Lambda K$在至多相差一个平移的意义下是唯一的,且还有
\begin{align*}
&f_{\Lambda K}(u)=h_K^{-(n+1)}(u)\\
\implies \;&V_n(K^\circ)=\frac1n\int_{\mathbb{S}^{n-1}}\rho_{K^\circ}^n(u)\,\mathrm{d}u=\frac1n\int_{\mathbb S^{n-1}}h_K^{-n}(u)\,\mathrm du= \frac1n\int_{\mathbb S^{n-1}}f_{\Lambda K}^{\frac n{n+1}}(u)\,\mathrm du \label{eq:6-5}\tag{6-5}\\
\implies\;& nV_n(K^\circ)=\int_{\mathbb S^{n-1}}f_{\Lambda K}^{\frac n{n+1}}(u)\,\mathrm du=:as(\Lambda K) \label{eq:6-6}\tag{6-6}
\end{align*}
这里补充一下定义,

\begin{definition}[曲率函数]
设 $K\subset \mathbb R^n$是凸体,且其表面积测度相对于球面测度绝对连续,即 $\mathrm{d}S(K,u)\ll \mathrm du$,那么称二者的 Radon-Nikodym 导数为 $K$的\emph{曲率函数} (curvature function),记作 $f_K$。
\end{definition}

\begin{definition}[仿射表面积]
设 $K\subset \mathbb R^n$是凸体且存在曲率函数 $f_K$,那么定义 $K$的\emph{仿射表面积} (affine surface area) 为
\[
as(K):=\int_{\mathbb S^{n-1}}f_K^{\frac n{n+1}}(u)\,\mathrm du
\]
\end{definition}

继续定理~\ref{thm:bs}的证明,回顾一下式~\eqref{eq:6-5},还能发现
\begin{align*}
V_n(K^\circ)&=\frac1n\int_{\mathbb S^{n-1}}h_K^{-n}(u)\,\mathrm du \tag{由式~\eqref{eq:6-5}}\\
&=\frac1n\int_{\mathbb S^{n-1}}h_K(u) h_K^{-(n+1)}(u)\,\mathrm du \\
&=\frac1n\int_{\mathbb S^{n-1}}h_K(u)\,\mathrm{d}S(\Lambda K,u) \tag{由式~\eqref{eq:6-4}}\\
&=V_1(\Lambda K, K) \tag{定理~\ref{thm:var}}\\
&\ge V_n(\Lambda K)^{\frac{n-1}n}V_n(K)^{\frac1n} \tag{定理~\ref{thm:minkowski}}
\end{align*}
即
\begin{align*}
&V_n(K^\circ)^n\ge V_n(\Lambda K)^{n-1}V_n(K)\\
\implies\;&V_n(K^\circ)^{n+1}\ge V_n(\Lambda K)^{n-1}V_n(K)V_n(K^\circ)\\
\implies\;&\left(\frac{as(\Lambda K)}{n}\right)^{n+1}\ge V_n(\Lambda K)^{n-1}V_n(K)V_n(K^\circ) \tag{由式~\eqref{eq:6-6}}
\end{align*}
我们发现自己找到了欲证目标 \eqref{eq:6-3}的一个上界,
\begin{equation}
\frac{as(\Lambda K)^{n+1}}{n^{n+1}V_n(\Lambda K)^{n-1}}\ge V_n(K)V_n(K^\circ) \label{eq:6-7}\tag{6-7}
\end{equation}
如果能再证明这个上界不超过 $V_n(B_n)^2$的话,就证明出来了 BS 不等式。

实际上可以证明(会在后文中介绍),
\begin{equation}
\frac{as(\Lambda K)^{n+1}}{n^{n+1}V_n(\Lambda K)^{n-1}}\le \frac{as(B_n)^{n+1}}{n^{n+1}V_n(B_n)^{n-1}} \label{eq:6-8}\tag{6-8}
\end{equation}
等号成立当且仅当 $\Lambda K$是椭球;而这正是我们想要的上界:
\begin{align*}
\frac{as(B_n)^{n+1}}{n^{n+1}V_n(B_n)^{n-1}}&=\frac{\displaystyle\left(\int_{\mathbb{S}^{n-1}}f_{B_n}^{\frac n{n+1}}(u)\,\mathrm{d}u\right)^{n+1}}{n^{n+1}V_n(B_2)^{n-1}}\\
&=\frac{\displaystyle \left(\int_{\mathbb{S}^{n-1}}\mathrm{d}u\right)^{n+1}}{n^{n+1}V_n(B_n)^{n-1}} \tag{$f_{B_n}\equiv 1$}\\
&=\frac{S(\mathbb S^{n-1})^{n+1}}{n^{n+1}V_n(B_n)^{n-1}}\\
&=\frac{(nV_n(B_n))^{n+1}}{n^{n+1}V_n(B_n)^{n-1}}\\
&=V_n(B_n)^2 \label{eq:6-9}\tag{6-9}
\end{align*}
结合 \eqref{eq:6-7}、 \eqref{eq:6-8}、 \eqref{eq:6-9}立即得到 BS 不等式 $V_n(K)V_n(K^\circ)\le V_n(B_n)^2$。

下面简单叙述一下取等条件,显见对任意的非奇异线性变换 $T$, 
\begin{align*}
(TK)^\circ&=\big\{y\in\mathbb{R}^n:\langle y,Tx\rangle\le 1,\,\forall x\in K\big\}\\
&=\big\{y\in\mathbb{R}^n:\big\langle T^\top y,x\big\rangle\le 1,\,\forall x\in K\big\}\\
&=\big(T^\top\big)^{-1}\big\{z\in\mathbb{R}^n:\langle z,x\rangle\le 1,\,\forall x\in K\big\} \tag{令 $z:=T^{\top}y$}\\
&=\big(T^\top\big)^{-1}K^\circ
\end{align*}
故
\begin{align*}
V_n(TK)V_n((TK)^\circ)&=V_n(TK)V_n\left(\big(T\big)^{-1}K^\circ\right)\\
&=\mathrm{det}(T)V_n(K)\cdot\frac1{\det(T)}V_n(K)\\
&=V_n(K)V_n(K^\circ)
\end{align*}
即, $V_n(K)V_n(K^\circ)$作为凸体 $K$的泛函在 $T$作用下是不变的。又显然在 $K=B_n$时 BS 不等式取等,根据椭球的定义,对任意中心在原点的椭球都有 BS 不等式取等。

另一方面,若 BS 不等式取等, \eqref{eq:6-8}就应取等, $\Lambda K$是一个椭球;又由于证明过程中用了 Minkowski 第一不等式 (定理~\ref{thm:minkowski}),其取等条件是 $K$和 $\Lambda K$位似——所以 $K$也得是一个椭球。根据引理 1, $K^\circ$重心在原点,自然 $K$也应该是一个中心在原点的椭球。
\end{proof}

\begin{remark}
\begin{itemize}
\item 回忆一下,椭球的定义是:设 $B_n(x,r)$是一个球,则对其进行非奇异线性变换后 $TB_n(x,r)$就是椭球 (见定义~\ref{def:ball});因而这里所说的``椭球"本身都是包含球的。
\item $V_n(K)V_n(K^\circ)$有时也称作\emph{体积积} (volume product)。
\item BS 不等式最早由 W. Blaschke 于 1920s 给出了 $n=2,3$情况下的证明,L. A. Santaló 则在 1949 年给出了一般维度的证明\footnotemark,思路和前面给出的证明基本相同;这个证明本质上是融合了两个重要结论:Minkowski 第一不等式,以及式 \eqref{eq:6-8}——它叫做\emph{仿射等周不等式} (affine isoperimetric inequality)。
\item 早期,仿射表面积 $as(K)$的概念来自于微分几何,是只定义在足够光滑的凸体上的。尽管,我们直觉上可以用光滑凸体来逼近非光滑凸体,从而表明与仿射表面积相关的不等式在一般凸体下也成立,但这样的话等号成立条件可能就失效了。鉴于上述原因,对仿射等周不等式和  BS 不等式的取等条件讨论直至 1985 年才由 C. M. Petty 给出\footnotemark。
\end{itemize}
\end{remark}
% \footnotetext{}
\footnotetext{Santaló, L. A. (1949). \textit{Un invariante afin para los cuerpos convexos del espacio de n dimensiones}. Portugaliae mathematica, 8, 155-161.}
\footnotetext{Petty, C. M. (1985). \textit{Affine isoperimetric problems}. Annals of the New York Academy of Sciences, 440(1), 113-127.}
\subsection{另一种证明}

下面考虑一种特殊情况: $K$是关于原点\emph{中心对称} (origin-symmetric) 的凸体,即 $K=-K$,此时可以利用 BM 不等式 + Steiner 对称化比较容易地给出 BS 不等式的证明。

回忆一下用 Steiner 对称化证明不等式的一般步骤(见第~\ref{chapter:3} 章),我们得先找到一个在 Hausdorff 度量下连续的泛函,让它在 Steiner 对称化操作下满足单调性。在这里,需要的是
\begin{equation}
V_n(K^\circ)\le V_n((S_uK)^\circ) \label{eq:6-10}\tag{6-10}
\end{equation}
只要上式成立,就可以根据 Gross 定理 (定理~\ref{thm:gross}),设凸体 $K$在一列单位方向$\{u_j\}_{j=1}^\infty$下依 Hausdorff 度量收敛至球 $\left(\frac{V_n(K)}{V_n(B_n)}\right)^{\frac1n}B_n$,并记 $K_j:=S_{u_j}S_{u_{j-1}}\cdots S_{u_2}S_{u_1}K$,则
\begin{align*}
V_n(K)V_n(K^\circ) \le V_n(K_1)V_n(K_1^\circ) \le V_n(K_2)V_n(K_2^\circ)\le \cdots\le V_n(K_j)V_n(K_j^\circ)\le\cdots
\end{align*}
取极限,
\begin{align*}
V_n(K)V_n(K^\circ)&\le \lim_{j\to\infty}V_n(K_j)V_n(K_j^\circ)\\
&= V_n(K)\lim_{j\to\infty}V_n(K_j^\circ) \tag{Steiner 对称化不改变体积}\\
&=V_n(K)V_n\left(\lim_{j\to\infty}K_j^\circ\right) \tag{体积对 Hausdorff 度量连续}\\
&=V_n(K)V_n\left(\left(\lim_{j\to\infty}K_j\right)^\circ\right)\\
&=V_n(K)V_n\left(\left(\frac{V_n(K)}{V_n(B_n)}\right)^{-\frac1n}B_n\right)\\
&=V_n(B_n)^2
\end{align*}
这就得到了 BS 不等式。上面第四行来自于这样的事实:

\begin{lemma}
设 $K_0,K_1,K_2,\dots\subset\mathbb{R}^n$是凸体,且在 Hausdorff 度量下 $K_j\to K_0$;如果 $0\in\mathrm{int}K_j $对任意 $j=1,2,\ldots$和 $j=0$都成立,则依 Hausdorff 度量收敛可以与取极集交换,即
\[
\lim_{j\to\infty}K_j^\circ=\left(\lim_{j\to\infty}K\right)^\circ=K_0^\circ
\]
\end{lemma}

\begin{proof}
$0\in\mathrm{int}K_j $保证了半径函数 $0<\rho_{K_j}<\infty$,且 $K_j\to K_0$意味着 $\rho_{K_j}\to\rho_{K_0}$一致收敛(因为 Hausdorff 度量的定义蕴含 $|\rho_{K_j}(u)-\rho_{K_0}(u)|\le d_H(K_j,K_0),\;\forall u\in\mathbb{S}^{n-1}$)。

设 $r_0:=\min\limits_{u\in\mathbb{S}^{n-1}}\rho_{K_0}(u)>0$,则对任意取定的 $0<\epsilon<1$和充分大的 $j$有 $ \rho_{K_j}\ge (1-\epsilon)r_0$。故
\begin{align*}
d_H(K_j^\circ,K_0^\circ)&=\sup\limits_{u\in\mathbb S^{n-1}}\big|h_{K_j^\circ}(u)-h_{K_0^\circ}(u)\big|\\
&=\sup\limits_{u\in\mathbb S^{n-1}}\left|\frac1{\rho_{K_j}(u)}-\frac1{\rho_{K_0}(u)}\right|\\
&\le\frac1{(r_0(1-\epsilon))^2}\sup_{u\in\mathbb S^{n-1}}\big|\rho_{K_j}(u)-\rho_{K_0}(u)\big|\\
&\to 0
\end{align*}
\end{proof}

可以看出,只要证明了 \eqref{eq:6-10},就可以立即得到 BS 不等式成立。再写作引理

\begin{lemma}\label{lem:ball}
设 $K\subset\mathbb R^n$是凸体且关于原点中心对称,那么对任意固定的 $u\in\mathbb S^{n-1}$有 $V_n(K^\circ)\le V_n((S_uK)^\circ)$,取等当且仅当每个平行于 $u^\perp$的超平面截 $K^\circ$得到的集合都是中心对称的。
\footnotemark
\end{lemma}
\footnotetext{Keith M. Ball, 1986}

\begin{proof}
由于 $u$具体方向不重要,不失一般性设 $u=e_n$是第 $n$个坐标向量,则 $u^\perp=\mathbb R^{n-1}$。于是我们可以按前 $n-1$维+第 $n$维来分解 $S_uK$(回忆一下式~\eqref{eq:3-3}):
\begin{align*}
S_uK&=\left\{(x,t):x\in\mathbb R^{n-1},|t|\le\frac{f(x)-g(x)}{2}\right\}\\
&=\left\{\left(x,\frac{t-s}{2}\right):(x,s)\in K, (x,t)\in K\right\}
\end{align*}
这里 $(x,t)$代表一个 $n$维向量, $x\in\mathbb R^{n-1}$、 $t\in\mathbb R$。

则 Steiner 对称化的极集为
\begin{align*}
(S_u K)^\circ&=\left\{(y,r)\in\mathbb R^n:\left\langle(y,r),\left(x,\frac{t-s}{2}\right)\right\rangle\le1,\;\forall\, (x,t)\in K,(x,s)\in K\right\}\\
&=\left\{(y,r)\in\mathbb R^n:\langle y,x\rangle+\frac{t-s}{2}r\le 1,\;\forall\, (x,t)\in K,(x,s)\in K\right\}
\end{align*}
对任意凸体 $L$,定义它在 $u=e_n$方向的截面为 $L_\tau:=\{z\in\mathbb{R}^{n-1}:(z,\tau)\in L\}$(可以视作用与 $u^\perp$平行的超平面截 $L$所获得的一系列切片)。那么对任意的 $r,s>0$,取定 $x\in K^\circ_r$、 $y\in K^\circ_{-s}$,则对任意 $\tau\in\mathbb R^n$和 $z\in K_\tau$有
\begin{align*}
(x,r)\in K^\circ&\implies \langle x,z\rangle+r\tau\le 1\\
(y,-s)\in K^\circ&\implies \langle y,z\rangle-s\tau\le 1
\end{align*}

若令 $w:=\frac{s}{r+s}x+\frac{r}{r+s}y$ (见图~\ref{fig:BS_proof}),那么上面两个关系可以推出:对 $(z,\tau_1)\in K$和 $(z,\tau_2)\in K$有
\begin{align*}
&\langle w,z\rangle+\frac{rs}{r+s}(\tau_1-\tau_2)\\
&=\frac s{r+s}\langle x,z\rangle+\frac r{r+s}\langle y,z\rangle+\frac{s}{r+s}r\tau_1-\frac{r}{r+s}s\tau_2\\
&=\frac{s}{r+s}\Big(\underbrace{\langle x,z\rangle+r\tau_1}_{\le 1}\Big)+\frac{r}{r+s}\Big(\underbrace{\langle x,z\rangle-s\tau_2}_{\le 1}\Big)\\
&\le \frac s{r+s}+\frac r{s+r}=1
\end{align*}
即 $\displaystyle\left\langle\left(w,\frac{2rs}{r+s}\right),\left(z,\frac{\tau_1-\tau_2}{2}\right)\right\rangle\le 1$;又由于 $(z,\tau_1)\in K$、 $(z,\tau_2)\in K$,有 $\displaystyle \left(z,\frac{\tau_1-\tau_2}{2}\right)\in S_uK$,故 $\displaystyle\left(w,\frac{2rs}{r+s}\right)\in (S_uK)^\circ$,或者说 $w\in (S_uK)^{\circ}_{\frac{2rs}{r+s}}$。

\begin{figure}[htbp]
\centering
\includegraphics[width=0.7\textwidth]{figures/BS_proof.pdf}
\caption{引理~\ref{lem:ball} 证明的示意图:在$K^\circ$ 的两个截面$K^\circ_{-s}$ 和 $K^\circ_{t}$ 上分别选取点$x$ 和 $y$;$w$ 为二者的插值。}
\label{fig:BS_proof}
\end{figure}

我们上面的分析揭示了这样的一个关系:
\[
\frac s{r+s}K^\circ_r +\frac r{r+s}K^\circ_{-s}\subseteq (S_uK)^\circ _{\frac{2rs}{r+s}}
\]
用 BM 不等式,有
\begin{align*}
V_{n-1}\left( (S_uK)^\circ _{\frac{2rs}{r+s}}\right)&\ge V_{n-1}\left(\frac s{r+s}K^\circ_r +\frac r{r+s}K^\circ_{-s}\right)\\
&\ge V_{n-1}(K_r^\circ)^{\frac s{r+s}}V_{n-1}(K^\circ_s)^{\frac r{r+s}} \tag{BM 不等式}
\end{align*}
上述分析对一般的凸体 $K$都成立,不过要是继续分析下去就会变得非常麻烦,故此我们只考虑更加简单的情形: $K$关于原点中心对称。这个条件的好处在于:由于 $K=-K$,有 $K^\circ=-K^\circ$,即 $K^\circ$也是关于原点中心对称的,可以直接在上式取 $s=r$,即
\begin{align*}
V_{n-1}((S_u K)^\circ_r)&\ge V_{n-1}(K^\circ_r)^{\frac12}V_{n-1}(K^\circ_{-r})^{\frac12}\\
&=V_{n-1}(K_r^\circ) \tag{由中心对称}
\end{align*}
直接对 $r$积分就得到了想要的结论:
\[
V_n((S_uK)^\circ)=2\int_0^\infty V_{n-1}((S_uK)^\circ_r)\,\mathrm dr\ge 2\int_0^\infty V_{n-1}(K^\circ_r)\,\mathrm dr=V_n(K^\circ)
\]
下面简单讨论一下取等条件,要使得等号成立,BM 不等式那一步就应取等,即 $K_r^\circ$和 $K_{-r}^\circ=-K_r^\circ$位似——二者只能相差一个平移,所以 $K^\circ_r$一定存在一个对称中心,这对所有 $r>0$都成立。
\end{proof}

\begin{remark}
\begin{itemize}
\item 上面的引理也可以参考 Böröczky, K. J., Figalli, A., \& Ramos, J. P. (2025). \textit{Isoperimetric inequalities, Brunn-Minkowski theory and Minkowski type Monge-Ampère equations on the sphere}.的 Proposition 6.5.3 (pp. 212-213)。
\item 一个一般的凸体 $C\subseteq\mathbb R^n$是中心对称的,当且仅当存在 $x_0\in C$使得 $C-x_0$是关于原点中心对称的,即 $(C-x_0)=-(C-x_0)$。
\item 有这样的结论 
\begin{theorem*}[Brunn]
给定 $2<m<n$, $K\subset\mathbb R^n$是凸体,若任意一个 $m$维超平面与 $K$的交集都是中心对称的 (可能为空集),那么 $K$一定是一个椭球。
\end{theorem*}
这个定理结合引理~\ref{lem:ball}表明,若 $V_n(K^\circ)= V_n((S_uK)^\circ)$对任意 $u\in\mathbb S^{n-1}$成立,则任意方向和位置的 $n-1$维超平面截 $K^\circ$得到的集合都是中心对称的,所以 $K^\circ$只能是椭球,故而 $K$也只能是椭球。这再次刻画了 BS 不等式的取等条件。
\item 引理~\ref{lem:ball}的思路——对比 $K^\circ$的截面和 $(S_uK)^\circ$的截面,其实可以推广到一般的、不一定中心对称的凸体,这个思路是行得通的,但是分析很复杂,超出了本文的范围。读者可以参阅 Meyer, M., \& Pajor, A. (1990). \textit{On the Blaschke-Santaló inequality}.  Archiv der Mathematik, 55(1), 82-93. 一文的 Lemma 7。
\end{itemize}
\end{remark}

\section{仿射等周不等式}

前文的式~\eqref{eq:6-8}挖了一个大坑,最后还是要填一下。我们需证明下面这样的结论:

\begin{theorem}[仿射等周不等式]\label{thm:as_isoperimetric}
设 $K\subset\mathbb R^n$是凸体,那么
\[
\left(\frac{as(K)}{as(B_n)}\right)^{\frac{n+1}{n-1}}\le \frac{V_n(K)}{V_n(B_n)}
\]
取等当且仅当 $K$是一个椭球。
\end{theorem}

\begin{remark}
根据式 \eqref{eq:6-9},仿射等周不等式的形式也可以改写作
\[
as(K)\le nV_n(B_n)^{\frac2{n+1}}V_n(K)^{\frac{n-1}{n+1}}
\]
\end{remark}

\subsection{仿射表面积}

先重新审视一下仿射表面积的定义。之前的定义是这样的:若凸体 $K\subset\mathbb R^n$存在曲率函数 $f_K\ge 0$使得 $\mathrm{d}S(K,u)=f_K(u)\mathrm{d}u$,那么
\[
as(K):=\int_{\mathbb S^{n-1}}f_K^{\frac n{n+1}}(u)\,\mathrm du
\]
就定义为 $K$的仿射表面积。早期这个概念是来自于微分几何,因此限制了 $K$需满足一定的光滑性。这里的光滑性一般指的是
\begin{itemize}
\item $K$的边界曲面 $\partial K$是一个 $\mathbb {R}^n$中的正则子流形,且二次可微;
\item Gauss 映射 $\nu_K:\partial K\to \mathbb{S}^{n-1}$是一个微分同胚。
\end{itemize}
满足上述条件的凸体集合一般会记为 $C_+^2$。我们不深入探讨这些,简单起见,后文中我们都假设 $K$的表面足够光滑,使得曲率函数 $f_K$存在且非负,并且 $f_K$是连续的。

仿射表面积有其它等价定义,这里也非常简单地介绍一下。

\textbf{仿射表面积等价定义 1}: 设 $L\subset\mathbb R^n$是任意一个星形体, $0\in\mathrm{int}(L)$,则其半径函数 $\rho_L$恒正且连续,有

\begin{align*}
as(K)&=\int_{\mathbb S^{n-1}}f_K^{\frac n{n+1}}\mathrm du\\
&=\int_{\mathbb S^{n-1}}f_K^{\frac n{n+1}}\rho_L^{-\frac n{n+1}}\rho_L^{\frac n{n+1}}\mathrm du\\
&\le \left(\int_{\mathbb S^{n-1}}\left(f_K^{\frac n{n+1}}\rho_L^{-\frac n{n+1}}\right)^{\frac {n+1}n}\mathrm du\right)^{\frac n{n+1}}\left(\int_{\mathbb S^{n-1}}\left(\rho_L^{\frac n{n+1}}\right)^{n+1}\mathrm du\right)^{\frac 1{n+1}} \tag{Hölder 不等式}\\
&=\bigg(\int_{\mathbb S^{n-1}}\rho_L^{-1}\,\underbrace{\vphantom{\int}f_K\mathrm du}_{S(K,u)}\bigg)^{\frac n{n+1}}\bigg(\underbrace{\int_{\mathbb S^{n-1}}\rho_L^{n}(u)\,\mathrm du}_{nV_n(L)}\bigg)^{\frac1{n+1}}\\
&= \bigg(\underbrace{\int_{\mathbb S^{n-1}}\rho_L^{-1}\mathrm dS(K,u)}_{=:\,nV_1(K,L^\circ)}\bigg)^{\frac n{n+1}}\big(nV_n(L)\big)^{\frac1{n+1}}\\
&=nV_1(K,L^\circ)^{\frac n{n+1}}V_n(L)^{\frac1{n+1}}  
\end{align*}
这里形式上定义 $\displaystyle V_1(K,L^\circ):=\int_{\mathbb S^{n-1}}\rho_L^{-1}(u)\,\mathrm{d}S(K,u)$(这避免了定义 $L^\circ$)。
上式对所有星形体 $L$都成立,且 $f_K$连续时,不等式取等当且仅当 $f_K=\rho_L^{n+1}$。

不难注意到 $nV_1(K,L^\circ)^{\frac n{n+1}}V_n(L)^{\frac1{n+1}}$在 $L$的缩放变换下是不变的(幂次正好消掉),于是,当 $f_K$连续时,可以将 $as(K)$写作如下极值问题的解\footnote{E. Lutwak, 1991}:
\begin{align*}
\left(\frac1nas(K)\right)^{\frac {n+1}{n}}&=\inf\Big\{V_1(K,L^\circ)V_n(L)^{\frac1n}:\; 0\in\mathrm{int}L,\;L\text{ 是星形体 }\Big\} \\
&=\inf\Big\{V_1(K,L^\circ):V_n(L)=1,\; 0\in\mathrm{int}L,\;L\text{ 是星形体 }\Big\} \label{eq:6-11}\tag{6-11}
\end{align*}
(这个式子在第~\ref{chapter:5} 章的例~\ref{ex:petty_body} 中就引入过,这里算是回收了伏笔 :-)


\textbf{仿射表面积等价定义 2}: 由于之前的定义对 $K$有光滑性限制,人们后来得出了一个推广的仿射表面积定义,是这样的:对一般的凸体 $K\subset\mathbb R^n$,
\[
as(K):=\int_{\partial K}\kappa(x)^{\frac1{n+1}}\mathrm{d}\mathcal{H}^{n-1}(x)
\]
这里的 $\kappa(x)$是\emph{Gauss 曲率} (Gauss curvature),
\begin{itemize}
\item 由于 $K$的表面 (某种程度上) 是几乎处处二次可微的\footnote{Alexandrov 定理}, $\kappa(x)$在 $\partial K$上也几乎处处有定义。
\item 当 $K$表面光滑,使得 $f_K$存在且恒正时,有 $\displaystyle\kappa(x)=\frac1{f_K(\nu_K(x))}$。可以验证该定义与之前给出的定义相容:
\begin{align*}
\int_{\partial K}  f_K(\nu_K(x))^{-\frac1{n+1}}\mathrm{d}\mathcal{H}^{n-1}(x)&=\int_{\mathbb S^{n-1}}f_K(u)^{-\frac1{n+1}}\underbrace{\vphantom{\int}\mathrm{d}S(K,u)}_{f_K(u)\mathrm{d}u} \tag{换元 $u=\nu_K(x)$}\\
&=\int_{\mathbb{S}^{n-1}}f_K(u)^{\frac n{n+1}}(u)\,\mathrm du
\end{align*}
\end{itemize}

最后我们研究 $as(K)$在 $K$的变换下相应的结论。显然, $as(K)$在平移变换下是不变的,因为 $S(K,\cdot)$平移不变,自然 $f_K$和 $as(K)$也应如此。此外,我们后文中还会用到一个结论,

\begin{lemma}\label{lem:as_scaling}
设 $\lambda>0$,则 $as(\lambda K)=\lambda^{\frac{n(n+1)}{n+1}}as(K)$。
\end{lemma}

\begin{proof}
最简单的方法是用 \eqref{eq:6-11}。由 $S(\lambda K,\cdot)=\lambda^{n-1}S(K,\cdot)$ (命题~\ref{prop:surface_scaling}),故
\[
V_1(\lambda K,L^\circ):=\int_{\mathbb S^{n-1}}\rho_L^{-1}(u)\mathrm dS(\lambda K,u)=\lambda^{n-1}\int_{\mathbb S^{n-1}}\rho_L^{-1}(u)\mathrm dS(K,u)
\]
因此,根据 \eqref{eq:6-11},
\begin{align*}
\left(\frac1nas(\lambda K)\right)^{\frac {n+1}{n}}&=\lambda^{n-1}\left(\frac1nas(K)\right)^{\frac {n+1}{n}}\\
\implies as(\lambda K)&=\lambda^{\frac{n(n-1)}{n+1}}as(K)
\end{align*}
\end{proof}

\begin{remark}
    为什么这个名字叫``仿射"表面积呢?因为实际上有一个更一般的仿射不变性的结论:设 $T:\mathbb R^n\to \mathbb{R}^n$是一个仿射变换,那么 $as(TK)=\mathrm{det}(T)^{\frac{n-1}{n+1}}as(K)$。特别地,对任意的 $T\in\mathrm{SL}(n)$,都有 $as(TK)=as(K)$。读者可以尝试自证。
\end{remark}

\subsection{仿射等周不等式的证明}

主要思路就是用 Steiner 对称化,分三步走。

\textbf{第一步},先证明仿射表面积在 Steiner 对称化下是单调的,

\begin{lemma}\label{lem:as_monotone}
设凸体 $K\subset \mathbb{R}^n$,那么 $as(K)\le as(S_nK)$对任意 $u\in\mathbb{S}^{n-1}$成立。
\end{lemma}

\begin{proof}
回忆一下,在 Steiner 对称化中,我们会将 $K$投影至 $u^\perp$,得到 $P_uK$;随之,对任意 $x\in P_uK$,作 $u$方向过 $x$的直线,与 $K$交于两点;最后平移两点之间的线段使得 $x$位于其中点,达到对称化的效果。不失一般性,设 $u=e_n$是第 $n$个坐标向量,则 $u^\perp=\mathbb{R}^{n-1}$, 见图~\ref{fig:steiner_2}。则 (式~\eqref{eq:3-3}):
\begin{align*}
K&=\big\{(x,t):g(x)\le t\le f(x),x\in \mathbb{R}^{n-1}\big\}\\
S_uK&=\left\{(x,t):|t|\le \frac{f(x)-g(x)}2,x\in \mathbb{R}^{n-1}\right\}
\end{align*}

\begin{figure}[htbp]
\centering
\includegraphics[width=0.5\textwidth]{figures/steiner_2.png}
\caption{Steiner 对称化的示意图。}\label{fig:steiner_2}
\end{figure}

可以看出, $(x,f(x))\in\partial K$、 $(x,g(x))\in\partial K$,且 $f$是凹函数、 $g$是凸函数,因而几乎处处二次可微,我们可以认为二者包含了 $K$的边界信息。例如,将\emph{上图} (upper graph) $\{(x,f(x)):x\in\mathrm{int}(P_uK)\}$视作曲面,不难得出
\begin{itemize}
\item $(x,f(x))$处的法向量为 $\displaystyle\frac{(-\nabla f(x),1)}{\sqrt{1+\|\nabla f\|^2}}$;
\item $(x,f(x))$处的 Gauss 曲率为 $\kappa (z)=\frac{\mathrm{det}(-\nabla^2 f)}{\left(\sqrt{1+\|\nabla f\|^2}\right)^{n+1}}$\footnote{读者可以算下第一和第二基本型矩阵的行列式,从而得到上式;注意凸体的 Gauss 曲率一定是非负的。};
\item 以及,该曲面上的 Hausdorff 测度满足 $\mathrm{d}\mathcal{H}^{n-1}(z)=\sqrt{1+\|\nabla f(x)\|^2}\,\mathrm{d}x$\footnote{这个是曲面的第一基本型矩阵的行列式决定的。}
\end{itemize}

好了,把这几个公式用起来,得到
\begin{align*}
\mathrm{as}(K)&=\int_{\partial K}\kappa(z)^{\frac1{n+1}}\mathrm{d}\mathcal H^{n-1}(z)\\
&=\int_{\{(x,f(x)):x\in\mathrm{int}(P_uK)\}}\kappa(z)^{\frac1{n+1}}\mathrm{d}\mathcal H^{n-1}(z)+\int_{\{(x,g(x)):x\in\mathrm{int}(P_uK)\}}\kappa(z)^{\frac1{n+1}}\mathrm{d}\mathcal H^{n-1}(z)\\
&=\int_{\mathrm{int}(P_uK)}\left(\frac{\mathrm{det}(-\nabla^2f(x))}{(\sqrt{1+\|\nabla f(x)\|^2})^{n+1}}\right)^{\frac1{n+1}}\sqrt{1+\|\nabla f(x)\|^2}\mathrm{d}x\\
&\quad + \int_{\mathrm{int}(P_uK)}\left(\frac{\mathrm{det}(\nabla^2g(x))}{(\sqrt{1+\|\nabla g(x)\|^2})^{n+1}}\right)^{\frac1{n+1}}\sqrt{1+\|\nabla g(x)\|^2}\mathrm{d}x\\
&=\int_{\mathrm{int}(P_uK)}\left(\mathrm{det}(-\nabla^2f(x))^{\frac1{n+1}}+\mathrm{det}(\nabla^2g(x))^{\frac1{n+1}}\right)\mathrm{d}x \label{eq:6-12}\tag{6-12}
\end{align*}
完全类似地,用 $\frac{f-g}2$替换上式的 $f$、 $-\frac{f-g}2$替换上式的 $g$,有
\begin{align*}
as(S_uK)&=\int_{\mathrm{int}(P_uK)}2\mathrm{det}\left(\frac{\nabla^2g(x)-\nabla^2f(x)}{2}\right) ^{\frac1{n+1}}\mathrm{d}x \label{eq:6-13}\tag{6-13}
\end{align*}

我们用一个矩阵的不等式来比较上面两个式子 

\begin{lemma}[Minkowski 行列式不等式]\label{lem:minkowski_det}
对任意半正定对称 $n\times n$矩阵 $A,B$,有
\[
\mathrm{det}(A+B)^{\frac1n}\ge\mathrm{det}(A)^{\frac1n}+\mathrm{det}(B)^{\frac1n}
\]
等号成立当且仅当 $A=B$。\footnotemark
\end{lemma}
\footnotetext{这个某种程度上其实可以算是矩阵的 BM 不等式。读者可以参考 \url{https://mathoverflow.net/q/65430}等。}


\begin{corollary}\label{col:matrix_ineq}
对任意半正定对称 $(n-1)\times (n-1)$矩阵 $A,B$,
\[
\left(\mathrm{det}\left(\frac{A+B}{2}\right)\right)^{\frac1{n+1}}\ge \frac12\mathrm{det}(A)^{\frac1{n+1}}+\frac12\mathrm{det}(B)^{\frac1{n+1}}
\]
\end{corollary}

\begin{proof}
设 $X$是 $(n-1)\times (n-1)$矩阵,引理~\ref{lem:minkowski_det} 表明函数 $F(X):=\mathrm{det}(X)^{\frac1{n-1}}$是凹函数;再结合 $\varphi(x):=x^{\frac{n-1}{n+1}}$是凹函数且单调递增,二者复合 $\varphi\circ F$就也是凹函数 (见 Boyd 凸优化的 3.2.4 节),用 Jensen 不等式立得。
\end{proof}

把 $-\nabla ^2f$和 $\nabla^2g$代入推论~\ref{col:matrix_ineq} 里,立即可以推出
\[
2\mathrm{det}\left(\frac{\nabla^2g(x)-\nabla^2f(x)}{2}\right)^{\frac1{n+1}}\ge \mathrm{det}(-\nabla^2f(x))^{\frac1{n+1}}+\mathrm{det}(\nabla^2g(x))^{\frac1{n+1}}
\]
因此 \eqref{eq:6-13}式 $\geq$ \eqref{eq:6-12}式,正是我们想要的结论。
\end{proof}

\textbf{第二步}: 证明仿射表面积是上半连续的。其实根据式 \eqref{eq:6-11}——仿射表面积可以表示成下确界的形式,上半连续是自然成立的:

\begin{lemma}\label{lem:as_semi_cont}
若一列凸体 $K_j$依 Hausdorff 度量收敛到 $K$,记作 $K_j\to K$,那么
\[
\limsup_{j\to\infty}as(K_j)\le as(K)
\]
\end{lemma}

\textbf{第三步}: 准备就绪,用 Steiner 对称化!设 $K$在一列单位方向$\{u_j\}_{j=1}^\infty$下依 Hausdorff 度量收敛至球 (定理~\ref{thm:gross}),记 $K_j=S_{u_j}S_{u_{j-1}}\cdots S_{u_2}S_{u_1}K$,那么根据引理~\ref{lem:as_monotone},
\[
as(K)\le as(K_1)\le \cdots\le as(K_j)\le \cdots
\]
取极限
\begin{align*}
as(K)&\le \lim_{j\to\infty}as(K_j)\\
&\le as\left(\lim_{j\to\infty}K_j\right) \tag{引理~\ref{lem:as_semi_cont}}\\
&=as\left(\left(\frac{V_n(K)}{V_n(B_n)}\right)^{\frac1n}B_n\right)\\
&=\left(\frac{V_n(K)}{V_n(B_n)}\right)^{\frac{n-1}{n+1}}as(B_n) \tag{引理~\ref{lem:as_scaling}}
\end{align*}
这正是定理~\ref{thm:as_isoperimetric} 叙述的仿射等周不等式。

如前述,定理~\ref{thm:as_isoperimetric} 的等号刻画条件证明起来比较复杂,从略。

\subsection{与 BS 不等式的等价性}

回顾一下本章证明的两个结论:
\begin{itemize}
\item BS 不等式 (定理~\ref{thm:bs}): $V_n(K)V_n(K^\circ)\le V_n(B_n)^2$;
\item 仿射等周不等式 (定理~\ref{thm:as_isoperimetric}): $as(K)\le nV_n(B_n)^{\frac2{n+1}}V_n(K)^{\frac{n-1}{n+1}}$;
\end{itemize}
在证明定理~\ref{thm:bs}的过程中用到了仿射等周不等式(在式 \eqref{eq:6-8}),因而后者是蕴含前者的。更进一步,可以证明它们俩实际上等价。下面我们用 BS 不等式推一下仿射等周不等式。

其实这个特别容易。由平移不变性,我们把 $K$的 Santaló 点平移到原点。仍用~\eqref{eq:6-11},注意到取 $L=K^\circ$时 $\displaystyle V_n(K,L^\circ)=\int_{\mathbb{S}^{n-1}}h_K(u)\,\mathrm{d}S(K,u)=V_n(K)$,则
\begin{align*}
as(K)&=\inf\Big\{nV_1(K,L^\circ)^{\frac n{n+1}}V_n(L)^{\frac1{n+1}}:0\in\mathrm{int}(L),\;L\text{ 是星形体}\Big\}\\
&\le nV_n(K)^{\frac n{n+1}}V_n(K^\circ)^{\frac1{n+1}} \tag{取 $L=K^\circ$}\\
&=n\big(V_n(K)V_n(K^\circ)\big)^{\frac1{n+1}}V_n(K)^{\frac {n-1}{n+1}}\\
&\le nV_n(B_n)^{\frac2{n+1}}V_n(K)^{\frac {n-1}{n+1}} \tag{由 BS 不等式}
\end{align*}
正是仿射等周不等式,其取等仅当 BS 不等式取等,即 $K$是一个椭球。
 