\chapter{混合体积与变分公式}\label{chapter:4}

\section{表面积与混合体积}

回忆一下我们最开始的例~\ref{ex:minkowski_sum}, 以及图~\ref{fig:Minkowski_sum}。
设有边长为 $l$ 的正方形 $K=[-l/2,l/2]\times[-l/2,l/2]$、圆盘 $\epsilon B_2=\{(x,y):x^2+y^2\le\epsilon^2\}$, 显见
\begin{align*}
V_n(K+\epsilon B_2)&=V_n(K)+4l\epsilon+V_n(\epsilon B_2)\\
&=l^2+4l\epsilon+\pi\epsilon^2
\end{align*}
于是存在极限
\begin{align*}
\lim_{\epsilon\to 0}\frac{V_n(K+\epsilon B_2)-V_n(K)}{\epsilon}&=\lim_{\epsilon\to 0^+}\frac{4l\epsilon+\pi\epsilon^2}{\epsilon}\\
&=4l
\end{align*}
正是正方形的周长。

这个现象可以推广:对于一般的 $K\subset\mathbb{R}^n$, 我们也可以把它与 $\epsilon B_n$ 相加,然后把原本的 $K$ 给挖掉,见图~\ref{fig:eps_extension};
剩下的部分可以近似看成一个能够铺展开成``长方形", 其高是 $\epsilon$, 底则可以看作是 $K$ 的``周长 (perimeter)"——只不过这个``周长"是 $n-1$ 维的,所以叫做``表面积"更加合适。
 
\begin{figure}[h]
    \centering
    \includegraphics[width=0.7\textwidth]{figures/surface_1.pdf}
    \caption{$K + \epsilon B_n$ 等同于 $K$ 的边界向外扩展出 $\epsilon$, 因而可用 $\epsilon\to 0^+$ 时的极限来定义 $K$ 的表面积。}
    \label{fig:eps_extension}
\end{figure}

因此可以说
\begin{align*}
& V_n(K+\epsilon B_n)-V_n(K)\approx K\text{ 的表面积}\times \epsilon\\
\implies & \lim_{\epsilon\to 0^+}\frac{V_n(K+\epsilon B_n)-V_n(K)}{\epsilon}=K\text{ 的表面积}
\end{align*}

\begin{definition}[表面积]\label{def:surface_area}
设 $K\subset\mathbb{R}^n$ 是凸体,则定义其\emph{表面积} (surface area) 为
\[
S(K):=\lim_{\epsilon\to 0^+}\frac{V_n(K+\epsilon B_n)-V_n(K)}{\epsilon}
\]
\end{definition}

\begin{corollary}
设 $\lambda>0$, 那么 $S(\lambda K)=\lambda^{n-1}S(K)$。
\end{corollary}

\begin{proof}
\begin{align*}
S(\lambda K) &=\lim_{\epsilon\to0^+}\frac{V_n(\lambda K+\epsilon B_n)-V_n(\lambda K)}{\epsilon}\\
&=\lim_{\epsilon\to0^+}\frac{\lambda^{n-1}\big(V_n(K+\tfrac{\epsilon}{\lambda}B_n)-V_n(K)\big)}{\frac \epsilon \lambda}\\
&=\lambda^{n-1}S(K)
\end{align*}
\end{proof}

\begin{remark}
    可以证明,对于凸体,定义~\ref{def:surface_area} 中的极限总是存在的,并且正是凸体在测度意义上的表面积,即 $\mathcal{H}^{n-1}(\partial K)$。
    其中 $\mathcal{H}^{n-1}$ 代表 $n-1$ 维 Hausdorff 测度,$\partial K$ 代表 $K$ 的边界。
    更一般地,这个定义还可以再推广到一类表面 ``足够良好" 的紧集,有这样的定理\footnotemark
    \begin{theorem*}
        对紧集$K\subset\mathbb R^n$, 若其内部非空且边界是\emph{可求长的} (rectifiable), 那么 
        \[S(K)=\lim_{\epsilon\to 0^+}\frac{V_n(K+\epsilon B_n)-V_n(K)}{\epsilon}=\mathcal{H}^{n-1}(\partial K)\]
    \end{theorem*}
    有时,$S(K)$ 也称 \emph{Minkowski content}。
\end{remark}
\footnotetext{
    该定理的叙述,笔者参考的是 Böröczky, K. J., Figalli, A., \& Ramos, J. P. (2025).\emph{ Isoperimetric inequalities, Brunn-Minkowski theory and Minkowski type Monge-Ampère equations on the sphere} . \url{https://users.renyi.hu/~carlos/Brunn-Minkowski-Book-2024-02-05.pdf} 一书的 Theorem 4.1.4 (p. 102)。
    不过对此叙述最详尽的专著应属 H. Federer (1969). \textit{Geometric measure theory}. Springer-Verlag.
    % 关于 Hausdorff 测度的定义和性质,可以参考例如 Mattila, P. (1995). \textit{Geometry of sets and measures in Euclidean spaces: Fractals and rectifiability}. Cambridge University Press。
}

在表面积的定义中,球 $\epsilon B_n$ 是个很特殊的对象,它是``各项同性"的,能保证 $K+\epsilon B_n$ 总沿 $K$ 的边界扩展出 $\epsilon$。
我们不禁会想:如果把``球"换成一个一般的凸体,会得到什么样的结果?显然,此时边界各处拓展出去的距离就不一样了:
如果 $K,L$ 是两个一般的凸体,那么在 $K$ 的边界上, $K+\epsilon L$ 关于表面法方向 $u\in\mathbb{S}^{n-1}$ 至多能扩展出去 $\epsilon h_L(u)$, 其中 $h_L(u)$ 代表凸体 $L$ 的支撑函数。再把 $K$ 挖掉,剩下的部分展开后就不能用长方形近似了,因为它的``高"在不停变化,是一个锯齿形,如图~\ref{fig:surface_mix}。
 
\begin{figure}[hb]
    \centering
    \includegraphics[width=0.5\textwidth]{figures/surface_2.png}
    \caption{将定义~\ref{def:surface_area} 中的球换成任意的凸体 $L$ , $K$ 边界向外扩展出去的距离就不再总是相同的。
    底部蓝色线代表 $K$ 的表面,灰色代表`` $K+\epsilon B_n$ 和 $K+\epsilon L$ 扩展出去的部分"}
    \label{fig:surface_mix}
\end{figure}

我们可以猜测,当 $\epsilon\to 0$ 时, $(K+\epsilon L)\setminus K$ 的体积应该可以写作积分的形式
\[
V_n(K+\epsilon L)-V_n(K)\approx \int_{\mathbb{S}^{n-1}}\epsilon h_L(u)\mathrm{d}A(u)
\]
即,
\begin{equation}
\lim_{\epsilon \to 0^+}\frac{V_n(K+\epsilon L)-V_n(L)}{\epsilon}=\int_{\mathbb{S}^{n-1}} h_L(u)\mathrm{d}A(u) \label{eq:4-1}\tag{4-1}
\end{equation}
这个 $A(u)$ 代表上图右下锯齿形的``底边", 表述某种和 $K$ 的表面积有关的测度\footnote{因为当 $L=B_n$ 时, $h_L(u)\equiv 1$, 根据定义应该有 $\displaystyle \int_{\mathbb{S}^{n-1}}\mathrm{d}A(u)=S(K)$}。具体如何定义 $A(u)$ 暂且不表,我们后文中会给出上述等式的严格论述。

\begin{definition}[混合体积]\label{def:mix_volume}
设 $K,L\subset\mathbb{R}^n$ 是凸体,定义 $K$ 和 $L$ 的\emph{第一混合体积} (first mixed volume) 为
\[
V_1(K,L)=\frac1n\lim_{\epsilon\to 0^+}\frac{V_n(K+\epsilon L)-V_n(K)}{\epsilon}
\]
\end{definition}

根据定义, $K$ 的表面积可以写作 $S(K)=nV_1(K,B_n)$。

\section{Minkowski 第一不等式}

\begin{theorem}[Minkowski 第一不等式]\label{thm:minkowski}
设 $K,L\subset\mathbb{R}^n$ 是凸体,那么成立不等式
\[
V_1(K,L)\ge V_n(K)^{\frac{n-1}{n}}V_n(L)^{\frac 1n}
\]
不等号成立当且仅当 $K,L$ 位似,即存在 $c>0$ 和 $x_0\in\mathbb{R}^n$ 使得 $L=cK+x_0$。
\end{theorem}

\subsection{特例:等周不等式和 Urysohn 不等式}

在证明该定理之前,我们先看看这个定理能推导出什么结论。

\begin{corollary}[等周不等式]
注意到根据定义~\ref{def:surface_area}, 单位球的表面积和体积满足关系
\begin{align*}
S(B_n)&=nV_1(B_n,B_n)\\
&=\lim_{\epsilon \to 0^+}\frac{V_n((1+\epsilon)B_n)-V_n(B_n)}{\epsilon}\\
&=\left(\lim_{\epsilon \to 0^+}\frac{(1+\epsilon)^n-1}{\epsilon}\right)V_n(B_n)\\
&=nV_n(B_n)
\end{align*}
在定理~\ref{thm:minkowski} 中取 $L=B_n$ 为单位球,那么
\begin{align*}
S(K)=nV_1(K,B_n)&\ge n V_n(K)^{\frac{n-1}n}V_n(B_n)^{\frac 1n}\\
&=\color{DarkBlue}{nV_n(B_n)}\left(\frac{V_n(K)}{V_n(B_n)}\right)^{\frac{n-1}n}\\
&=\color{DarkBlue}{S(B_n)}\left(\frac{V_n(K)}{V_n(B_n)}\right)^{\frac{n-1}n}
\end{align*}
得出
\[
\left(\frac{S(K)}{S(B_n)}\right)^{\frac1{n-1}}\ge\left(\frac{V_n(K)}{V_n(B_n)}\right)^{\frac1{n}}
\]
取等当且仅当 $K$ 是一个球。
这个不等式表明了:表面积一定的凸体,其体积在该凸体为球时取得最大值;体积一定的凸体,表面积在该凸体为球时取得最小值。这便是著名的\emph{等周不等式} (Isoperimetric inequality)。
\end{corollary}

\begin{corollary}[Urysohn 不等式]
如果在定理~\ref{thm:minkowski}  中取 $K=B_n$ 为单位球,又会怎样?有
\begin{align*}
V_1(B_n,L)&\ge V_n(B_n)^{\frac{n-1}{n}}V_n(L)^{\frac1n}\\
&=V_n(B_n)\left(\frac{V_n(L)}{V_n(B_n)}\right)^{\frac 1n}
\end{align*}
即
\[
\frac{V_1(B_n,L)}{V_n(B_n)} \ge \left(\frac{V_n(L)}{V_n(B_n)}\right)^{\frac 1n}
\]
取等当且仅当 $L$ 是一个球。

上面的不等式左侧可以改写作
\begin{align*}
\frac{V_1(B_n,L)}{V_n(B_n)}&=\frac1{\color{DarkBlue}{nV_n(B_n)}}\int_{\mathbb{S}^{n-1}}h_L(u)\mathrm{d}\sigma(u)\\
&=\frac1{\color{DarkBlue}{S(B_n)}}\int_{\mathbb{S}^{n-1}}h_L(u)\mathrm{d}\sigma(u)\\
&:=W(L)
\end{align*}
其中 $\sigma$ 代表球面 $\mathbb{S}^{n-1}$ 上的面积测度。上式所定义的 $W(L)$ 称作凸体 $L$ 的\emph{均宽} (mean-width), 则有
\[
W(L)\ge\left(\frac{V_n(L)}{V_n(B_n)}\right)^{1/n}
\]
这被称作 \emph{Urysohn 不等式}。
\end{corollary}

\begin{remark}
上面两个不等式都可以用第~\ref{chapter:3} 章中介绍的 Steiner 对称化的技巧证明——回忆一下,我们曾在说过,如果不等式在球的情况下达到最大或最小值时,很可能就能利用 Steiner 对称化的技巧解决。
以等周不等式为例,我们只需证明 Steiner 对称化的另一性质:
\begin{proposition}
设 $K\subset\mathbb{R}^n$ 是凸体,则有 $S(S_uK)\le S(K)$。
\end{proposition}
\begin{proof}
用命题~\ref{prop:steiner_monotone} 和~\ref{prop:steiner_scaling}, 对 $\epsilon >0$,
\[
S_u(K+\epsilon B_n)\supseteq S_u K+\epsilon S_u(B_n)=S_u K+\epsilon B_n
\]
因而再结合体积不变 (命题~\ref{prop:steiner_volume}),
\begin{align*}
\frac{V_n(S_uK+\epsilon B_n)-V_n(S_uK)}{\epsilon} \le \frac{V_n(S_u(K+\epsilon B_n))-V_n(S_uK)}{\epsilon}=\frac{V_n(K+\epsilon B_n)-V_n(K)}{\epsilon}
\end{align*}
\end{proof}

该性质告诉我们表面积泛函 $S(K)$ 在 Steiner 对称化操作下是单调递减的,由 \hyperref[thm:gross]{Gross 定理} 以及表面积在 Hausdorff 度量下连续,立即可以证明等周不等式。
至于如何用 Steiner 对称化证明 Urysohn 不等式,我们留给读者 :-)
\end{remark}

\subsection{Minkowski 第一不等式的证明}

接下来我们用 BM 不等式来证明 Minkowski 第一不等式。在此之前,先证明 BM 不等式的一个推论:

\begin{lemma}\label{lem:concave}
设 $K,L\subset\mathbb{R}^n$ 是凸体,那么 $f(t):=V_n\big((1-t)K+tL\big)^{1/n}$,  $0\le t\le 1$ 是凹函数。
\end{lemma}

\begin{proof}
对任意 $0\le t_1,t_2\le 1$ 和 $0\le \lambda\le1$,
\begin{align*}
&f\big((1-\lambda)t_1+\lambda t_2\big)\\
&=V_n\Big((1-(1-\lambda)t_1-\lambda t_2)K+((1-\lambda)t_1+\lambda t_2)L\Big)^{1/n}\\
&=V_n\Big((1-\lambda)\big((1-t_1)K+t_1L\big)+\lambda\big((1-t_2)K+t_2L\big)\Big)^{1/n}\tag{定理~\ref{thm:BM-convex}}\\
&\ge (1-\lambda)V_n\big((1-t_1)K+t_1L\big)^{1/n}+\lambda V_n\big((1-t_2)K+t_2L\big)^{1/n}\\
&=(1-\lambda)f(t_1)+\lambda f(t_2)
\end{align*}
其中第三行用了 BM 不等式。
\end{proof}

\begin{remark}
事实上, $V_n\big((1-t)K+tL\big)$ 可以写作关于 $t$ 的 $n$ 次多项式。
更一般地,对于凸体 $K_1,\ldots,K_m\subset\mathbb{R}^n$ 和实数 $\lambda_1,\ldots,\lambda_m\ge 0$,  $V_n\left(\lambda_1K_1+\cdots+\lambda_mK_m\right)$ 可以写作 $\lambda_1,\ldots,\lambda_m$ 的 $n$ 次齐次多项式。
这个结论一般会写成
\[V_n(K+\lambda L)=\sum_{i=0}^n c_i \lambda^i\]
其中 $c_i$ 是某些与 $K,L$ 有关的非负常数。特别地,当 $L=B_n$ 时, 该式称作 \emph{Steiner 公式}。
详细的理论可以参考 Gruber, P. M. (2007). \textit{Convex and discrete geometry}. Berlin, Heidelberg: Springer Berlin Heidelberg 的 Theorem 6.5 等。
\end{remark}

\begin{proof}[Minkowski 第一不等式的证明]
第一混合体积 $V_1(K,L)$ 的定义式不太好用,为此,我们考虑改写其形式
\begin{align*}
V_1(K,L)&=\frac1n\lim_{\epsilon\to 0^+}\frac{V_n(K+\epsilon L)-V_n(K)}{\epsilon}\\
&=\frac1n\lim_{t\to 0^+}\frac{V_n\left(K+\frac{t}{1-t} L\right)-V_n(K)}{\frac{t}{1-t}} \tag{令 $\epsilon=\frac t{1-t}$}\\
&=\frac1n\lim_{t\to 0^+}\frac{1-t}{t}\bigg(\bigg(\frac 1{1-t}\bigg)^nV_n\big((1-t)K+tL\big)-V_n(K)\bigg)\\
&=\frac1n\lim_{t\to 0^+}\underbrace{\frac1{(1-t)^{n-1}}}_{\to 1}\frac1{t}\bigg(V_n\big((1-t)K+tL\big)-(1-t)^nV_n(K)\bigg)\\
&=\frac1n\lim_{t\to 0^+}\frac{V_n\big((1-t)K+tL\big)-V_n(K)}{t}+\frac1n V_n(K)\underbrace{\lim_{t\to 0^+}\frac{1-(1-t)^n}{t}}_{=n}\\
&=\frac1n\lim_{t\to 0^+}\frac{V_n\big((1-t)K+tL\big)-V_n(K)}{t}+V_n(K)
\end{align*}
因而,若记 $f(t):= V_n\big((1-t)K+tL\big) ^{1/n}$,  $0\le t\le1$, 则上面的结果可写作
\begin{align*}
V_1(K,L)-V_n(K)&=\frac1n\lim_{t\to 0^+}\frac{V_n\big((1-t)K+tL\big)-V_n(K)}{t}\\
&=\frac1n \frac{\mathrm{d}f^n}{\mathrm{d}t}(0^+)\\
&=f^{n-1}(0)f'(0^+)
\end{align*}

我们知道 $f^{n-1}(0)=V_n(K)^{\frac{n-1}{n}}$, 于是,要证 Minkowski 第一不等式即 $V_1(K,L)\ge V_n(K)^{\frac{n-1}{n}}V_n(L)^{\frac 1n}$ 成立,只需证明
\begin{align*}
f'(0^+)&=\frac{V_1(K,L)-V_n(K)}{V_n(K)^{\frac{n-1}n}}\\
&\ge \frac{V_n(K)^{\frac{n-1}n} V(L)^{\frac1n}-V_n(K)}{V_n(K)^{\frac{n-1}n}}\\
&=V_n(L)^{\frac1n}-V_n(K)^{\frac1n}\\
&=f(1)-f(0)
\end{align*}

而根据引理~\ref{lem:concave},
\begin{align*}
& f(t)=f((1-t)\cdot 0+t\cdot 1)\ge (1-t)f(0)+t f(1),\;\forall\, 0<t<1\\
\implies&\frac{f(t)-f(0)}{t}\ge f(1)-f(0),\;\forall\, 0<t<1\\
\overset{t\to0^+}{\implies}& f'(0^+) \ge f(1)-f(0)
\end{align*}
这就证明了 Minkowski 第一不等式。

下证取等条件。一方面 (充分性), 若存在 $c>0$ 和 $x_0\in\mathbb{R}^n$ 使得 $L=cK+x_0$, 则
\begin{align*}
& V_n(K+\epsilon L)=V_n(K+\epsilon cK+\epsilon x_0)=(1+\epsilon c)^nV_n(K)\\
\implies & V_1(K,L)=\frac1n\lim_{\epsilon\to 0^+}\frac{ V_n(K+\epsilon L)-V_n(K)}{\epsilon}\\
& \quad =\frac1n V_n(K)\lim_{\epsilon\to 0^+}\frac{(1+\epsilon c)^n-1}{\epsilon}\\
& \quad =cV_n(K)\\
& \quad =V_n(K)^{\frac{n-1}n}(c^n V_n(K))^{\frac1n}\\
& \quad =V_n(K)^{\frac{n-1}n}V_n(L)^{\frac1n}
\end{align*}
即 Minkowski 第一不等式取等。

反过来 (必要性), 若取等,则 $f'(0^+)=f(1)-f(0)$ 需成立,由引理~\ref{lem:concave},  $f$ 必须是一个线性函数,即 $f(t)=(1-t)f(0)+tf(1),\;\forall\, 0\le t\le 1$, 这正是 BM 不等式取等。故 Minkowski 第一不等式取等蕴含 BM 不等式的取等条件,即 $K,L$ 位似。
\end{proof}

\section{混合体积的变分公式}

接下来我们回到~\eqref{eq:4-1}式,来解决这样一个问题: $V_1(K,L)$ 的积分形式应该是什么?或者说,要使~\eqref{eq:4-1}式: $\displaystyle V_1(K,L)=\int_{\mathbb{S}^{n-1}} h_L(u)\mathrm{d}A(u) $ 成立,这个 $A(u)$ 到底应该是什么东西?

\subsection{准备工作}

为此,我们得先补充一些定义,

\begin{definition}[支撑函数和超平面]
设 $K\subset\mathbb{R}^n$ 是凸体, $u\in\mathbb{S}^{n-1}$, 则称 $h_K(u):=\sup\limits_{x\in K} \langle x,u\rangle$ 是 $K$ 的\emph{支撑函数} (support function),  $H_K(u):=\{x\in\mathbb{R}^n:\langle x,u\rangle=h_K(u)\}$ 是 $K$ 的一个\emph{支撑超平面} (support plane)。
\end{definition}

\begin{definition}[法向量]
设 $K\subset\mathbb{R}^n$ 是凸体, $\partial K$ 是其边界,若对 $x\in\partial K$, 方向向量 $u\in\mathbb{S}^{n-1}$ 满足 $\langle x,u\rangle=h_K(u)$, 即 $u$ 是过 $x$ 点的支撑超平面的法向量,则称 $u$ 是曲面 $\partial K$ 在 $x$ 处的\emph{法向量} (normal vector)。
\end{definition}

我们上述定义的法向量不一定唯一:在某个 $x\in\partial K$ 处可能存在多个法向量。如下左图所示,在左上方的尖点处会存在多个法向量 (事实上这些法向量构成一个闭锥,称作\emph{法锥} normal cone), 见~\ref{fig:normal_radial} 左图。

同时,给定法向量 $u$, 其对应的 $x$ 也可能不唯一 (例如可能 $K$ 的一个``面"都有相同的法向量) 。

\begin{figure}[h]
    \centering
    \includegraphics[width=0.8\textwidth]{figures/normal_radial.pdf}
    \caption{左图:凸体 $K$ 上某些点可能存在多个法向量 (如尖点处);给定法向量 $u$, 其对应的边界点也可能不唯一 (如平面部分)。
    右图:从原点出发,沿方向 $v$ 走到边界 $\partial K$ 上的点 $r_K(v)$, 走过距离为 $\rho_K(v)$。}
\label{fig:normal_radial}
\end{figure}

\begin{definition}[Gauss 映射和其逆]
设 $x\in\partial K$, 定义集合 $\nu_K(x)$: $u\in\nu_K(x)$ 当且仅当 $\langle x,u\rangle =h_K(u)$, 即 $u$ 是 $x$ 处的法向量;称 $\nu_K$ 是 \emph{Gauss 映射}, 它将 $\partial K$ 上的点映射到 $\mathbb{S}^{n-1}$ 上的点。
反过来,对于给定的 $u\in\mathbb{S}^{n-1}$, 定义集合 $\nu_K^{-1}(u)$: $x\in \nu_K^{-1}(u)$ 当且仅当 $\langle x,u\rangle =h_K(u)$, 称其为 \emph{Gauss 逆映射}, 它将 $\mathbb{S}^{n-1}$ 上的点映射到  $\partial K$。
\end{definition}

这里我们把 $\nu_K$ 和 $\nu_K^{-1}$ 都定义为集合,原因正如我们刚刚所说,在某些点处它们可能不唯一;不过实际上,这种点都是零测的 (相对于后面讨论的球面/表面积测度)。换句话说,这两个映射是``几乎处处"单值的,我们无需担心多值造成的影响。

\begin{remark}
简而言之,对于任意一个凸体 $K$, 我们总能将其边界 $\partial K$ 划分成有限个部分,使得每个部分 (在合适坐标系下) 都是一个凸函数。\emph{Aleksandrov 定理}告诉我们凸函数几乎处处可微,因而我们可以说, $\partial K$ 上几乎处处每一个点存在唯一的支撑超平面以及法向量 (这种点又叫做正则点), 这说明我们可以把 $\nu_K$ 视作是几乎处处单值的。

反过来也一样, $h_K$ 是一个凸函数,因而几乎处处可微。凸几何中有这样的结论: $h_K(u)$ 在 $u$ 处可微当且仅当支撑超平面 $H_K(u)$ 与 $K$ 的交集 (称作支撑集) 仅有一个点 , 即 $H_K(u)\cap K=\{x\}$ \footnotemark, 这说明我们可以把 $\nu_K^{-1}$ 视作是几乎处处单值的。
% 由于那些性质不好的、造成映射多值的点也不会产生影响,我们简单起见,
后文中均将 $\nu_K(u)$ 和 $\nu^{-1}_K(x)$ 视作是唯一定义的。
\end{remark}
\footnotetext{见 Schneider, R. (2013). \textit{Convex bodies: the Brunn–Minkowski theory} (Vol. 151). Cambridge university press. 的 Corollary 1.7.3, p.47}

\begin{definition}[半径函数]\label{def:radial_func}
设 $0\in K$, 定义 $K$ 的\emph{半径函数} (radial function) $\rho_K:\mathbb{S}^{n-1}\to [0,\infty)$ 为从原点出发,在 $K$ 内沿 $v$ 方向的最大距离,
\[
\rho_K(v):=\sup\{r>0:rv\in K\}
\]
定义 $r_K(v):=\rho_K(v)v\in\partial K$, 这是一个 $\mathbb{S}^{n-1}\to\partial K $ 的函数。
\end{definition}

上述定义亦见于~\ref{fig:normal_radial} 右图。本章接下来均假设原点 $0\in K$。

\begin{definition}[半径 Gauss 映射及其逆]\label{def:radial_gauss_map}
记  $\alpha_K:=\nu_k\circ r_K$ 为\emph{半径 Gauss 映射}。

记 $r_K^{-1}:\partial K\to\mathbb{S}^{n-1}$ 为
\[
r_K^{-1}(x):=\frac{x}{\|x\|},\quad x\ne 0
\]

\emph{半径 Gauss 逆映射} 定义为 $\alpha_K^{-1}:=(\nu_k\circ r_K)^{-1}=r_K^{-1}\circ \nu_K^{-1}$。
\end{definition}

通俗地说,从原点出发,向 $u$ 方向一直走到 $K$ 的边界,在边界上这个点处的法向量就是 $\alpha_K(u)$。反过来,如果 $u$ 是边界上某个点的法向量,那么将这个点归一化,所得方向向量就是 $\alpha_K^{-1}(u)$。 $\alpha_K$ 和 $\alpha_K^{-1}$ 都是 $\mathbb{S}^{n-1}\to\mathbb{S}^{n-1}$ 的映射。

如前对 Gauss 映射和其逆的讨论,我们可以将 $\alpha_K(u)$ 和 $\alpha_K^{-1}(u)$ 视作是几乎处处单值的。

\subsection{关于凸体的体积}

首先,可以将凸体 $K\subset\mathbb{R}^n$ 的体积写作积分的形式,并应用极坐标变换,有
\begin{align*}
V_n(K)&=\int_K\mathrm{d}x=\int_{\mathbb{S}^{n-1}}\int_0^{\rho_K(v)}r^{n-1}\mathrm{d}r\mathrm{d}v\\
&=\frac1n\int_{\mathbb{S}^{n-1}}(\rho_K(v))^n\mathrm{d}v \label{eq:4-2}\tag{4-2}
\end{align*}

另一方面,记 $\displaystyle\mathrm{div}(x)=\sum_{i=1}^n\frac{\partial x_i}{\partial x_i}=n$ 代表 $x$ 的散度, $\mathcal H^{n-1}(x)$ 是 $\partial K$ 上的 $n-1$ 维 Hausdorff 测度 (边界上的表面积微元) , 我们还能将体积写成
\begin{align*}
V_n(K)&=\int_K\mathrm{d}x=\frac1n\int_Kn\,\mathrm{d}x\\
&=\frac1n\int_K\mathrm{div}(x)\,\mathrm{d}x\\
&=\frac1n\int_{\partial K}\langle x,\nu_K(x)\rangle\,\mathrm{d}\mathcal H^{n-1}(x) \tag{散度定理}\\
&=\frac1n\int_{\mathbb{S}^{n-1}}h_K(u)\mathrm{d}S(K,u) \label{eq:4-3}\tag{4-3}
\end{align*}

这里最后一行用 $u=\nu_K(x)$ 换元后,Gauss 映射把 $\mathcal{H}^{n-1}(x)$ 推送到了一个新的测度 $\mathrm{d}S(K,u)$, 可以算出来是
\[
S(K,U):=\int_{\{x\in\partial K:x\in\nu^{-1}(u),u\in U\}}\mathrm{d}\mathcal H^{n-1}=\mathcal H^{n-1}(\nu_K^{-1}(U))
\]
其中 $U$ 是球面 $\mathbb{S}^{n-1}$ 中的 Borel 集。

我们称 $\mathrm{d}S(K,u)$ 是 $K$ 的\emph{表面积测度}, 由上可知,它由Gauss映射和 $\partial K$ 上的 $n-1$ 维 Hausdorff 测度诱导而来。什么意思呢?请看图~\ref{fig:gauss_map} 中的例子。

\begin{figure}[h]
\centering
\includegraphics[width=0.8\textwidth]{figures/gauss_map.pdf}
\caption{Gauss 逆映射 $\nu_K^{-1}$ 的示意图。对于方向 $u_1$, 我们发现 $K$ 中有一个``面"即蓝色线段,上面每一个点的法向量都是 $u_1$, 于是这条蓝色线段就是 $\nu^{-1}_K(u_1)$, 其长度就是 $S(K,u_1)$;对于圆周上的一段橙色弧 $U$, 对应到 $K$ 的边界上,也能找到一段橙色部分,其中点的法向量在 $U$ 中,这段橙色部分的长度就是 $S(K,U)$。}
\label{fig:gauss_map}
\end{figure}

式~\eqref{eq:4-2}和式~\eqref{eq:4-3}为我们带来了两个测度:

\begin{itemize} 
\item (用半径函数) $\frac1n\rho_K^n (v)\mathrm{d}v$, 其中 $\mathrm{d}v$ 代表球面测度;
\item (用支撑函数) $\frac1n h_K(u)\mathrm{d}S(K,u)$。
\end{itemize}


\begin{figure} 
\centering
\includegraphics[width=0.7\textwidth]{figures/measure_comp.pdf}
\caption{两个不同测度的比较:$\frac1n\rho_K^n (v)\mathrm{d}v$ 计算扇形面积, $\frac1n h_K(u)\mathrm{d}S(K,u)$ 计算三角形面积。}
\label{fig:measure_comp}
\end{figure}
% \hfill\break

举个二维的例子,见图~\ref{fig:measure_comp}, 可以这样去理解:设 $u$ 是 $\partial K$ 上某一点的法向量,此点附近面积微元与原点所围成的部分能近似为一个扇形 (高维情况下则是``锥") , 上面第一个测度相当于用扇形面积来近似这部分面积;同时这部分也能视作一个三角形,第二个测度相当于用`` $\frac1n$×底面积×高"来近似这部分面积—— $h_K(u)$ 就是高,而 $S(K,u)$ 就是那个底面积。
一个重要的观察是,这两个测度可以通过半径 Gauss 映射和其逆映射联系起来:

\begin{corollary}\label{col:map_diagram}
若作变换 $u=\alpha_K(v)$, 则可将测度 $\frac1n\rho_K^n \mathrm{d}v$ 变为 $\frac1n h_K\mathrm{d}S(K,u)$;反之若作变换 $v=\alpha_K^{-1}(u)$, 则可将测度 $\frac1n h_K\mathrm{d}S(K,u)$ 变回 $\frac1n\rho_K^n \mathrm{d}v$。 
\end{corollary}
\begin{figure}[h]
\centering
\includegraphics[width=0.6\textwidth]{./figures/map_diagram.pdf}
\caption{推论~\ref{col:map_diagram} 的示意图: $\nu_K$、$r_K$ 和 $\alpha_K$ 三者的关系。}
\end{figure}   
后面定理~\ref{thm:var_wulff} 的证明中我们会用到这个代换。

\subsection{推论:Minkowski 第一不等式等价于 BM 不等式}

接下来我们终于能够回答一开始的问题了,答案是这样的

\begin{theorem}[混合体积的变分公式]\label{thm:var}
设 $K,L\subset\mathbb{R}^n$ 是凸体,有
\begin{equation}
V_1(K,L)=\frac1n\int_{\mathbb{S}^{n-1}}h_L(u)\mathrm{d}S(K,u) \label{eq:4-4}\tag{4-4}
\end{equation}
\end{theorem}

通俗地说,这个式子说的是:``第一混合体积" $V_1(K,L)$ 可以视作是用 $K$ 做``底面积", 用 $L$ 做``高"算出来的体积。值得注意的是,测度 $\mathrm{d}S(K,u)$ 是仅依赖于 $K$ 的,而 $L$ 则是任意的。

证明放在最后,我们先看它的一个推论。

\begin{corollary}\label{cor:minkowski-bm}
Minkowski 第一不等式等价于 Brunn-Minkowski 不等式。
\end{corollary}

\begin{proof}
后者推前者已证 (定理~\ref{thm:minkowski} ), 现用 Minkowski 第一不等式推 BM 不等式,首先,我们知道集合的 Minkowski 和总对应着支撑函数的和:
\begin{align*}
h_K(u)+h_L(u)&=\sup_{x\in K}\langle x,u\rangle+\sup_{y\in L}\langle y,u\rangle\\
&=\sup\big\{\langle x+y,u\rangle:x\in K,y\in L\}\\
&=\sup\big\{\langle z,u\rangle:z\in K+L\}\\
&=h_{K+L}(u)
\end{align*}
于是由式~\eqref{eq:4-4}有
\begin{align*}
&V_1(K+L,K)+V_1(K+L,L)\\
=&\frac1n\int_{\mathbb S^{n-1}}h_K(u)\mathrm{d}S(K+L,u)+\frac1n\int_{\mathbb S^{n-1}}h_L(u)\mathrm{d}S(K+L,u)\\
=&\frac1n\int_{\mathbb S^{n-1}}h_{K+L}(u)\mathrm{d}S(K+L,u)\\
=&V_n(K+L)
\end{align*}
而根据 Minkowski 第一不等式有
\begin{align*}
V_1(K+L,K)&\ge V_n(K+L)^{\frac{n-1}{n}}V_n(K)\\
V_1(K+L,L)&\ge V_n(K+L)^{\frac{n-1}{n}}V_n(L) \tag{$*$}
\end{align*}
两式相加,立即得到
\[
V_n(K+L)=V_1(K+L,K)+V_1(K+L,L)\ge V_n(K+L)^{\frac{n-1}{n}}\big(V_n(K)^{\frac1n}+V_n(L)^{\frac1n}\big)
\]
正是 BM 不等式的等价形式 (定理~\ref{thm:BM-equivalence})。等号成立当且仅当 $(*)$ 两式都取等,即 $K$ 位似于 $K+L$ 且 $L$ 位似于 $K+L$, 亦即, $K$ 与 $L$ 位似。
\end{proof}

\subsection{Wulff 形}

相比于~\eqref{eq:4-4}, 我们将证明一个更一般的结论。不过在这之前,还要介绍一个概念。

回忆一下,我们在第一篇文章中就曾介绍过:可以用半空间的交来构造凸集
\[
K=\bigcap_{u\in\mathbb{S}^{n-1}}\{x\in\mathbb{R}^n:\langle x,u\rangle \le h_K(u)\}
\]

这种思想可以推广到函数$f:\mathbb{S}^{n-1}\to (0,\infty)$, 只需将上式中的 $h_K(u)$ 替换成 $f(u)$, 即可得到它的 \emph{Wulff 形} (Wulff shape)
\begin{definition}[Wulff 形]\label{def:wulff}
对连续正值函数 $f:\mathbb{S}^{n-1}\to (0,\infty)$ , 定义其 \emph{Wulff 形} 为
\[
[f]:=\bigcap_{u\in\mathbb{S}^{n-1}}\{x\in\mathbb{R}^n:\langle x,u\rangle \le f(u)\}
\]
    
\end{definition}

图~\ref{fig:wulff} 是一个 Wulff 形的形象例子。
 
\begin{figure}
    \centering 
    \includegraphics[width=1.0\textwidth]{./figures/wulff/Wulff.png}
    \caption{$f(\theta) = 1+0.45 \cos( 3\theta) + 0.15 \cos\theta$ 的 Wulff 形 (左) 和图像 (右), 五张图分别对应 $\theta=0,\frac{\pi}{3},\frac{2\pi}{3}, \pi, \frac{4\pi}{3}, \frac{5\pi}{3}$ 的情形。}
    \label{fig:wulff}
\end{figure}


我们总结一下 Wulff 形的性质:

\begin{proposition}\label{prop:wulff_convex}
$[f]$ 是一个凸体。
\end{proposition}

\begin{proof}
由于 $[f]$ 可以表示为下半空间的交,因而是一个闭凸集;进一步,由于 $f$ 连续,因而其在紧集上有界,故 $[f]$ 是凸体。
\end{proof}

\begin{proposition}\label{prop:wulff_origin}
由于 $f>0$, 可以注意到 $0\in\mathrm{int}[f]$ 也是成立的。
\end{proposition}

\begin{proposition}\label{prop:wulff_support}
记 Wulff 形 $[f]$ 的支撑函数为 $h_{[f]}$, 那么在 $\mathbb{S}^{n-1}$ 上有 $h_{[f]}\le f$。
\end{proposition}
\begin{remark}
    这个其实从图~\ref{fig:wulff} 的例子里很容易看出来,由于 $[f]$ 是用下半空间 $H_{u,f(u)}^-:=\{x:\langle x,u\rangle \le f(u)\}$ 的交定义的,显然生成的凸体之支撑函数必然不超过 $f$。同时有这样的结论: $[f]$ 是最大的凸体使得其支撑函数满足 $h_{[f]}\le f$。
\end{remark}

\begin{proposition}\label{prop:wulff_measure}
对 Wulff 形的表面测度 $\mathrm{d}S([f],u)$,  $h_{[f]}=f$ 几乎处处成立。
\end{proposition}

\begin{proof}
首先,对于任意 $x\in\partial [f]$, 它必须位于至少一个半空间 $H_{u,f(u)}^-$ 的边界上 (否则的话 $x$ 必为内点,不会在 $\partial [f]$) , 即,至少存在一个 $u\in\mathbb{S}^{n-1}$ 使得 $\langle x,u\rangle =f(u)$;如我们之前所说,在凸体的边界上,相对于其表面积测度,Gauss 映射几乎处处为单值,于是 $\partial [f]$ 上几乎所有的点都有唯一的法向量——对这些点有 $\langle x,u\rangle =h_{[f]}(u)$。这样一来,对几乎所有的 $u$,  $h_{[f]}(u)=\langle x,u\rangle=f(u)$。
\end{proof}

\begin{proposition}\label{prop:wulff_eq}
若 $K$ 是凸体且 $0\in\mathrm{int}(K)$, 则 $[h_K]=K$。
\end{proposition}

最后,注意到
\begin{align*}
[f]&=\bigcap_{u\in\mathbb{S}^{n-1}}\big\{y\in\mathbb{R}^n:\langle y,u\rangle\le f(u)\big\}\\
&=\bigcap_{u\in\mathbb{S}^{n-1}}\left\{y\in\mathbb{R}^n:\left\langle y,\frac{u}{f(u)}\right\rangle\le 1\right\}\\
&=\left\{y\in\mathbb{R}^n:\text{ 对任意 }u\in\mathbb{S}^{n-1}\text{ 有 }\langle y,z\rangle \le 1,z:=\frac{u}{f(u)}\right\}
\end{align*}
如果我们定义函数 $g:\mathbb{S}^{n-1}\to (0,\infty)$ 的\emph{凸包} (convex hull) 为
\[
\langle g\rangle:=\mathrm{Conv}\big(\{g(u)u:u\in\mathbb{S}^{n-1}\}\big)
\]
那么有
\begin{align*}
[f]&=\left\{y\in\mathbb{R}^n:\langle y,z\rangle\le 1,\forall z\in\left\langle\frac1f\right\rangle\right\}\\
&=\left\langle\frac1f\right\rangle^\circ
\end{align*}
即: 

\begin{proposition}\label{prop:wulff_polar} 
$f$ 的 Wulff 形是 $\langle 1/f\rangle$ 的极集。
\end{proposition}

后文中性质~\ref{prop:wulff_polar} 需要配合如下两个引理使用,

\begin{lemma}[双极定理 (Bipolar theorem) 的一个特例]\label{lem:bipolar}
设 $K\subset\mathbb{R}^n$ 是凸体, $0\in\mathrm{int}(K)$, 那么 $K^{\circ\circ}=K$。(其证明已在第~\ref{chapter:1}章中提及过。)
\end{lemma}

\begin{lemma}\label{lem:support_radial}
设 $K\subset\mathbb{R}^n$ 是凸体,$0\in\mathrm{int}(K)$, 那么 $h_{K^\circ}=\frac1{\rho_K}$、 $\rho_{K^\circ}=\frac1{h_K}$。
\end{lemma}

\begin{proof}
只需用定义,
\begin{align*}
\rho_{K^\circ}(v)&=\inf\{r>0:rv\in K^\circ\}\\
&= \inf\{r>0:\langle x,rv\rangle\le 1,\forall x\in K\}\\
&=1/\big(\sup\{r^{-1}>0:\langle x,v\rangle\le r^{-1},\forall x\in K\}\big)\\
&=1/\Big(\sup_{x\in K}\langle x,v\rangle\Big)\\
&=1/h_K(x)
\end{align*}
根据引理~\ref{lem:bipolar}, 我们可以用 $K^\circ$ 取代上式中的 $K$, 得到 $\rho_K=\rho_{K^{\circ\circ}}=1/h_{K^\circ}$。
\end{proof}

\subsection{变分公式的证明}
我们本节证明一个更加一般的结论
\begin{theorem}\label{thm:var_wulff}
设 $f:\mathbb{S}^{n-1}\to(0,\infty)$ 是一个正的连续函数, $K\subset\mathbb{R}^n$ 为凸体,且 $0\in\mathrm{int}(K)$。记
\[
f_\epsilon:=h_K+\epsilon f,\quad \epsilon >0
\]
那么
\begin{equation}
\frac1n\lim_{\epsilon\to 0^+}\frac{V_n([f_\epsilon])-V_n(K)}{\epsilon}=\frac1n\int_{\mathbb{S}^{n-1}}f(u)\mathrm{d}S(K,u) \label{eq:4-5}\tag{4-5}
\end{equation}
\end{theorem}

\begin{remark}
\begin{itemize}
\item 定理~\ref{thm:var} 中,不失一般性设 $0\in\mathrm{int}(K)$ 且 $0\in\mathrm{int}(L)$, 则变分公式~\eqref{eq:4-4} 是定理~\ref{thm:var_wulff} 式~\eqref{eq:4-5}在取 $f=h_L$ 时的特例。
\item 由于 $0\in\mathrm{int}K$,  $h_K(u)>0$ 总成立,我们实际上能把 $f$ 的值域放宽至 $\mathbb{R}$——因为总能选取 $\epsilon>0$ 足够小使得 $f_\epsilon>0$ 成立;实际上可以证明定理的结论此时也是成立的。
\end{itemize}
\end{remark}

\begin{proof}
为了叙述清晰,我们将证明分为 6 步。
\begin{enumerate}[label=\textbf{第\chinese*步}:, leftmargin=4em]
\item 由于 $[f_\epsilon]$ 是凸体且 $0\in\mathrm{int}[f_\epsilon]$, 应有 $\rho_{[f_\epsilon]}(v)>0$ 对一切 $v\in\mathbb{S}^{n-1}$ 都成立, 且有
\[
V_n([f_\epsilon])=\frac1n\int_{\mathbb{S}^{n-1}}\rho^n_{[f_\epsilon]}(v)\mathrm{d}v
\]

另一方面,由 Wulff 形性质~\ref{prop:wulff_polar} 和引理~\ref{lem:support_radial}, 有 $\rho_{[f_\epsilon]}=\rho_{\langle 1/f_\epsilon\rangle^\circ}=1/h_{\langle 1/f_\epsilon\rangle}$, 带入上式,得到
\[
V_n([f_\epsilon])=\frac1n\int_{\mathbb{S}^{n-1}}\frac1{h^{n}_{\langle 1/f_\epsilon\rangle} (u)}\mathrm{d}u
\]

欸, $1/f_\epsilon$ 写起来太不方便了,我们不妨把它记作 $g_\epsilon:=1/f_\epsilon$, 这样一来就有  $\displaystyle V_n([f_\epsilon])=\frac1n\int_{\mathbb{S}^{n-1}}\frac1{h^n_{\langle g_\epsilon\rangle}(u)}\mathrm{d}u$,  且
\begin{align*}
&g_0=\frac1{f_0}=\frac1{h_K}=\rho_{K^\circ}\\
\implies \; &\langle g_0\rangle=K^\circ\\
\implies \;& V_n(K)=\int_{\mathbb{S}^{n-1}}\rho^n_K(v)\mathrm{d}v=\int_{\mathbb{S}^{n-1}}\frac1{h_{\langle g_0\rangle}^n(u)}\mathrm{d}u
\end{align*}

\eqref{eq:4-5} 中左侧的极限就可以改写成
\begin{equation}
\lim_{\epsilon\to0^+}\frac{V_n([f_\epsilon])-V_n(K)}{\epsilon}=\lim_{\epsilon\to 0^+}\frac1n\int_{\mathbb{S}^{n-1}}\frac{h^{-n}_{\langle g_\epsilon\rangle}(u)-h^{-n}_{\langle g_0\rangle}(u)}{\epsilon}\mathrm{d}u \label{eq:4-6}\tag{4-6}
\end{equation}

\item 选定 $u\in\mathbb{S}^{n-1}$, 我们把目光放在  $\frac{h^{-n}_{\langle g_\epsilon\rangle}(u)-h^{-n}_{\langle g_0\rangle}(u)}{\epsilon}$ 上面,先来看一下它在 $\epsilon\to 0^+$ 的极限是什么,后面可以想办法用控制收敛定理交换~\eqref{eq:4-6}中极限与积分。

由于 $g_\epsilon$ 一致收敛于 $g_0$,  $\epsilon\to 0^+$, 所以在 Hausdorff 度量下 $\langle g_\epsilon\rangle\to \langle g_0\rangle$, 可得 $h_{\langle g_\epsilon\rangle}\to h_{\langle g_0\rangle}$ 一致收敛,有
\begin{align*}
&\lim_{\epsilon\to 0^+}\frac{h^{-n}_{\langle g_\epsilon\rangle}(u)-h^{-n}_{\langle g_0\rangle}(u)}\epsilon\\
&=\lim_{\epsilon\to 0^+}\frac{h^{-n}_{\langle g_\epsilon\rangle}(u)-h^{-n}_{\langle g_0\rangle}(u)}{\color{DarkBlue}{h_{\langle g_\epsilon\rangle}(u)-h_{\langle g_0\rangle}(u)}}\frac{\color{DarkBlue}{h_{\langle g_\epsilon\rangle}(u)-h_{\langle g_0\rangle}(u)}}{\epsilon}\\
&=-nh_{\langle g_0\rangle}^{-n-1}(u)\lim_{\epsilon\to 0^+}\frac{h_{\langle g_\epsilon\rangle}(u)-h_{\langle g_0\rangle}(u)}{\epsilon} \label{eq:4-7}\tag{4-7}
\end{align*}

现在问题又变成了算 $\displaystyle \lim_{\epsilon\to 0^+}\frac{h_{\langle g_\epsilon\rangle}(u)-h_{\langle g_0\rangle}(u)}{\epsilon}$。

首先,取定 $\epsilon>0$,
\begin{align*}
h_{\langle g_\epsilon\rangle}(u)&=\max_{x\in\langle g_\epsilon\rangle}\langle x,u\rangle\\
&=\max\Big\{\langle x,u\rangle:x\in\mathrm{Conv}\big(\{g_\epsilon(v)v,v\in\mathbb{S}^{n-1}\}\big)\Big\}\\
&=:\langle x_\epsilon,u\rangle
\end{align*}
第三行是因为连续函数在紧集上一定存在最大值,因而我们记其最大值点为 $x_\epsilon$;并且进一步,我们知道这个最大值必然在 $\langle g_\epsilon\rangle$ 的极点处取得,换言之,必然有某 $v_\epsilon\in\mathbb{S}^{n-1}$ 使得 $x_\epsilon=g_\epsilon(v_\epsilon)v_\epsilon$。总结起来就是:
\begin{equation}
\begin{aligned}
\exists\, v_\epsilon\in\mathbb{S}^{n-1},&\quad h_{\langle g_\epsilon\rangle }(u)=\langle g_\epsilon(v_\epsilon)v_\epsilon,u\rangle\\
\forall\,v\in\mathbb{S}^{n-1},&\quad h_{\langle g_\epsilon\rangle }(u)\ge\langle g_\epsilon(v)v,u\rangle 
\end{aligned}\label{eq:4-8}\tag{4-8}
\end{equation}

对 $h_{\langle g_0\rangle}(u)$ 同理,
\begin{equation}
\begin{aligned}
\exists\, v_0\in\mathbb{S}^{n-1},&\quad h_{\langle g_0\rangle }(u)=\langle g_0(v_0)v_0,u\rangle\\
\forall\,v\in\mathbb{S}^{n-1},&\quad h_{\langle g_0\rangle }(u)\ge\langle g_0(v)v,u\rangle
\end{aligned} \label{eq:4-9}\tag{4-9}
\end{equation}

综合起来可以推出
\begin{align*}
h_{\langle g_\epsilon\rangle}(u)-h_{\langle g_0\rangle}(u)&\le \langle g_\epsilon (v_\epsilon)v_\epsilon,u\rangle- \langle g_0 (v_\epsilon)v_\epsilon,u\rangle=\langle v_\epsilon,u\rangle\big(g_\epsilon (v_\epsilon)-g_0(v_\epsilon)\big)\\
h_{\langle g_\epsilon\rangle}(u)-h_{\langle g_0\rangle}(u)&\ge \langle g_\epsilon (v_0)v_0,u\rangle- \langle g_0 (v_0)v_0,u\rangle=\langle v_0,u\rangle\big(g_\epsilon (v_0)-g_0(v_0)\big)
\end{align*}
故
\begin{equation}
\langle v_0,u\rangle{\color{red}\underbrace{\frac{g_\epsilon(v_0)-g_0(v_0)}{\epsilon}}_{\Large\text{①}}}\le \frac{h^{-n}_{\langle g_\epsilon\rangle}(u)-h^{-n}_{\langle g_0\rangle}(u)}\epsilon\le {\color{blue}\underbrace{\frac{g_\epsilon(v_\epsilon)-g_0(v_\epsilon)}{\epsilon}}_{\Large\text{②}}}{\color{DarkGreen}\underbrace{\vphantom{\frac{}{}}\langle v_\epsilon,u\rangle}_{\Large\text{③}}} \label{eq:4-10}\tag{4-10}
\end{equation}
这里我们把~\eqref{eq:4-10} 按不同颜色分成了三个部分,需要逐个击破。

\item  先来看 $\Large{\color{red}{\text{①}}}$, 这个比较简单,有
\begin{align*}
g_{\epsilon}(v_0)-g_0(v_0) & =\frac{1}{(h_K+\epsilon f)(v_0)}-\frac1{h_K(v_0)} \\
& =\frac{h_K(v_0)-(h_K(v_0)+\epsilon f(v_0))}{h_K(v_0)(h_K(v_0)+\epsilon f(v_0))} \\
& =\frac{-\epsilon f(v_0)}{h_K^{2}(v_0)+\epsilon h_{k}(v_0)f(v_0)}
\end{align*}
故
\[
\lim_{\epsilon\to 0^+}\frac{g_\epsilon(v_0)-g_0(v_0)}{\epsilon}=-\lim_{\epsilon\to 0^+}\frac{f(v_0)}{h_K^2(v_0)+\epsilon h_K(v_0)f(v_0)}=-\frac{f(v_0)}{h_K^2(v_0)}
\]

接下来看 $\Large{\color{DarkGreen}{\text{③}}}$, 理想情况下,我们期望应该有 $\lim\limits_{\epsilon\to 0^+}v_\epsilon=v_0$。这要怎么证明呢?

首先, $g_\epsilon$ 一致收敛于 $g$, 我们从中任取一列 $\{g_j\}_{j=1}^\infty$, 设 $g_j$ 通过式~\eqref{eq:4-8}选取出来 $v_j$。由于 $\mathbb{S}^{n-1}$ (列) 紧,那么其中任一序列必存在收敛子列,不妨设 $\{v_j\}_{j=1}^\infty$ 收敛到某 $v\in\mathbb{S}^{n-1}$ (不然,可以用其收敛子列替代之)  ;则
\[
g_j(v_j)\to g_0(v),\quad j\to\infty
\]
再根据
\begin{align*}
h_{\langle g_j\rangle}(u)&=g_j(v_j)\langle v_j,u\rangle\to g_0(v)\langle v,u\rangle\\
h_{\langle g_j\rangle}(u)&\to h_{\langle g_0\rangle}(u)
\end{align*}
这两个极限应该是相同的,故 $h_{\langle g_0\rangle}(u)=\langle g_0(v)v,u\rangle$。同时,对于几乎处处 $u\in\mathbb{S}^{n-1}$, 该等式不依赖于 $\{v_j\}_{j=1}^\infty$ 的选取,原因正如之前所说, $u$ 的半径逆 Gauss 映射 $\alpha_{\langle g_0\rangle}^{-1}(u)$ 是几乎处处唯一的,有
\begin{align*}
h_{\langle g_0\rangle}(u)=\langle g_0(v),u\rangle &\implies g_0(v)v\in \nu^{-1}_{\langle g_0\rangle}(u) \\
&\implies v=\frac{g_0(v)v}{\|g_0(v)v\|}=\big(r_{\langle g_0\rangle}^{-1}\circ \nu_{\langle g_0\rangle}^{-1}\big)(u)=\alpha_{\langle g_0\rangle}^{-1}(u)
\end{align*}
所以 (对几乎处处 $u$),  $v$ 是独立于我们序列的选取的,有 $\lim\limits_{\epsilon\to 0^+}v_\epsilon=v=\alpha^{-1}_{\langle g_0\rangle}(u)$。

哎!可别忘了,根据式~\eqref{eq:4-9},  $h_{\langle g_0\rangle}(u)=\langle g_0(v_0)v_0,u\rangle\implies v_0=\alpha^{-1}_{\langle g_0\rangle}(u)$, 那只能有 $v_0=v$。这就证明了  $\lim\limits_{\epsilon\to 0^+}v_\epsilon\to v_0$。

最后算 $\Large{\color{blue}{\text{②}}}$。和 $\Large{\color{red}{\text{①}}}$ 类似,有
\begin{align*}
g_{\epsilon}(v_\epsilon)-g_0(v_\epsilon) & =\frac{1}{(h_K+\epsilon f)(v_\epsilon)}-\frac1{h_K(v_\epsilon)} \\
& =\frac{-\epsilon f(v_\epsilon)}{h_K^{2}(v_\epsilon)+\epsilon h_{k}(v_\epsilon)f(v_\epsilon)}
\end{align*}
于是
\[
\lim_{\epsilon\to 0^+}\frac{g_\epsilon(v_\epsilon)-g_0(v_\epsilon)}{\epsilon}=-\lim_{\epsilon\to 0^+}\frac{f(v_\epsilon)}{h_K^2(v_\epsilon)+\epsilon h_K(v_\epsilon)f(v_\epsilon)}=-\frac{f(v_0)}{h_K^2(v_0)}
\]
最后一步用了 $\Large{\color{DarkGreen}{\text{③}}}$ 的结论。

整理一下,如果 $u\in\mathbb{S}^{n-1}$ 满足 $\alpha^{-1}_{\langle g_0\rangle}(u)$ 唯一,则根据式~\eqref{eq:4-9},
\begin{align*}
\lim_{\epsilon\to 0^+}\frac{h^{-n}_{\langle g_\epsilon\rangle}(u)-h^{-n}_{\langle g_0\rangle}(u)}\epsilon&\ge \langle v_0,u\rangle{\color{red}\underbrace{\frac{g_\epsilon(v_0)-g_0(v_0)}{\epsilon}}_{\Large\text{①}}}=\color{red}{-\frac{f(v_0)}{h^2_K(v_0)}}\langle v_0,u\rangle\\
\lim_{\epsilon\to 0^+}\frac{h^{-n}_{\langle g_\epsilon\rangle}(u)-h^{-n}_{\langle g_0\rangle}(u)}\epsilon&\le {\color{blue}\underbrace{\frac{g_\epsilon(v_\epsilon)-g_0(v_\epsilon)}{\epsilon}}_{\Large\text{②}}}\color{DarkGreen}{\underbrace{\vphantom{\frac{}{}}\langle v_\epsilon,u\rangle}_{\Large\text{③}}} =\color{blue}{-\frac{f(v_0)}{h^2_K(v_0)}}\color{DarkGreen}{\langle v_0,u\rangle}
\end{align*}
即,
\begin{equation}
\lim_{\epsilon\to 0^+}\frac{h^{-n}_{\langle g_\epsilon\rangle}(u)-h^{-n}_{\langle g_0\rangle}(u)}\epsilon= -\frac{f(v_0)}{h^2_K(v_0)} \langle v_0,u\rangle \label{eq:4-11}\tag{4-11}
\end{equation}

\item 来研究一下 $v_0$ 和 $u$ 的关系。由于 $v_0=\alpha^{-1}_{\langle g_0\rangle}(u)=\alpha^{-1}_{K^\circ}(u)$,
\begin{align*}
&h_{K^\circ}(u)=h_{\langle g_0\rangle }(u)=\langle u,g_0(v_0)v_0\rangle=\langle u,\rho_{K^\circ}(v_0)v_0\rangle \tag{由~\eqref{eq:4-9}式}\\
\implies& h_K(v_0)=\frac1{\rho_{K^\circ}(v_0)}=\left\langle\frac u{h_{K^\circ}(u)},v_0\right\rangle=\langle \rho_K(u)u,v_0\rangle\\
\implies& v_0 = \alpha_K(u) \label{eq:4-12}\tag{4-12}
\end{align*}
当然,这里我们仍然可以认为对于几乎处处的 $u\in\mathbb{S}^{n-1}$,  $\alpha_K(u)$ 唯一确定,所以写作 $v_0=\alpha_K(u)$ 没问题。
至此,我们终于能够算出第二步想计算的的目标了。整理一下~\eqref{eq:4-7},
\begin{align*}
&\lim_{\epsilon\to 0^+}\frac{h^{-n}_{\langle g_\epsilon\rangle}(u)-h^{-n}_{\langle g_0\rangle}(u)}\epsilon\\
&=-nh_{\langle g_0\rangle}^{-n-1}(u)\lim_{\epsilon\to 0^+}\frac{h_{\langle g_\epsilon\rangle}(u)-h_{\langle g_0\rangle}(u)}{\epsilon} \tag{由~\eqref{eq:4-7}式}\\
&=nh_{K^\circ}^{-n-1}(u) \frac{f(v_0)}{h^2_K(v_0)} \langle v_0,u\rangle \tag{由 ~\eqref{eq:4-11} 式}\\
&=\frac{nf(v_0)\langle v_0,u\rangle \rho_K^{n+1}(u)}{h^2_K(v_0)}\\
&=\frac{nf(v_0)\overbrace{\langle v_0, \rho_K^{n}(u)u\rangle}^{h_K(v_0)} \rho_K^{n}(u)}{h^2_K(v_0)}\\
&=\frac{nf(v_0)\rho_K^n(u)}{h_K(v_0)}\\
&=\frac{nf(\alpha_K(u))\rho_K^n(u)}{h_K(\alpha_K(u))} \tag{由 ~\eqref{eq:4-12} 式} 
\end{align*}
即,对几乎处处的 $u\in\mathbb{S}^{n-1}$,
\[
\lim_{\epsilon\to 0^+}\frac{h^{-n}_{\langle g_\epsilon\rangle}(u)-h^{-n}_{\langle g_0\rangle}(u)}\epsilon=\frac{nf(\alpha_K(u))\rho_K^n(u)}{h_K(\alpha_K(u))}\label{eq:4-13}\tag{4-13}
\]
这是变分公式的证明中最重要的极限。

\item 证明完极限,我们证明 $\frac{h^{-n}_{\langle g_\epsilon\rangle}(u)-h^{-n}_{\langle g_0\rangle}(u)}\epsilon$ 在 $\mathbb{S}^{n-1}$ 上能被控制,以便能用控制收敛定理搞定~\eqref{eq:4-6}式。也就是说,我们证明:存在一个 $M\in(0,\infty)$ 使得
\[
\left|\frac{h^{-n}_{\langle g_\epsilon\rangle}(u)-h^{-n}_{\langle g_0\rangle}(u)}\epsilon\right|\le M,\quad \forall \,u\in\mathbb{S}^{n-1}
\]
这个其实很简单,直接用式 ~\eqref{eq:4-10}, 有
\[
\left|\frac{h^{-n}_{\langle g_\epsilon\rangle}(u)-h^{-n}_{\langle g_0\rangle}(u)}\epsilon\right|\le\max\Bigg\{\left|\frac{g_\epsilon(v_0)-g_0(v_0)}{\epsilon}\right|,\left|\frac{g_\epsilon(v_\epsilon)-g_0(v_\epsilon)}{\epsilon}\right|\Bigg\}
\]
 (根据 Cauchy-Schwarz 不等式, $|\langle v_0,u\rangle|$ 和 $|\langle v_\epsilon,u\rangle|$ 都不超过 1, 这里将之放缩为了 1。) 

而上式右侧处理起来也很方便:
\[
\left|\frac{g_\epsilon(v_\epsilon)-g_0(v_\epsilon)}{\epsilon}\right|=\left|\frac{f(v_\epsilon)}{h_K(\epsilon)(h_K(\epsilon)+\epsilon f(v_\epsilon))}\right|\le \frac{\max\limits_{v\in\mathbb{S}^{n-1}}|f(v)|}{\min\limits_{v\in\mathbb{S}^{n-1}}h_K^2(v)}=:M
\]
同理 $\displaystyle \left|\frac{g_\epsilon(v_0)-g_0(v_0)}{\epsilon}\right|\le M$。

\item 收尾!根据式~\eqref{eq:4-6},
\begin{align*}
&\lim_{\epsilon\to0^+}\frac{V_n([f_\epsilon])-V_n(K)}{\epsilon}\\
&=\lim_{\epsilon\to 0^+}\frac1n\int_{\mathbb{S}^{n-1}}\frac{h^{-n}_{\langle g_\epsilon\rangle}(u)-h^{-n}_{\langle g_0\rangle}(u)}{\epsilon}\mathrm{d}u \tag{由~\eqref{eq:4-6}式}\\
&=\frac1n\int_{\mathbb{S}^{n-1}}\lim_{\epsilon\to 0^+}\frac{h^{-n}_{\langle g_\epsilon\rangle}(u)-h^{-n}_{\langle g_0\rangle}(u)}{\epsilon}\mathrm{d}u \tag{控制收敛定理}\\
&= \frac1n \int_{\mathbb{S}^{n-1}}\frac{nf(\alpha_K(u))}{h_K(\alpha_K(u))}\rho_K^n(u)\mathrm{d}u \tag{由~\eqref{eq:4-13} 式}\\
&=\int_{\mathbb{S}^{n-1}}\frac{f(v)}{h_K(v)}h_K(v)\mathrm{d}S(K,v) \tag{$v=\alpha_K(u)$ 换元}\\
&=\int_{\mathbb{S}^{n-1}}f(v)\mathrm{d}S(K,v)
\end{align*}
得证!其中第四行用到了我们前文推论~\ref{col:map_diagram}。\end{enumerate}
\end{proof}

\begin{remark}
    上述证明来自 Huang, Y., Lutwak, E., Yang, D., \& Zhang, G. (2016). \textit{Geometric measures in the dual Brunn–Minkowski theory and their associated Minkowski problems}. Acta Mathematica, 216(2), 325-388。
\end{remark}
\hfill\break

在凸几何中,混合体积的变分公式~\eqref{eq:4-4}是非常重要的结论,有相当一部分研究对象都可以视作是 $V_1(K,L)$ 的优化问题。包括但不限于

\begin{itemize}
\item Minkowski 问题;
\item Petty 体 (Petty bodies);
\item 著名的 John 椭球 (John ellipsoid, 也称 Löwner–John ellipsoid);
\item 仿射表面积 (affine surface areas)。
\end{itemize}
因此单独一个混合体积的变分公式可以引申出来解决很多问题。
 