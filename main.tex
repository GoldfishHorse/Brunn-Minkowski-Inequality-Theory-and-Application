\documentclass[lang=cn]{elegantbook}
% https://www.zhihu.com/question/444215565
\let\proof\relax
\let\endproof\relax
\usepackage{amsmath, amssymb, amsthm}
\usepackage[dvipsnames,svgnames,table]{xcolor}    % see http://ctan.org/pkg/xcolor
\usepackage{enumitem}
\usepackage{hyperref}
\def\UrlBreaks{\do\/\do-}
\usepackage{graphicx}
\usepackage{bm}
\renewcommand{\examplename}{例}


\title{Brunn-Minkowski 不等式:理论与应用}
\author{\href{https://www.zhihu.com/people/he-eeeeeeeee}{金鱼马}}
\date{\today}
\institute{浙江大学}
\bioinfo{License}{\href{https://creativecommons.org/licenses/by-nc-sa/4.0/deed.en}{CC BY-NC-SA 4.0}}
\bioinfo{封面}{图片来自 pixiv \href{https://www.pixiv.net/artworks/17640094}{17640094} (画师@\href{https://www.pixiv.net/users/2932037}{いずみべる})}
\cover{figures/17640094_p0.jpg}

\begin{document}

\maketitle

\chapter*{引言}

这份笔记来自笔者在山东大学数学学院 2025 暑期课程《\href{https://www.math.sdu.edu.cn/info/1020/20825.htm}{Brunn-Minkowski 不等式:理论与应用}》中的笔记;
课程主讲为\href{https://www.math.mun.ca/~depingy/}{叶德平}老师,整体纲举目张、深入浅出,同时清晰易懂,大概只需要读者有基本的实分析基础。
另外,课中很多证明也是精华所在,其结果、思想、工具都是凸几何领域的核心,也都可以用于其它相关结论。
\hfill\break

{
    \renewcommand{\problemname}{摘要}
    \renewcommand{\theprob}{} % 清空编号
\begin{problem}
\textit{
在数学理论与实际应用中,对函数或集合施加凸性条件往往是关键前提,因为凸函数与凸集合不仅具备丰富的优良性质,还蕴含边界正则性、可微性等核心信息。Brunn-Minkowski 理论作为专注于凸体研究的数学领域,系统整合了凸体的代数结构、几何特性与分析性质。其理论根基源于体积与凸体 Minkowski 加法的深度结合,由此衍生的Brunn-Minkowski不等式及对应的变分公式,推动了一系列里程碑式成果的诞生,如Minkowski 不等式、Minkowski 问题等。本次课程将聚焦Brunn-Minkowski 理论的核心:Brunn-Minkowski 不等式,将从不等式的严格证明出发,深入探讨其在几何分析、泛函不等式等领域的经典应用,并延伸至若干在实际问题中具有重要价值的关联不等式,揭示凸体理论的跨学科应用潜力。
}
\end{problem}
}
\begin{center}
    \includegraphics[width=0.5\textwidth]{figures/poster.jpg}
\end{center}

笔记的六章内容安排如下:
\begin{itemize}
    \item 第~\ref{chapter:1} 章:引入、 Brunn-Minkowski 不等式的等定义,两种证明;
    \item 第~\ref{chapter:2} 章:BM 不等式的泛函形式:Prékopa–Leindler 不等式和 Borell-Brascamp-Lieb 不等式,以及最优传输的视角;
    \item 第~\ref{chapter:3} 章:Steiner 对称化;
    \item 第~\ref{chapter:4} 章:混合体积、Minkowski 第一不等式、混合体积的变分公式;
    \item 第~\ref{chapter:5} 章:Minkowski 问题的解决;
    \item 第~\ref{chapter:6} 章:Blaschke–Santaló 不等式和仿射等周不等式。
\end{itemize}

\begin{center} 
    \includegraphics[width=0.7\textwidth]{figures/outline.png}
\end{center}

当然,Brunn-Minkowski 理论涉及的内容很多,不仅有凸几何、微分几何、泛函、最优传输,还可以推广到非常丰富的应用中 (如群论、环论、流形等等\footnote{一个代表例子是菲尔兹奖得主 \href{https://www.math.columbia.edu/~okounkov/}{Andrei Okounkov} 的论文 Okounkov, A. (1996). \textit{Brunn–Minkowski inequality for multiplicities.Inventiones mathematicae},125(3), 405-411.}) , 
这门课程可以说是只介绍了最基础也最精华的一些部分。关于更多的内容,读者可以去阅读相关教材,笔者参考到的有
\begin{itemize}
    \item Schneider, R. (2013). \textit{Convex bodies: the Brunn–Minkowski theory}(Vol. 151). Cambridge university press.
    \item Gruber, P. M. (2007).\textit{Convex and discrete geometry}. Berlin, Heidelberg: Springer Berlin Heidelberg.
    \item Böröczky, K. J., Figalli, A., \& Ramos, J. P. (2025).\textit{Isoperimetric inequalities, Brunn-Minkowski theory and Minkowski type Monge-Ampère equations on the sphere}. \url{https://users.renyi.hu/~carlos/Brunn-Minkowski-Book-2026-06-06.pdf} 
\end{itemize}

以及 Gruber 的一篇经典综述,Gardner, R. (2002).\textit{The brunn-minkowski inequality}. Bulletin of the American mathematical society,39(3), 355-405。

\tableofcontents

\pagenumbering{arabic}
\mainmatter

\chapter{Brunn-Minkowski 不等式}
\label{chapter:1}

\section{动机}

所谓的"Brunn-Minkowski 不等式"其实不是很神奇的东西,它的动机其实很简单。大家肯定都见过下面的 \emph{Young 不等式}:

\begin{lemma*}[Young 不等式]\label{lem:young}
对任意的 $a,b\ge 0$ 以及 $\lambda \in [0,1]$,有 
\[
\lambda a+(1-\lambda)b\ge a^\lambda b^{1-\lambda}
\]YY
取等当且仅当 $a=b$ 或 $\lambda$ 为 0、1。
\end{lemma*}

Young 不等式本质上是一个"算数平均 $\geq$ 几何平均"的定理,可以看成是 AM-GM 的推广(后者对应 $\lambda=1/2$ 的特殊情况)。后面的证明中我们会经常用到它。

Young 不等式的证明也很容易。最直接和本质的证明方法是"log 函数的凹性":

\begin{proof}
当 $a=0$ 或 $b=0$ 时显然成立,当 $a,b>0$ 时,根据 Jensen 不等式:
\[
\log(\lambda a + (1-\lambda)b)\ge \lambda \log a +(1-\lambda) \log b,\quad \forall \lambda\in[0,1]
\] 
\end{proof}

接下来的问题是:\emph{在集合里面,有没有对应的 Young 不等式}?

如何把``数"与``集合"对应起来呢?自然的想法是,用集合的``大小"—— 如一维空间中线段的长度、二维的面积、三维的体积——在 $n$ 维中则是 \emph{Lebesgue 测度}。此外,加法和乘法也可以推广到集合。

\begin{definition}[Minkowski 和与 Minkowski 扩张]
设 $A,B \subset\mathbb{R}^n$,定义其 \emph{Minkowski 和}为 
\[
A+B:=\{a+b:a\in A,b\in B\}
\]
对于 $\lambda\ge 0$,定义集合 $A$ 的 \emph{Minkowski 扩张}为 $\lambda A:=\{\lambda a:a\in A\}$。
\end{definition}

我们举几个例子:

\begin{example}\label{ex:line_segment}
在实数轴 $\mathbb{R}^1$ 上有两个集合,$A=[0,a]$、$B=[0,b]$,$a,b\ge 0$。则 
\[
\lambda A+(1-\lambda)B:=\{\lambda x+(1-\lambda )y:x\in A,y\in B\}=[0,\lambda a+(1-\lambda )b]
\]
如果用 $\ell(\cdot)$ 代表区间长度,那么显然地,
\[
\ell(\lambda A+(1-\lambda)B)=\lambda a+(1-\lambda )b= \lambda \ell(A)+(1-\lambda) \ell(B)
\]
更重要的是:根据 Young 不等式,对 $\lambda\in[0,1]$:
\[
\ell (\lambda A+(1-\lambda)B)\ge  \ell(A)^\lambda\ell (B)^{1-\lambda}
\]
\end{example}
\hfill \break

例~\ref{ex:line_segment}比较简单,不过当来到二维,事情开始变得有趣起来。

\begin{example}\label{ex:minkowski_sum}
在 $\mathbb{R}^2$ 中有两个集合:$A=[-1,1]\times [-1,1]$ 是一个方形,$B=\{(x,y):x^2+y^2\le \epsilon\}$ 是一个半径为 $\epsilon$ 的圆。简单起见,我们只考察以下 $A+B$。$A+B$ 是什么样子呢?容易看出,其实就是``正方形四周扩展出去一个圆",见图~\ref{fig:Minkowski_sum}。

\begin{figure}[h]
    \centering
\includegraphics[width=0.8\linewidth]{figures/Minkowski_sum.jpg}
\caption{正方形和圆的 Minkowski 和示意图。\footnotemark}
\label{fig:Minkowski_sum}
\end{figure}
\footnotetext{图片来自 Gardner, R. (2002). \textit{The brunn-minkowski inequality}. Bulletin of the American mathematical society,39(3), 355-405. p. 359.}
 
这时,用 $\ell(\cdot)$ 代表集合的面积,则容易验证 $\ell(A+B)$ 的面积就是 $4+8\epsilon+\pi\epsilon^2$,同时 $\ell(A)=4$、$\ell(B)=\pi\epsilon^2$。

可以注意到的事实有:

\begin{itemize}
\item 当 $\epsilon\to 0$ 时,$\displaystyle\lim_{\epsilon\to 0}\frac{l(A+B)-l(A)}{\epsilon}=\frac{4+8\epsilon+\pi\epsilon^2-4}{\epsilon}=8$,这个量对应了 $A$ 的周长——在高维情况下,某种程度上,可以用这种方法来定义"表面积"。
\item 显见,$\ell(A+B)\ge \ell(A)+\ell(B)$。
\item 由于是在二维,对应维度的提升,我们考虑一下面积的 $\frac12$ 次方,可以验证下面两个不等式也是成立的 (见图~\ref{fig:ineq_comp}):
\begin{align*}
\ell(A+B)^{1/2}&=\sqrt{4+8\epsilon+\pi\epsilon^2}\ge 2+\sqrt \pi\epsilon=\ell(A)^{1/2}+\ell(B)^{1/2}\\
\ell\left(\frac{A+B}{2}\right)&=\frac14\big(4+8\epsilon+\pi\epsilon^2\big)\ge 2\sqrt \pi\epsilon=\ell(A)^{1/2}\ell(B)^{1/2}
\end{align*}
\end{itemize}


\begin{figure}
    \centering
    % \includestandalone[width=0.8\textwidth]{figures/function_compare_1}
    \includegraphics[width=0.8\textwidth]{figures/function_compare_1.pdf}
    \caption{比较不等式两侧的值。}\label{fig:ineq_comp}
\end{figure}

\end{example}
\hfill\break

在一般的 $\mathbb{R}^n$ 中,能不能有类似的结论呢?

\section{BM 不等式的几个等价形式}

若照搬 Young 不等式,那么我们想要的``$\mathbb{R}^n$ 上的推广"的样子应该是:对内点非空的紧集 $K,L\subset \mathbb{R}^n$,$\lambda\in[0,1]$,记 $V_n$ 代表二者的 ($n$ 维) 体积,则
\[
V_n((1-\lambda)K+\lambda L)^{1/n}\ge(1-\lambda)V_n(K)^{1/n}+\lambda V_n(L)^{1/n}
\]
在此之前,我们来证明一下,上述不等式有一些等价变形。

\begin{theorem}[BM 不等式的等价形式]\label{thm:BM-equivalence}
如下不等式等价:

\begin{enumerate}
\item[(a)] [1/n-凹] $V_n((1-\lambda)K+\lambda L)^{1/n}\ge(1-\lambda)V_n(K)^{1/n}+\lambda V_n(L)^{1/n}$;

\item[(b)] [维度相关] $V_n(K+L)^{1/n}\ge V_n(K)^{1/n}+V_n(L)^{1/n}$;

\item[(c)] [维度无关] $V_n((1-\lambda)K+\lambda L)\ge V_n(K)^{1-\lambda}\cdot V_n(L)^{\lambda}$; 或者写作对数凸的形式 $\log V_n((1-\lambda)K+\lambda L)\ge (1-\lambda)\log V_n(K) +\lambda\log V_n(L)$。(由于维度无关,这种情况更容易推广到其它空间。)

\item[(d)] [体积归一化] 若 $V_n(K)=V_n(L)=1$,那么 $V_n((1-\lambda)K+\lambda L)\ge 1$。
\end{enumerate}
\end{theorem}

\begin{proof}
$\lambda$ 为 0 或 1 的情况下这些不等式都显然取等。接下来我们均不考虑这些情况。

(a) $\Rightarrow$ (b):只需在令 $K'=\frac1{1-\lambda}K$、$L'=\frac1\lambda L$,用 $K'$ 和 $L'$ 代入 (a)。

(b) $\Rightarrow$ (a):令 $K'=(1-\lambda)K$、$L'=\lambda L$ 代入 (b)。

(a) $\Rightarrow$ (c) 用 Young 不等式,
\begin{align*}
V_n((1-\lambda)K+\lambda L)&\ge\left[(1-\lambda)V_n(K)^{1/n}+\lambda V_n(L)^{1/n}\right]^n \tag{由 (a)}\\
&\ge \left[\left(V_n(K)^{1/n}\right)^{1-\lambda}\left(V_n(L)^{1/n}\right)^{\lambda}\right]^n \tag{\hyperref[lem:young]{Young 不等式}}\\
&=V_n(K)^{1-\lambda}V_n(L)^{\lambda}
\end{align*}

(c) $\Rightarrow$ (d) 显然。

(d) $\Rightarrow$ (a) 对于给定的集合 $K$、$L$,我们将二者体积归一化,即取 $K'=\frac{K}{V_n(K)^{1/n}}$、$L'=\frac{L}{V_n(L)^{1/n}}$。令 $\lambda':=\frac{\lambda V_n(L)^{1/n}}{(1-\lambda)V_n(K)^{1/n}+\lambda V_n(L)^{1/n}}$,那么,
\begin{align*}
1&\le V_n((1-\lambda')K'+\lambda' L') \tag{由 (d)}\\
&=V_n\left(\frac{(1-\lambda)K+\lambda L}{(1-\lambda)V_n(K)^{1/n}+\lambda V_n(L)^{1/n}}\right)\\
&=\frac{V_n((1-\lambda)K+\lambda L)}{\left[(1-\lambda)V_n(K)^{1/n}+\lambda V_n(L)^{1/n}\right]^n}
\end{align*}
两边取 $1/n$ 次方就得到了 (a)。
\end{proof}

前面例~\ref{ex:minkowski_sum} 中观察到的两个不等式对应这里的 (b)、(c)。 

\section{凸体的 BM 不等式}

\subsection{凸集与凸体的概念}

我们首先来研究一种重要的情况,即 $K$ 和 $L$ 都是\emph{凸体}(Convex Body)。

\begin{definition}[凸集与凸体]
集合 $K\subseteq\mathbb{R}^n$ 若满足对任意的 $\lambda\in[0,1]$ 和任意的 $x,y\in K$,有 $\lambda x+(1-\lambda)y\in K$,则称 $K$ 是\emph{凸集}。若 $K$ 是内点非空的紧凸集,则称 $K$ 为\emph{凸体}。
\end{definition}

在继续介绍 BM 不等式之前,先简单补充几个关于凸集/凸体的构造方法,

\begin{enumerate}[label=\textbf{方法 \chinese*}:, leftmargin=6em]

\item 给定一个集合 $E\subseteq\mathbb{R}^n$,那么可以定义 $E$ 的闭凸包 
\[
\mathrm{Conv}(E):=\bigcap\Big\{F:E\subseteq F,\,F\subseteq\mathbb{R}^n\text{ 是闭凸集 }\Big\}
\]

\item 闭凸包也可用 $E$ 中点的凸组合定义 
\[
\mathrm{Conv}(E)=\bigg\{\sum_{i=1}^m\lambda_ix_i:m\in\mathbb{N},x_i\in E,\lambda_i\in[0,1],\sum_{i=1}^m\lambda_i=1\bigg\}
\]

\item 用 Hahn–Banach 定理的几何形式,即点与闭凸集的分离定理。简而言之,对于某闭凸集和不在此集合中的一个点,总能找到一个超平面将点与集合分离开来。设 $a\in\mathbb{R}$ 代表水平 (level)、$u\in\mathbb{S}^{n-1}$ 代表法向量,我们用 
\[
H_{u,a}:=\{x\in\mathbb{R}^n:\langle x,u\rangle=a\}
\]
代表超平面;同时,记 $H_{u,a}^-:=\{x\in\mathbb{R}^n:\langle x,u\rangle\le a\}$ 和 $H_{u,a}^+:=\{x\in\mathbb{R}^n:\langle x,u\rangle \ge a\}$ 代表超平面划分出的半空间。分离超平面定理可以用下面的图表示。


\begin{figure}
    \centering
    \includegraphics[width=0.4\textwidth]{figures/separate_plane.pdf}
    \caption{分离超平面图示。}
\end{figure}

则 $E$ 的闭凸包也可以表示为 
\[
\mathrm{Conv}(E)=\bigcap\big\{H_{u,a}^-:E\subset H_{u,a}^-\big\}
\]
即用半空间来替代方法一中的闭凸集 $F$。

\item 注意到,若 $0\in\mathrm{int}(E)$,则在方法 3 里,每个包含 $E$ 的半空间 $H_{u,a}^-$ 必然都满足 $a>0$。故 
\begin{align*}
\mathrm{Conv}(E)&=\Big\{x\in\mathbb{R}^n:\langle x,u\rangle\le a,\text{ 其中 }(u,a)\text{ 满足对任意 }y\in E,\,\langle y,u\rangle \le a\Big\}\\
&=\left\{x\in\mathbb{R}^n:\left\langle x,\frac ua\right\rangle\le 1,\text{ 其中 }(u,a)\text{ 满足对任意 }y\in E,\,\left\langle y,\frac ua\right\rangle \le 1\right\}\\
&=\big\{x\in\mathbb{R}^n:\langle x,z\rangle \le 1,z\in E^\circ\big\}\\
&=(E^\circ)^\circ
\end{align*}
其中 
\[
E^\circ:=\{z\in\mathbb{R}^n:\langle x,z\rangle\le 1\text{ 对任意 }x\in E\text{ 成立 }\}
\]
代表 $E$ 的\emph{极集}(polar set)。在某种程度上,极集可以视作集合层面的``倒数"。
\end{enumerate}
\subsection{凸体的 BM 不等式之证明}

\begin{theorem}[凸体的 BM 不等式]\label{thm:BM-convex}
设 $K,L\subset\mathbb{R}^n$ 是凸体,$\lambda\in[0,1]$,则 
\[
V_n((1-\lambda)K+\lambda L)^{1/n}\ge(1-\lambda)V_n(K)^{1/n}+\lambda V_n(L)^{1/n}
\]
取等当且仅当 $K$ 和 $L$ 是\emph{位似}(homothetic) 的,即,存在 $c>0$ 和 $x_0\in\mathbb{R}^n$ 使得 $K=cL+x_0$。
\end{theorem}
\begin{remark}
    由于体积与集合的位置没有关系,在接下来的证明中,我们经常会为了计算方便而把集合平移到特殊位置。
\end{remark}

\begin{proof}

用数学归纳法,对维数 $n$ 进行归纳。

当 $n=1$ 时,$K,L$ 都是 $\mathbb{R}^1$ 中的区间,显然 $V_1((1-\lambda)K+\lambda L)=  (1-\lambda)V_1(K) +\lambda V_1(L)$。

下设 BM 不等式对某 $n-1$ 成立,$n\ge 2$。要证 BM 不等式对 $n$ 成立,我们证明定理 \ref{thm:BM-equivalence} 里面等价形式的 (d),即,设 $V_n(K)=V_n(L)=1$,证 $V_n((1-\lambda)K+\lambda L)\ge 1$。接下来的目标是用 $n-1$ 维的对象来把 $n$ 维凸体体积表示出来。

取定 $u\in\mathbb{S}^{n-1}$ 是一个任意单位方向。记 $K_\lambda:=(1-\lambda)K+\lambda L$,则 $K_\lambda$ 也是凸体(留给读者验证),不妨设 0 是 $K_\lambda$ 的内点 (不然的话总可以平移集合使之成立)。我们总可以找到一个\emph{支撑函数}(support function):
\[
h(K_\lambda,u):=\max_{x\in K_\lambda}\langle x,u\rangle
\]

注意到 
\begin{align*}
h(K_\lambda,u)&=h((1-\lambda)K+\lambda L,u)\\
&=\max_{x\in K,y\in L}\langle (1-\lambda)x+\lambda y,u\rangle\\
&=(1-\lambda)\max_{x\in K}\langle x,u\rangle+\lambda\max_{y\in L}\langle y,u\rangle\\
&=(1-\lambda)h(K,u)+\lambda h(L,u)
\end{align*}
所以\emph{集合的 Minkowski 和总对应支撑函数的和}。


记 
\begin{align*}
\alpha_\lambda&:=-h(K_\lambda,-u)\\
\beta_\lambda&:=h(K_\lambda,u)
\end{align*}

则两个超平面 $H_{-u,-\alpha_\lambda}=H_{u,\alpha_\lambda}$、$H_{u,\beta_\lambda}$ 就把集合 $K_\lambda$``夹"了起来。如果我们在 $H_{u,\alpha_\lambda}$、$H_{u,\beta_\lambda}$ 二者间沿着 $u$ 方向平移超平面,将超平面与 $K_\lambda$ 相交的部分称作"截面 (section)",则这个截面的"面积"就是 $n-1$ 维的体积。见图~\ref{fig:BM_proof_illustration}。


\begin{figure}[h]
    \centering 
    \includegraphics[width=0.8\textwidth]{figures/BM_proof_illustration.pdf}
    \caption{左:当原点在凸体内部时,支撑函数相当于原点到支撑超平面的距离。右:截面是 $n-1$ 维的,可以对其用归纳假设。}
    \label{fig:BM_proof_illustration}
\end{figure}


为此,记 
\begin{align*}
v_\lambda(t)&:=V_{n-1}(K_\lambda \cap H_{u,t})\\
w_\lambda(t)&:=\int_{\alpha_\lambda}^t v_\lambda(s)\mathrm{d}s
\end{align*}

显见,$\displaystyle V_n(K_\lambda)=w_\lambda(\beta_\lambda)=\int_{\alpha_\lambda}^{\beta_\lambda}v_\lambda (t)\mathrm{d}t$,我们成功用 $n-1$ 维度的体积来刻画了 $n$ 维体积。

定义 $z_\lambda(s):=(1-\lambda)w_0^{-1}(s)+\lambda w_{1}^{-1}(s)$,其中 $^{-1}$ 代表反函数,$s\in[0,1]$(因为 $V_n(K)=V_n(L)=1$)。这个通过插值定义的量很重要,一方面有 
\begin{align*}
z_\lambda(0)&=(1-\lambda)w_0^{-1}(0)+\lambda  w_1^{-1}(0)=(1-\lambda)\alpha_0+\lambda\alpha_1=\alpha_\lambda\\
z_\lambda(1)&=(1-\lambda)w_0^{-1}(1)+\lambda  w_1^{-1}(1)=(1-\lambda)\beta_0+\lambda\beta_1=\beta_\lambda
\end{align*}

并且不难注意到下式成立:
\[
(1-\lambda)\big(K\cap H_{u,z_0(s)}\big)+\lambda\big(L\cap H_{u,z_1(s)}\big)\subseteq K_\lambda\cap H_{u,z_\lambda(s)}
\]

方便起见,我们记 $K(s):=K\cap H_{u,z_0(s)}$、$L(s):=L\cap H_{u,z_1(s)}$。

此时,我们可以直接推导出
\begin{align*}
V_n(K_\lambda)&=\int_{\alpha_\lambda}^{\beta_\lambda}v_\lambda(t)\mathrm{d}t \\
&=\int_0^1v_\lambda(z_\lambda(s))\mathrm{d}z_\lambda(s) \tag{积分换元}\\
&=\int_0^1V_{n-1}\big(K_\lambda\cap H_{u,z_\lambda(s)}\big)\mathrm{d}z_\lambda(s) \\
&\ge \int_0^1{\color{blue} V_{n-1}((1-\lambda)K(s)+\lambda  L(s))}\mathrm{d}z_\lambda(s)  \\
&\ge \int_0^1{\color{blue} V_{n-1}(K(s))^{1-\lambda}V_{n-1}(L(s))^\lambda}\mathrm{d}z_\lambda(s) \tag{归纳假设}\\
&=\int_0^1v_0(z_0(s))^{1-\lambda}v_1(z_1(s))^\lambda\mathrm{d}z_\lambda(s) \label{eq:1-1} \tag{$*$}
\end{align*}

最后一步归纳假设用了定理 \ref{thm:BM-equivalence} 中等价形式 (c)。 

由于对 $\lambda\in [0,1]$,根据 Young 不等式,
\begin{equation}
\mathrm{d}z_\lambda(s)=(1-\lambda)\mathrm{d}{w_0^{-1}}(s)+\lambda\mathrm{d}w_1^{-1} (s)=\frac{1-\lambda}{v_0(z_0(s))}+\frac{\lambda}{v_1(z_1(s))}\ge \frac{1}{v_0(z_0(s))^{1-\lambda}v_1(z_1(s))^\lambda} \label{eq:1-2}\tag{$**$}
\end{equation}
所以前面最后推导得到的~\eqref{eq:1-1} 式 $\ge 1$,我们就完成了 BM 不等式的证明。

下面证明取等条件。 ~\eqref{eq:1-2} 式取等当且仅当 $v_0(z_0(s))=v_1(z_1(s))$ 对任意 $s\in[0,1]$ 成立,亦即,对任意 $s\in[0,1]$,$z_0'(s)=z_1'(s)$。于是 $z_0(s)-z_1(s)$ 是常数。

方便起见,我们把 $K$ 和 $L$ 的质心 (centroid) 都平移到原点,即,
\[
\int_K\langle x,u\rangle\mathrm{d}x=\int_L\langle y,u\rangle\mathrm{d}y=0,\quad \forall u\in\mathbb{S}^{n-1}
\]

研究一下上述积分:如果考虑变量代换 $t:=\langle x, u\rangle$,则 $x\in H_{u,t}$,则可以得出 
\begin{align*}
0&=\int_K\langle x,u\rangle\mathrm{d}x\\
&=\int_{\alpha_0}^{\beta_0}V_{n-1}(K\cap H_{u,t})t\mathrm{d}t \tag{换元 $t=\langle x,u\rangle$}\\
&=\int_0^1 V_{n-1}(K\cap H_{u,z_0(s)})z_0(s)\mathrm{d}z_0(s) \tag{换元 $z_0(s)=t$}\\
&=\int_0^1  v_0(z_0(s)) z_0(s)\mathrm{d}z_0(s)\\
&=\int_0^1z_0(s)\mathrm{d}s
\end{align*}
同理 $\displaystyle \int_0^1z_1(s)\mathrm{d}s=0$,于是 $\displaystyle\int_0^1(\underbrace{z_0(s)-z_1(s)}_{\text{常数}})\mathrm{d}s=0$;显见这个常数只能为 0,那么 $z_0(s)=z_1(s)$ 对所有的 $s\in[0,1]$ 成立。特别地,当 $s=1$ 时,$h(K,u)=z_0(s)=z_1(s)=h(L,u)$,即对所有的 $u\in\mathbb{S}^{n-1}$,$K$ 和 $L$ 的支撑函数都相同,必然 $K=L$,两个集合也相同。

注意,在定理证明的过程中,我们假设了二者体积 $V_n(K)=V_n(L)=1$,同时又将二者质心平移到了原点。在一般的情况下,对仿射变换 $K\mapsto  cK+x_0$、$L\mapsto dL+y_0$,不等式 $V_n((1-\lambda)K+\lambda L)\ge V_n(K)^{1-\lambda}V_n(L)^\lambda$ 都仍成立 (平移不改变体积,乘以常数不改变不等式)。因此 $K=L$ 的条件应该改作:存在 $c>0$、$x_0\in\mathbb{R}$,有 $K=cL+x_0$ 成立。这就是凸体的 BM 不等式的取等条件。
\end{proof}

虽然上述证明稍显复杂,但是思路非常清晰。BM 不等式最早就是研究的凸体的情形。1887 年, Hermann Brunn 证明了凸体的 BM 不等式在 $n=3$ 的情况\footnote{Hermann Brunn (1887). \textit{Ueber Ovale Und Eiflächen}. PhD thesis.},后来 Hermann Minkowski 修正了其证明,并推广到了任意的 $n$\footnote{Hermann Minkowski (1910). \textit{Geometrie der Zahlen}.}。上述经典证明来自 H. Kneser 和 W. Süss\footnote{H. Kneser and W. Süss (1932). \textit{Die Volumina in linearen Scharen konvexer körper}, Matematisk. Tidsskrift. B , 19–25. },也可参考 Schneider, R. (2014).\textit{Convex bodies: the Brunn–Minkowski theory (Vol. 151)}. Cambridge university press 的 Theorem 7.1.1 (pp. 369-371) 等。
 

\section{一般 BM 不等式的证明}

之前的证明利用了凸体的特殊性质。能否证明更一般的、Lebesgue 可测的 $K,L$,也有 BM 不等式成立?答案是肯定的。下面我们仍以 $V_{n}(K)$ 代表集合 $K$ 的 $n$ 维"体积"——即 Lebesgue 测度。

\begin{theorem}[一般 BM 不等式]\label{thm:BM-general}
设 $K,L\subset\mathbb{R}^n$ 是非空的 Lebesgue 可测集,其测度均有限;设对 $\lambda\in[0,1]$ 有 $(1-\lambda)K+\lambda L$ 可测,那么
\[
V_n((1-\lambda)K+\lambda L)^{1/n}\ge (1-\lambda)V_n(K)^{1/n}+\lambda V_n(L)^{1/n}
\]
\end{theorem}

\begin{proof}
分三步走:

\begin{enumerate}[label=\textbf{第\chinese*步}:, leftmargin=6em]
    \item 证明 $K,L$ 都是 $n$ 维盒子 (boxes,即各边与坐标超平面平行的长方体) 的情形;
    \item 证明 $K,L$ 都是有限个不交盒子之并的情形;
    \item 用有限个盒子来逼近任意一个可测集。
\end{enumerate}

第一步,先证明最简单的情况,$K,L$ 都是 $n$ 维盒子,即 
\begin{align*}
K&=[0,a_1]\times\cdots\times [0,a_n]\\
L&=[0,b_1]\times\cdots\times [0,b_n]
\end{align*}

那么 
\[
(1-\lambda)K+\lambda L=[0,(1-\lambda)a_1+\lambda b_1]\times \cdots\times [0,(1-\lambda)a_n+\lambda b_n]
\]

有 
\begin{align*}
V_n((1-\lambda)K+\lambda L)&=\prod_{i=1}^n((1-\lambda)a_i+\lambda b_i)  \\
&\ge \prod_{i=1}^n\big(a_i^{1-\lambda}b_i^\lambda\big) \tag{\hyperref[lem:young]{Young 不等式}}\\
&=\left(\prod_{i=1}^na_i\right)^{1-\lambda}\left(\prod_{i=1}^nb_i\right)^\lambda  \\
&=V_n(K)^{1-\lambda}V_n(L)^\lambda
\end{align*}

可见,当 $K$ 和 $L$ 都是 $n$ 维盒子时,\emph{BM 不等式就是 Young 不等式}。

第二步,证明 $K$ 和 $L$ 都是有限多个不交盒子之并的情况。用归纳法,对盒子的个数进行归纳,设 BM 不等式对``$K\cup L$ 一共是不超过 $N$ 个盒子之并"成立,假设现有共 $N+1$ 个盒子且 $K$ 至少有两个。

我们记 $e_n$ 是第 $n$ 个单位坐标向量,考虑用超平面 $H:=H_{e_n,0}=\{x\in\mathbb{R}^n:\langle x,e_n\rangle=0\}$``切" $K$ 和 $L$,并且,不失一般性地,设 $H$ 把 $K$ 中的两个盒子分离开,见图~\ref{fig:HO_cut}。

\begin{figure}
    \centering
    \includegraphics[width=0.6\textwidth]{figures/HO_cut.png}
    \caption{用超平面切割盒子。}
    \label{fig:HO_cut}
\end{figure}

记 $H$ 将空间分出的两个半平面分别为 $H^+$ 和 $H^-$,并记 
\begin{align*}
K^+:=K\cap H^+,&\quad K^-:=K\cap H^-\\
L^+:=L\cap H^+,&\quad L^-:=L\cap H^-
\end{align*}

那么:$K^+\cup L^+$、$K^-\cup L^-$ 这两个集合都是不超过 $N$ 个盒子之并(因为至少 $K^+$ 会比 $K$ 少一个盒子、$K^-$ 会比 $K$ 少一个盒子)。我们对 $L$ 进行适当的平移,使得 $L^+$ 与 $L^-$ 两部分的体积比例与 $K^+$、$K^-$ 体积比例相同,即
\[
\begin{aligned}
\alpha&=\frac{V_n(K^+)}{V_n(K)}=\frac{V_n(L^+)}{V_n(L)}\\
1-\alpha&=\frac{V_n(K^-)}{V_n(K)}=\frac{V_n(L^-)}{V_n(L)}
\end{aligned}
\]

由于有 
\begin{align*}
&K+L\supseteq (K^++L^+)\cup(K^-+L^-)\\
\implies\;& V_n(K+L)\ge V_n(K^++L^+)+V_n(K^-+L^-)
\end{align*}

我们用归纳假设,
\begin{align*}
V_n(K+L)&\ge V_n(K^++L^+)+V_n(K^-+L^-)\\
&\ge \left(V_n(K^+)^{1/n}+V_n(L^+)^{1/n}\right)^n+\left(V_n(K^-)^{1/n}+V_n(L^-)^{1/n}\right)^n\\
&= \left[\left(\frac{V_n(K^+)}{V_n(K)}\right)^{1/n}V_n(K)^{1/n}+\left(\frac{V_n(L^+)}{V_n(L)}\right)^{1/n}V_n(L)^{1/n}\right]^n\\
&\quad +\left[\left(\frac{V_n(K^-)}{V_n(K)}\right)^{1/n}V_n(K)^{1/n}+\left(\frac{V_n(L^-)}{V_n(L)}\right)^{1/n}V_n(L)^{1/n}\right]^n\\
&=\alpha\left(V_n(K)^{1/n}+V_n(L)^{1/n}\right)^n+(1-\alpha)\left(V_n(K)^{1/n}+V_n(L)^{1/n}\right)^n\\
&=\left(V_n(K)^{1/n}+V_n(L)^{1/n}\right)^n
\end{align*}
这就完成了归纳递推。这种非常巧妙的``切割"技术被称作 \emph{Hadwiger–Ohmann cut}。

第三步:令 $n\to\infty$,任何一个 Lebesgue 可测集合都可以用盒子来逼近,且这些盒子的体积总和与 Lebesgue 测度一致(读者可以回忆一下 Lebesgue 外测度的定义 :-)。
\end{proof}



定理~\ref{thm:BM-general} 的结论由 H. Hadwiger 与 D. Ohmann 于 1956 年证明\footnote{H. Hadwiger and D. Ohmann (1960). \textit{Brunn-Minkowskischer Satz und Isoperimetrie}. Journal of the Society for Industrial and Applied Mathematics.}。回顾证明可见,第二、三步其实只是一些技术上的处理,证明本质还是来源于第一步的 Young 不等式。因而,我们可以肯定:BM 不等式就是 Young 不等式在集合上面的推广。

定理~\ref{thm:BM-general} 的取等条件是:取等当且仅当 $K$、$L$ 是位似的凸集(至多相差一个零测集)\footnote{确切来说,这指的是存在凸集$Q\subset\mathbb R^n$ 以及$c_1,c_2>0$、$x_1,x_2\in\mathbb{R}^n$,使得 $c_1K+x_1\subset Q$ 且 $c_2L+x_2\subset Q$,并且 $V_n(Q\setminus (c_1K+x_1))=V_n(Q\setminus (c_2L+x_2))=0$。Alessio Figalli 做了很多这方面的工作。},这个比较繁琐,不再赘述。
 

\chapter{BM 不等式的泛函推广}
\label{chapter:2}

\section{泛函形式的 BM 不等式?}

%\hyperref[thm:BM-general]{BM 不等式} 可以看作 \hyperref[lem:young]{Young 不等式}在集合上的推广。
我们知道,很多几何对象可以``泛函化",而且相比于集合,泛函能表达的东西也更加丰富。一个自然的问题便是,能否有 BM 不等式的泛函形式?
这就需要先想办法把集合“提升”到函数。这其实很简单,最方便且好用的方法就是用集合的\emph{示性函数} (Indicator Function),对 $E\subseteq\mathbb{R}^n$:
\[
1_E(x):=
\begin{cases}
1, & \text{若 } x\in E\\
0, & \text{若 } x\notin E
\end{cases}
\]

接着又可以想到:之前用到的一些定义在集合上的东西也可以改成示性函数的形式,并且进一步扩展到一般的函数上。例如,集合 $E$ 可以定义``体积":
\[
V_n(E):=\int_E\mathrm{d}x=\int_{\mathbb{R}^n}1_E\mathrm{d}x
\]
因而对于 $\mathbb{R}^n$ 上的非负可积函数 $f$,我们可以自然地将
\[
\int_{\mathbb{R}^n}f\mathrm{d}x=\|f\|_1
\]
视作函数 $f$ 的``体积"。

集合的 Minkowski 加法为
\[
E+F:=\{x+y:x\in E,y\in F\}
\]
来看一下如何用示性函数来表示:设 $z=x+y$,分如下四种情况

\begin{enumerate}
\item $x\in E,y\in F\implies z\in E+F$,此时 $1_{E+F}(z)=1=1_E(x)1_F(y)$;
\item $x\in E,y\notin F\implies z\notin E+F$,此时 $1_{E+F}(z)=0=1_E(x)1_F(y)$;
\item $x\notin E,y\in F\implies z\notin E+F$,此时 $1_{E+F}(z)=0=1_E(x)1_F(y)$;
\item $x\notin E,y\notin F$,$z$ 属于或不属于 $E+F$ 都有可能,$1_E(x)1_F(y)=0\le 1_{E+F}(z)=\text{0 或 1}$。
\end{enumerate}

可以看出:无论如何,如果 $1_{E+F}(z)=0$,那么 $1_E(x)1_F(y)$ 必然也总是 0;如果 $1_{E+F}(z)=1$,那么 $1_E(x)1_F(y)$ 既可能是 0 也可能是 1。于是有
\[
1_{E+F}(z)=\sup_{x+y=z} 1_E(x)1_F(y)
\]

推广至一般的两个函数 $f,g:\mathbb{R}^n\to[0,\infty)$,
\begin{definition}
\[
(f\oplus g)(z):=\sup_{x+y=z}f(x)g(y)
\]
这叫做``\emph{Asplund 和}"。
\end{definition}
上述定义可以视作是函数空间对应于``Minkowski 加法"的运算。之前的分析表明 $1_E\oplus 1_F=1_{E+F}$。

集合的“数乘”定义为 $\lambda E:=\{\lambda x:x\in E\}$($\lambda > 0$),有
\[
1_{\lambda E}(x)=
\begin{cases}
1, & \text{若 } \frac{x}{\lambda}\in E\\
0, & \text{若 } \frac{x}{\lambda}\notin E
\end{cases}
\]
类似地,我们\textbf{定义}一个实数与函数的``乘法":
\begin{definition}
\[
(\lambda \cdot f)(x):=f^\lambda\left(\frac{x}{\lambda}\right)
\]    
\end{definition}
这种``乘法"运算满足 $\lambda \cdot 1_E=1_{\lambda E}$。

至此,有了泛函的``体积",``加法"和``乘法"。回顾一下 BM 不等式(维度无关版本,见定理 \ref{thm:BM-equivalence}):%设 $E,F\subset\mathbb{R}^n$ 可测,$\lambda\in[0,1]$,$(1-\lambda)E+\lambda F$ 可测,则有
\[
V_n((1-\lambda) E+\lambda F)\ge V_n(E)^{1-\lambda}V_n(F)^\lambda
\]
用示性函数可将之改写为
\[
\int_{\mathbb{R}^n}1_{(1-\lambda)E+\lambda F}\,\mathrm{d}x\ge \left(\int_{\mathbb{R}^n}1_{E}\,\mathrm{d}x\right)^{1-\lambda}\left(\int_{\mathbb{R}^n}1_{F}\,\mathrm{d}x\right)^{\lambda}
\]
于是便\emph{猜想}:在泛函中,可能会有对应的结论:对 $\lambda\in(0,1)$,
\[
\int_{\mathbb{R}^n}(1-\lambda)\cdot f\oplus \lambda \cdot g\,\mathrm{d}x\ge \left(\int_{\mathbb{R}^n}f\,\mathrm{d}x\right)^{1-\lambda}\left(\int_{\mathbb{R}^n}g\,\mathrm{d}x\right)^\lambda
\]
其中
\[
((1-\lambda) \cdot f \oplus \lambda \cdot g)(z):=\sup_{(1-\lambda)x+\lambda y=z}f(x)^{1-\lambda}g(y)^{\lambda}
\]
这个形式还是稍微有些复杂,不妨把 $(1-\lambda) \cdot f \oplus \lambda \cdot g$ 替换成一个新的非负函数 $h$,它满足:对任意 $x,y\in\mathbb{R}^n$,有
\[
h\big((1-\lambda)x+\lambda y\big)\ge f(x)^{1-\lambda}g(y)^\lambda
\]

最终,我们就得到了——

\section{Prékopa–Leindler 不等式}

\subsection{证明}

\begin{theorem}[Prékopa–Leindler 不等式]\label{thm:PL}
设 $f,g,h$ 是 $\mathbb{R}^n$ 上非负的 Lebesgue 可积函数,满足对任意 $\lambda\in(0,1)$、$x,y\in\mathbb{R}^n$ 都有
\[
h\big((1-\lambda)x+\lambda y\big)\ge f(x)^{1-\lambda}g(y)^{\lambda}
\]
那么,如下不等式成立
\[
\int_{\mathbb{R}^n}h\mathrm{d}x\ge\left(\int_{\mathbb{R}^n}f\mathrm{d}x\right)^{1-\lambda}\left(\int_{\mathbb{R}^n}g\mathrm{d}x\right)^{\lambda} \label{eq:2-1}\tag{2-1}
\]
\end{theorem}
\begin{proof}
    
分两步走,

\begin{enumerate}[label=\textbf{第\chinese*步}:, leftmargin=6em]
\item 证明一维即 $n=1$ 的情形;
\item 用数学归纳法,对 $n$ 进行归纳。
\end{enumerate}

第一步:设 $n=1$,首先注意到,若 $\displaystyle\int_{\mathbb{R}}f\mathrm{d}x=0$ 或者 $\displaystyle\int_{\mathbb{R}}g\mathrm{d}x=0$,那么不等式显然成立,排除掉这一情况。记
\[
F:=\int_{\mathbb{R}}f\,\mathrm{d}x>0,\quad G:=\int_{\mathbb{R}}g\,\mathrm{d}x>0
\]
或者也可以写作
\[
\int_{\mathbb{R}}\frac{f}{F}\mathrm{d}x=\int_{\mathbb{R}}\frac{g}{G}\mathrm{d}x=1 \label{eq:2-2}\tag{2-2}
\]
接下来,定义两个辅助函数 $u,v:[0,1]\to\mathbb{R}$ 使得 $u(t)$ 和 $v(t)$ 是最小的值满足
\[
\int_{-\infty}^{u(t)}\frac{f}{F}\mathrm{d}x=\int_{-\infty}^{v(t)}\frac{g}{G}\mathrm{d}x=t
\]
考察一下这两个函数的性质

\begin{itemize}
\item $u,v$ 不一定在 $[0,1]$ 上连续,不过这没问题;
\item $u,v$ 在 $[0,1]$ 上严格单调上升;
\item 由于上一条,它们的导数 $u',v'$ 也几乎处处存在;
\item 于是,几乎处处成立
\[
\begin{cases}
1=\displaystyle \frac{\mathrm{d}}{\mathrm{d}t}\int_{-\infty}^{u(t)}\frac{f}{F}\mathrm{d}x=\frac{f(u(t))u'(t)}{F}\\
1=\displaystyle \frac{\mathrm{d}}{\mathrm{d}t}\int_{-\infty}^{v(t)}\frac{g}{G}\mathrm{d}x=\frac{g(v(t))v'(t)}{G}
\end{cases}
\]
\end{itemize}

下面定义 $u(t)$ 和 $v(t)$ 的凸组合(或者说插值)
\[
w(t):=(1-\lambda)u(t)+\lambda v(t)
\]
则 $w(t)$ 也是严格单调上升,因而在 $[0,1]$ 上导数几乎处处存在。Young 不等式此时又派上用场了:
 
\begin{align*}
w'(t)&=(1-\lambda)u'(t)+\lambda v'(t)\\
&\ge (u'(t))^{1-\lambda}(v'(t))^\lambda \tag{\hyperref[lem:young]{Young 不等式}}\\
&=\left(\frac{F}{f(u(t))}\right)^{1-\lambda}\left(\frac{G}{g(v(t))}\right)^{\lambda} \label{eq:2-3}\tag{2-3}
\end{align*}
 
因此
 
\begin{align*}
\int_{\mathbb{R}}h(x)\mathrm{d}x&\ge \int_0^1 h(w(t))w'(t)\mathrm{d}t\\
&=\int_0^1 h\big((1-\lambda)u(t)+\lambda v(t)\big)w'(t)\mathrm{d}t\\
&\ge\int_0^1 f(u(t))^{1-\lambda}g(v(t))^\lambda \left(\frac{F}{f(u(t))}\right)^{1-\lambda}\left(\frac{G}{g(v(t))}\right)^{\lambda}\mathrm{d}t\\
&=F^{1-\lambda}G^\lambda\\
&=\left(\int_\mathbb{R}f\mathrm{d}x\right)^{1-\lambda}\left(\int_\mathbb{R}g\mathrm{d}x\right)^{\lambda}
\end{align*}
 
这里第一行的不等式是因为 $w(t)$ 的值域不一定能覆盖 $\mathbb{R}$,因而积分换元可能存在“损失”;第三行的不等式来自定理假设 ~\eqref{eq:2-1} 以及 ~\eqref{eq:2-3} 式。第一步证毕。

第二步:用归纳法,假设 PL 不等式在维数小于 $n$ 时成立,$n\ge2$。现考虑 $n$ 维的情况,接下来要做的就是想办法把函数 $f,g,h$“降维”。这里要用到一个技巧:以 $h:\mathbb{R}^n\to\mathbb{R}$ 为例,定义函数
\[
h_s(z):=h(z,s),\quad z\in\mathbb{R}^{n-1},s\in\mathbb{R}^1
\]
这样函数 $h_s$ 就是 $\mathbb{R}^{n-1}\to\mathbb{R}$ 的函数。同理定义 $f_s$、$g_s$。

接着,在 $f,g,h$ 满足式 ~\eqref{eq:2-1} 的前提下,我们验证 ~\eqref{eq:2-1} 对 $f_s,g_s,h_s$ 是否还成立。对于任意 $x,y\in\mathbb{R}^{n-1}$、$\lambda\in(0,1)$,记 $(x,a)\in\mathbb{R}^n$、$(y,b)\in\mathbb{R}^n$ 是两个 $n$ 维向量,$a,b\in\mathbb{R}^1$,$c:=(1-\lambda)a+\lambda b$,于是
 
\begin{align*}
&h\big((1-\lambda)(y,a)+\lambda(z,b)\big)\\
=\;&h\big((1-\lambda)y+\lambda z,\,(1-\lambda)a+\lambda b\big)=h_c((1-\lambda)y+\lambda z)\\
\ge\;& f(y,a)^{1-\lambda}g(z,b)^\lambda \tag{由式~\eqref{eq:2-1}}\\
=\; & f_a(y)^{1-\lambda}g_b(z)^\lambda
\end{align*}
 
总结一下,我们得到了这样的一个结论:对任意 $x,y\in\mathbb{R}^{n-1}$ 以及固定的 $a,b\in\mathbb{R}^1$、$c:=(1-\lambda)a+\lambda b$,有
\[
h_c((1-\lambda)x+\lambda y)\ge f_a(x)^{1-\lambda}g_b(y)^\lambda
\]
这就可以用 $n-1$ 维的 PL 不等式(归纳假设)了:
 
\begin{align*}
&\underbrace{\int_{\mathbb{R}^{n-1}}h_c(z)\mathrm{d}z}_{\text{定义为 }H(c)}\ge\bigg(\underbrace{\int_{\mathbb{R}^{n-1}}f_a(x)\mathrm{d}x}_{\text{定义为 }F(a)}\bigg)^{1-\lambda}\bigg(\underbrace{\int_{\mathbb{R}^{n-1}} g_b(y)\mathrm{d}y}_{\text{定义为 }G(b)}\bigg)^\lambda\\
\implies & H(c)=H\big((1-\lambda)a+\lambda b\big)\ge F(a)^{1-\lambda}G(b)^\lambda
\end{align*}
 
哎!可以发现,这又完美满足了 PL 不等式在 1 维的条件。

接着用一维的 PL 不等式,有
 
\begin{align*}
&\int_{\mathbb{R}}H(c)\mathrm{d}c\ge\left(\int_{\mathbb{R}} F(a)\mathrm{d}a\right)^{1-\lambda}\left(\int_{\mathbb{R}} G(b)\mathrm{d}b\right)^\lambda\\
\implies & \int_{\mathbb{R}}\int_{\mathbb{R}^{n-1}}h_c(z)\mathrm{d}z\mathrm{d}c\ge\left(\int_{\mathbb{R}}\int_{\mathbb{R}^{n-1}}f_a(x)\mathrm{d}x\mathrm{d}a\right)^{1-\lambda}\left(\int_{\mathbb{R}}\int_{\mathbb{R}^{n-1}}g_b(y)\mathrm{d}y\mathrm{d}b\right)^\lambda\\
\implies & \int_{\mathbb{R}}\int_{\mathbb{R}^{n-1}}h(z,c)\mathrm{d}z\mathrm{d}c\ge\left(\int_{\mathbb{R}}\int_{\mathbb{R}^{n-1}}f(x,a)\mathrm{d}x\mathrm{d}a\right)^{1-\lambda}\left(\int_{\mathbb{R}}\int_{\mathbb{R}^{n-1}}g(y,b)\mathrm{d}y\mathrm{d}b\right)^\lambda\\
\implies & \int_{\mathbb{R}^n}h(z)\mathrm{d}z\ge \left(\int_{\mathbb{R}^n}f(x)\mathrm{d}x\right)^{1-\lambda}\left(\int_{\mathbb{R}^n}g(y)\mathrm{d}y\right)^\lambda
\end{align*}
 
最后一步用了 Fubini 定理。% 这就证出了 PL 不等式。
\end{proof}


PL 不等式由 András Prékopa 和 László Leindler 在上世纪 70 年代分别给出。在上述证明中有几个要点:

\begin{itemize}
\item 第一步引入“体积参数”$t$,以及对 $w'(t)$ 用 Young 不等式;
\item 第二步将 $n$ 维情形拆成 $n-1$ 维和 1 维。
\end{itemize}

后来人们又提出了上述第一步的一个更加简单的证明方法,这里补充如下。

\begin{proof}[一维情形的另一种证明]
    
不失一般性地可以假设 $f,g$ 有界且 $\|f\|_\infty=\|g\|_\infty=1$(为什么?)。我们引入一个很常用的工具:\emph{上水平集}(super-level set,也译作超水平集):对于函数 $f:\mathbb{R}\to\mathbb{R}$ 以及常数 $t\in\mathbb{R}$,定义其上水平集为
\[
L(f,t):=\big\{x\in\mathbb{R}:f(x)\ge t\big\}
\]
 

\begin{figure}[h]
    \centering
    \includegraphics[width=0.7\textwidth]{figures/level_set.pdf}
    \caption{上水平集的示意图。}
    \label{fig:level_set}
\end{figure}

根据式 ~\eqref{eq:2-1},对任意的 $t$,若 $f(x)\ge t$、$g(y)\ge t$,那么
\[
h\big((1-\lambda)x+\lambda y\big)\ge f(x)^{1-\lambda}g(y)^\lambda\ge t
\]
翻译一下也可以说,若 $x\in L(f,t)$、$y\in L(g,t)$,那么 $(1-\lambda)x+\lambda y\in L(h,t)$,因而
\[
L(h,t)\supseteq (1-\lambda)L(f,t)+\lambda L(g,t)
\]
由于 $f,g,h$ 可测,则这里涉及的水平集也都是非空的 Borel 可测集,并且 $(1-\lambda)L(f,t)+\lambda L(g,t)$ 也是 Lebesgue 可测的
\footnote{ 对任意非空的 Borel 可测集 $X,Y$,Minkowski 和 $(1-\lambda)X+\lambda Y$ 都可以看作是 $X\times Y$ 在连续函数 $(x,y)\mapsto \lambda x+(1-\lambda)y$ 下的像,因而是解析集,故 Lebesgue 可测。
因此,我们能放心讨论 Borel 可测集合的和;如果改成两个 Lebesgue 可测集的和就不行了。
}。

则根据一维的 BM 不等式,
\[
V_1(L(h,t))\ge (1-\lambda)V_1(L(f,t))+\lambda V_1(L(g,t))
\]
如何把``上水平集的体积"转变回函数呢?这用到第二个重要的技巧:设 $f$ 是非负实值可测函数,那么
 
\begin{align*}
\int_{\mathbb{R}}f(x)\mathrm{d}x&=\int_{\mathbb{R}}\int_0^{f(x)}\mathrm{d}t\mathrm{d}x\\
&=\int_0^\infty\int_{\{x:f(x)\ge t\}}\mathrm{d}x\mathrm{d}t \tag{Fubini}\\
&=\int_0^\infty V_1(L(f,t))\mathrm{d}t \label{eq:2-4}\tag{2-4}
\end{align*}
 
这也叫``\emph{千层饼表示} (Layer-cake representation)"\footnote{\url{https://en.wikipedia.org/wiki/Layer_cake_representation}}。你可以把 $V_n(L(f,t))$ 看成一个长方体的长、而 $\mathrm{d}t$ 看成长方体的宽,而非负函数 $f$ 与 $x$ 轴围成的面积就可以用这些长方体来覆盖,见图~\ref{fig:layer_cake}。
 
 

\begin{figure}
    \centering
    \includegraphics[width=0.7\textwidth]{figures/layer_cake.png}
    \caption{``千层饼表示":用与横轴平行的长方形来覆盖函数下的面积。\footnotemark}
    \label{fig:layer_cake}
\end{figure}
\footnotetext{图片来自 \url{https://math.stackexchange.com/q/4631735}}

回到证明,
 
\begin{align*}
\int_{\mathbb{R}}h\,\mathrm{d}x&=\int_0^\infty V_1(L(h,t))\,\mathrm{d}t\\
&\ge\int_0^\infty \Big((1-\lambda)V_1(L(f,t))+\lambda V_1(L(g,t))\Big)\,\mathrm{d}t\\
&=(1-\lambda)\int_0^\infty V_1(L(f,t))\,\mathrm{d}t+\lambda\int_0^\infty V_1(L(g,t))\,\mathrm{d}t\\
&=(1-\lambda)\int_{\mathbb{R}} f\,\mathrm{d}x+\lambda \int_{\mathbb{R}}g\,\mathrm{d}x\\
&\ge \left(\int_{\mathbb{R}}f\,\mathrm dx\right)^{1-\lambda}\left(\int_{\mathbb{R}}g\,\mathrm{d}x\right)^{\lambda} \tag{\hyperref[lem:young]{Young 不等式}}
\end{align*}
证讫。
\end{proof}


\begin{remark}
\begin{itemize}
\item 该证明可以参考 Pisier, G. (1999). \textit{The volume of convex bodies and Banach space geometry} (Vol. 94). Cambridge University Press. 里面的 Lemma 1.2 (pp. 3-4)。主要思想就是利用上水平集 $L(f,t)$“把函数转成集合”,对集合用 BM 不等式,再利用~\eqref{eq:2-3}“把集合转成函数”。
\item 请思考:这种证明思路为什么\emph{不能}推广到 $n$ 维? 
\item 式~\eqref{eq:2-4} 其实大有文章,其中 $V_n(L(f,\mu))$ 和``分布函数"的概念差不多(distribution function,这是分析里的概念,不是概率论里那个分布函数)。对于测度空间 $(\Omega,\mathcal M,\mu)$ 上一个实值可测函数 $f$,定义其\emph{分布函数} $\lambda_f:(0,\infty)\to[0,\infty]$ 为
\[
\lambda_f(t):=\mu\big(\big\{x:|f(x)|>t\big\}\big)
\]
分布函数很重要的功能就是把对 $f$ 的 Lebesgue 积分与一个广义 Riemann 积分对应起来。式 ~\eqref{eq:2-3} 可以推广成这样的结论:若 $0<p<\infty$,那么
\[
\int_\Omega |f|^p\,\mathrm{d}\mu= p\int_0^\infty t^{p-1}\lambda_f(t)\,\mathrm{d}t
\]
推荐参阅 Folland, G. B. (1999). \textit{Real analysis: modern techniques and their applications}. John Wiley \& Sons. 的 §6.4。
\end{itemize}
\end{remark}

\subsection{最优传输的视角}

\emph{最优传输} (Optimal Transport, OT) 来自于 Gaspard Monge 提出的一个 ``运输沙子" 的问题, 见图~\ref{fig:optimal_transport}。简而言之,设 $\mu$ 和 $\nu$ 分别是空间 $X$、$Y$ 上的概率测度——你可以将 $\mu$ 想象为一堆沙子,而 $\nu$ 是一个我们想用沙子填满的洞。$c:X\times Y\to [0,\infty]$ 称为\emph{代价函数} (cost function),$c(x,y)$ 代表将单位质量的沙子从 $x\in X$ 运输到 $y\in Y$ 的成本。

% \begin{center}
% \includegraphics[width=0.7\textwidth]{https://picx.zhimg.com/v2-f66684d2575f1f17854281d9327ddc69_720w.jpg?source=d16d100b}
% \end{center}

% \textbf{\url{https://www.cit.tum.de/fileadmin/w00byx/cit/Studium/Studiengaenge/TopMath/TopMath_Studierende/Poster/Bachelorposter_Voegler.pdf}}

\begin{figure}
    \centering
    \includegraphics[width=0.7\textwidth]{figures/OT_1.png}
    \caption{Monge 的``运输沙子问题" \footnotemark。}
    \label{fig:optimal_transport}
\end{figure}
\footnotetext{图片来自 \url{https://www.cit.tum.de/fileadmin/w00byx/cit/Studium/Studiengaenge/TopMath/TopMath_Studierende/Poster/Bachelorposter_Voegler.pdf}}

\textbf{定义:} 在上述记号下,如果 $T:X\to Y$ 满足对任意的 $\nu$ 可测集 $A$,有
\[
\nu(A)=\mu(T^{-1}(A))
\]
则称 $T$ 将 $\mu$``\emph{传输} (transport)"到 $\nu$,文献里一般记作 $T\#\mu=\nu$;也称 $T$ 是一个 ``传输映射",见图~\ref{fig:transport_map}。

% \begin{center}
% \includegraphics[width=0.7\textwidth]{https://pic1.zhimg.com/v2-968bc37030b2e359ceaa7b14dabf6ce2_720w.jpg?source=d16d100b}
% \end{center}

\begin{figure}
    \centering
    \includegraphics[width=0.5\textwidth]{figures/OT_2.png}
    \caption{传输映射 $T$ 将测度 $\mu$ 传输到测度 $\nu$ \footnotemark。}
    \label{fig:transport_map}
\end{figure}
\footnotetext{图片来自 雷娜, 顾险峰. (2021). \textit{最优传输理论与计算}. 高等教育出版社. p. 24.}



最优传输问题便是找到一个传输 $T$ 最小化下述总代价:
\[
\int_X c(x,T(x))\mathrm{d}\mu(x)
\]

\begin{theorem*}[Brenier 定理]\label{thm:brenier}
设 $X=Y=\mathbb{R}^n$,代价函数为 $c(x,y)=\frac12\|x-y\|^2$,如果 $\mu$ 对 Lebesgue 测度绝对连续,则存在一个唯一的最优传输映射 $T$。并且,存在某个凸函数 $\varphi:\mathbb{R}^n\to\mathbb{R}$,使得 $T=\nabla \varphi$,称 $\varphi$ 为 \emph{Brenier 映射}。更进一步地,有
\[
\mathrm{det}\big(\nabla^2\varphi(x)\big)=\frac{\mathrm{d}\mu(x)}{\mathrm{d}\nu(T(x))}
\]
\end{theorem*}

我们可以从最优传输的视角为 PL 不等式提供一个非常漂亮的证明。

\begin{proof}[PL 不等式的 OT 证明]
    
思路与定理 4 第一步的证明差不多,但是可以直接推广到一般的 $n$ 维,不借助归纳法。和 ~\eqref{eq:2-2} 式一样,依旧是设 $\displaystyle F:=\int_{\mathbb{R}^n}f(x)\mathrm{d}x>0$、$\displaystyle G:=\int_{\mathbb{R}^n}g(x)\mathrm{d}x>0$,考虑这样的两个概率测度:
\[
\mathrm{d}\mu(x):=\frac{f(x)}{F}\mathrm{d}x,\quad \mathrm{d}\nu(x):=\frac{g(x)}{G}\mathrm{d}x
\]
记 $T$ 是 $\mu$ 到 $\nu$ 的最优传输,则 Brenier 映射 $\varphi$ 满足
\[
\mathrm{det}\big(\nabla^2\varphi(x)\big)=\frac{\mathrm{d}\mu(x)}{\mathrm{d}\nu(T(x))}=\frac{f(x)}{F}\frac{G}{g(T(x))}
\]

下面考虑凸组合
\[
z_\lambda(x):=(1-\lambda)x+\lambda T(x)
\]
则其 Jacobian 为
\[
\nabla z_\lambda =(1-\lambda)\mathbf{I}_n+\lambda\nabla^2\varphi
\]
由于 $\varphi$ 凸,则其 Hessian $\nabla^2\varphi$ 半正定,我们总可以将之特征值分解:$\nabla^2\varphi=Q\Sigma Q^{-1}$,其中 $\Sigma=\mathrm{diag}(s_1,s_2,\ldots,s_n)$。于是
 
\begin{align*}
\mathrm{det}\big(\nabla z_\lambda\big)&=\mathrm{det}\big((1-\lambda)\mathbf{I}_n +\lambda\nabla^2\varphi\big)\\
&=\mathrm{det}\big((1-\lambda)\mathbf{I}_n +\lambda \Sigma  \big)\\
&=\prod_{i=1}^n\Big((1-\lambda)+\lambda s_i\Big)\\
&\ge\bigg(\prod_{i=1}^n s_i\bigg)^\lambda \tag{\hyperref[lem:young]{Young 不等式}}\\
&=\big(\mathrm{det}\big(\nabla^2\varphi\big)\big)^\lambda \label{eq:2-5}\tag{5}
\end{align*}
 
则对满足 PL 不等式条件 ~\eqref{eq:2-1} 的 $f,g,h:\mathbb{R}^n\to\mathbb{R}$,
\begin{align*}
\int_{\mathbb{R}^n}h(z)\,\mathrm{d}z&\ge\int_{\mathbb{R}^n}h\big((1-\lambda)x+\lambda T(x)\big)\mathrm{d}z_\lambda \tag{换元}\\
&\ge \int_{\mathbb{R}^n}f(x)^{1-\lambda}g(T(x))^\lambda\mathrm{d}z_\lambda \tag{由式~\eqref{eq:2-1}}\\
&\ge\int_{\mathbb{R}^n}f(x)^{1-\lambda}g(T(x))^\lambda\left(\frac{f(x)}{F}\frac{G}{g(T(x))}\right)^\lambda\mathrm{d}x \tag{由式~\eqref{eq:2-5}}\\
&=\left(\frac{G}{F}\right)^\lambda\underbrace{\int_{\mathbb{R}^n}f(x)\mathrm{d}x}_{F}\\
&=F\left(\frac{G}{F}\right)^\lambda=F^{1-\lambda}G^\lambda\\
&=\left(\int_{\mathbb{R}^n}f\,\mathrm{d}x\right)^{1-\lambda}\left(\int_{\mathbb{R}^n}g\,\mathrm{d}y\right)^{\lambda}
\end{align*}
 
\end{proof}


\subsection{其它说明}

\begin{enumerate}
\item PL 不等式也可以反过来推出 BM 不等式:对集合 $K,L$,取 $f=1_K$、$g=1_L$,$h=1_{(1-\lambda) f+\lambda g}$ 即可。
\item PL 不等式的取等条件比较麻烦,这里只叙述一下结论:PL 不等式取等当且仅当 $f,g,h$ 满足:存在 $c>0$、$x_0\in\mathbb{R}^n$,以及一个可积的对数凹函数 $\psi$ 使得几乎处处成立
\begin{equation*}\left\{
\begin{aligned}
h(x)&=\psi(x)\\
f(x)&=c^{-\lambda}\psi(x-\lambda x_0)\\
g(x)&=c^{1-\lambda}\psi\big(x+(1-\lambda)x_0\big)
\end{aligned}\right.
\end{equation*}
这个条件由 S. Dubuc \footnote{Dubuc, S. (1977). \textit{Critères de convexité et inégalités intégrales}. In Annales de l'institut Fourier (Vol. 27, No. 1, pp. 135-165).}给出。可见,$f$ 与 $g$ 之间必须也存在类似于集合的“位似”关系。

\item 与 Hölder 不等式的联系。

\begin{theorem*}[$\mathbb{R}^n$ 上的 Hölder 不等式]
设非负函数 $f_i\in L^{p_i}(\mathbb{R}^n)$,其中 $p_i\ge 1$,$i=1,\ldots,m$,$\displaystyle\sum_{i=1}^m\frac1{p_i}=1$,则
\[
\int_{\mathbb{R}^n}\prod_{i=1}^mf_i(x)\,\mathrm{d}x\leq\prod_{i=1}^m\|f_i\|_{p_i}=\prod_{i=1}^m\left(\int_{\mathbb{R}^n}f_i(x)^{p_i}\,\mathrm{d}x\right)^{1/p_i}
\]
\end{theorem*}

%\textbf{证明(概要):} 对 $m=2$ 用 Young 不等式;对 $m> 2$ 可以用归纳法(或者 Jensen 不等式)证明 Young 不等式的 $n$ 元情形,亦称“加权 AM-GM 不等式”(可见 \url{https://www.zhihu.com/question/424154792} 或者 \url{https://www.zhihu.com/question/611603291})。$\square$

记 $\lambda\in(0,1)$,在上述结论中取 $m=2$、$1/p_1=1-\lambda$、$1/p_2=\lambda$,函数 $f,g:\mathbb{R}^n\to\mathbb{R}$ 定义为 $f=f_1^{p_1}$、$g=f_2^{p_2}$,那么 Hölder 不等式可以改写为
\[
\int_{\mathbb{R}^n} f^{1-\lambda} g^{\lambda}\,\mathrm{d}x\le\left(\int_{\mathbb{R}^n} f\,\mathrm{d}x\right)^{1-\lambda}\left(\int_{\mathbb{R}^n}g\,\mathrm{d}x\right)^{\lambda}
\]
可以看出,\emph{PL 不等式刚好是 Hölder 不等式的反向}。

我们可以把 PL 不等式换个形式,写作
\[
\overline{\int}_{\mathbb{R}^n}\sup\{f(x)^{1-\lambda}g(y)^\lambda:(1-\lambda)x+\lambda y=z\}\,\mathrm{d}z\geq\left(\int_{\mathbb{R}^n}f(x)\,\mathrm{d}x\right)^{1-\lambda}\left(\int_{\mathbb{R}^n}g(x)\,\mathrm{d}x\right)^\lambda
\]
其中 $\overline\int$ 代表``上 Lebesgue 积分(Upper Lebesgue integral)"(因为对那个 $\sup$ 项的积分不一定存在);这种形式能自然地推广至 $n$ 元情形:
\[
\overline{\int}_{\mathbb{R}^n}\sup\left\{\prod_{i=1}^mf_i(x_i):\sum_{i=1}^m\frac{x_i}{p_i}=z\right\}\,\mathrm{d}z\geq\prod_{i=1}^m\|f_i\|_{p_i}=\prod_{i=1}^m\left(\int_{\mathbb{R}^n}f_i(x)^{p_i}\,\mathrm{d}x\right)^{1/p_i}
\]
\end{enumerate}

\section{Borell-Brascamp-Lieb 不等式}

既知 BM 不等式有多种等价形式 (定理~\ref{thm:BM-equivalence}),其中有维度相关的,也有维度无关的。PL 不等式是一种``维度无关"的形式,自然,我们会想:它是否也有一个维度相关的形式?下面要介绍的 Borell-Brascamp-Lieb (BLL) 不等式便是。

\subsection{准备和证明}

在介绍 BLL 不等式之前,先做一些准备。

\begin{definition}
    
设 $\lambda\in(0,1)$,实数 $a,b\ge 0$。

\begin{itemize}
\item 当 $a,b$ 均不为 0 时,
\begin{itemize}
\item 若 $p\ne 0$,定义 $M_p(a,b,\lambda):=((1-\lambda)a^p+\lambda b^p)^{1/p}$
\item 当 $p=0$ 时,特别定义 $M_0(a,b,\lambda):=a^{1-\lambda}b^\lambda$;
\item 当 $p=\infty$ 时,取 $M_\infty(a,b,\lambda):=\max\{a,b\}$;当 $p=-\infty$ 时,取 $M_{-\infty}(a,b,\lambda):=\min\{a,b\}$。
\end{itemize}
\item 当 $ab=0$ 时,对任意的 $p\in[-\infty,\infty]$,都取 $M_p(a,b,\lambda)=0$。
\end{itemize}
\end{definition}

该记号是幂平均(power mean)的自然推广。其性质有:

\begin{itemize}
\item 对 $ab\ne 0$,$\displaystyle \frac1{M_p(a,b,\lambda)}=M_{-p}\left(\frac1a,\frac1b,\lambda\right)$;
\item 对 $c\ge 0$,有 $M_p(ca,cb,\lambda)=c\cdot M_p(a,b,\lambda)$;
\item 若 $q>p$ 那么 $M_q(a,b,\lambda)\ge M_p(a,b,\lambda)$(可由 Jensen 不等式得出);
\item 极限性质:$\lim\limits_{p\to 0}M_p(a,b,\lambda)=M_0(a,b,\lambda)$;$\lim\limits_{p\to \infty}M_p(a,b,\lambda)=M_\infty(a,b,\lambda)$;$\lim\limits_{p\to -\infty}M_p(a,b,\lambda)=M_{-\infty}(a,b,\lambda)$。
\end{itemize}

\begin{lemma}\label{lem:power_mean_inequality}
设 $\lambda\in(0,1)$,$a,b,c,d\ge 0$,$p\in\mathbb{R}\cup\{+\infty\}$、$q\in\mathbb{R}$。若 $p+q\ge 0$,那么
\[
M_p(a,b,\lambda)M_q(c,d,\lambda)\ge M_r(ac,bd,\lambda)
\]
其中,当 $pq=0$ 时,$r=0$;当 $p=-q\ne 0$ 时 $r=-\infty$;除此之外,$\frac1r=\frac{1}{p}+\frac{1}{q}$,即 $r=\frac{pq}{p+q}$。
\end{lemma}
\begin{remark}
    该引理来自 \url{https://faculty.gardner.wwu.edu/gorizia12.pdf} 的 Lemma 10.1。
\end{remark}
\begin{proof}
    
仅证明 $p,q>0$ 的情形。由 Hölder 不等式,
 
\begin{align*}
\left(1+\frac{\lambda b^{p}}{(1-\lambda)a^{p}}\right)^{p}\left(1+\frac{\lambda d^{q}}{(1-\lambda)c^{q}}\right)^{q}&\ge\left(1+\left(\frac{\lambda^{1/p}b}{(1-\lambda)^{1/p}a}\frac{\lambda^{1/q}d}{(1-\lambda)^{1/q}c}\right)^r\right)^{1/r}\\
&=\left(1+\frac{\lambda}{1-\lambda}\left(\frac{bd}{ac}\right)^r\right)^{1/r}
\end{align*}
 
这等价于
\[
\big((1-\lambda)a^p+\lambda b^p\big)^{1/p}+\big((1-\lambda)c^q+\lambda d^q\big)^{1/q}\ge \big((1-\lambda)(ac)^r+\lambda(bd)^r\big)^{1/r}
\]
正是
\[
M_p(a,b,\lambda)M_q(c,d,\lambda)\ge M_{r}(ac,bd,\lambda)
\]

其余情况容易,举个例子,$p<0$,$q,r>0$ 的情况,我们用 $-p$ 替换 $p$,再利用 $\displaystyle M_p(a,b,\lambda)=\frac1{M_{-p}\left(\frac1a,\frac1b,\lambda\right)}$,证明目标就变成了 $M_q(c,d,\lambda)\ge M_r(ac,bd,\lambda)M_{-p}\left(\frac1a,\frac1b,\lambda\right)$,正是已经证明的情况。从略。
\end{proof}



\begin{theorem}[Borell-Brascamp-Lieb 不等式]\label{thm:BBL}
设 $\lambda\in(0,1)$,$n\ge 1$,$p\in\left[-\frac1n,\infty\right]$,$f,g,h$ 是 $\mathbb{R}^n$ 上非负可积的函数,满足对任意 $x,y\in\mathbb{R}^n$,
\begin{equation}
h((1-\lambda)x+\lambda y)\ge M_p(f(x),g(y),\lambda) \label{eq:2-6}\tag{2-6}
\end{equation}
那么定义 $q:=\frac{p}{1+np}$,成立如下的不等式
\[
\int_{\mathbb{R}^n}h \,\mathrm{d}x\geq M_{q}\left(\int_{\mathbb{R}^n}f \,\mathrm{d}x,\int_{\mathbb{R}^n}g \,\mathrm{d}x,\lambda\right)
\]
\end{theorem}

\begin{remark}
\begin{itemize}
\item 当 $p=0$ 时,BBL 不等式正是 PL 不等式;在后面的证明中,我们排除掉 $p=0$ 的情形。
\item 定理中 $q:=\frac{p}{1+np}$,在特殊取值时需要``自然"地理解——当 $p=-\frac1n$ 时,我特别定义 $q=-\infty$;$p=\infty$ 时定义 $q=-\frac1n$。
\item 当 $p\in\left(-\frac{1}{n},0\right)\cup(0,\infty)$ 时,BBL 等式还可以写作
\[
\begin{cases}
\displaystyle\left(\int_{\mathbb{R}^n}h\,\mathrm{d}x\right)^{\frac{p}{1+np}}\ge(1-\lambda)\left(\int_{\mathbb{R}^n}f\,\mathrm{d}x\right)^{\frac{p}{1+np}}+\lambda \left(\int_{\mathbb{R}^n}g\,\mathrm{d}x\right)^{\frac{p}{1+np}},&\text{ 若 }p>0\\
\displaystyle\left(\int_{\mathbb{R}^n}h\,\mathrm{d}x\right)^{\frac{p}{1+np}}\le(1-\lambda)\left(\int_{\mathbb{R}^n}f\,\mathrm{d}x\right)^{\frac{p}{1+np}}+\lambda \left(\int_{\mathbb{R}^n}g\,\mathrm{d}x\right)^{\frac{p}{1+np}},&\text{ 若 }-\frac1n< p<0
\end{cases}
\]
\end{itemize}
\end{remark}

\begin{proof}
    
BBL 是 PL 不等式的推广,其证明过程也与 PL 的证明完全类似。

第一步:证明 $n=1$ 的情形,当 $\displaystyle\int_{\mathbb{R}}f\,\mathrm{d}x$ 或者 $\displaystyle\int_{\mathbb{R}}g\,\mathrm{d}x$ 有一方为 0 时,BBL 不等式右侧项是 0,显然成立。排除掉这种情况,设
\[
F:=\int_{\mathbb{R}}f\mathrm{d}x>0,\quad G:=\int_{\mathbb{R}}g\mathrm{d}x>0
\]
定义两个辅助函数 $u(t)$ 和 $v(t)$,$t\in[0,1]$,使得二者分别是最小的值使得下式成立
\[
\int_{-\infty}^{u(t)}\frac{f}{F}\mathrm{d}x=\int_{-\infty}^{v(t)}\frac{g}{G}\mathrm{d}x=t
\]
以及 $w(t):=(1-\lambda)u(t)+\lambda v(t)$,这些都和 PL 不等式的证明(定理~\ref{thm:PL})一模一样。几乎处处成立
\begin{align*}
w'(t)&=(1-\lambda)u'(t)+\lambda v'(t)=(1-\lambda)\frac{F}{f(u(t))}+\lambda\frac{G}{g(v(t))}\\
&=M_1\left(\frac{F}{f(u(t))},\frac{G}{g(v(t))},\lambda\right)
\end{align*}
于是
\begin{align*}
\int_{\mathbb{R}}h\mathrm{d}x&\ge \int_0^1h(w(t))w'(t)\mathrm{d}t\\
&=\int_0^1h\big((1-\lambda)u(t)+\lambda v(t)\big)M_1\left(\frac{F}{f(u(t))},\frac{G}{g(v(t))},\lambda\right)\mathrm{d}t\\
&\ge \int_0^1 M_p\big(f(u(t)),g(v(t)),\lambda\big)M_1\left(\frac{F}{f(u(t))},\frac{G}{g(v(t))},\lambda\right)\mathrm{d}t \tag{由式~\eqref{eq:2-6}}\\
&\ge \int_0^1 M_{p/(p+1)}(F,G,\lambda)\mathrm{d}x= \big((1-\lambda)F^q+\lambda G^q\big)^{1/q} \tag{引理~\ref{lem:power_mean_inequality}}\\
&=\left((1-\lambda)\left(\int_\mathbb{R}f\,\mathrm{d}x\right)^q+\lambda\left(\int_\mathbb{R}g\,\mathrm{d}x\right)^q\right)^{1/q}
\end{align*}
其中 $q=\frac{p}{p+1}$。

第二步:下面对 $n$ 进行归纳,设 BBL 不等式对小于 $n$ 的正整数成立。用证明 PL 不等式相同的思路,定义
\[
h_s(z):=h(z,s),\,z\in\mathbb{R}^{n-1},s\in\mathbb{R}^1
\]
同理定义 $f_s$、$g_s$。则对 $a,b\in\mathbb{R}$、$c:=(1-\lambda)a+\lambda b$ 有
\begin{align*}
h_c((1-\lambda)x+\lambda y)&=h\big((1-\lambda)x+\lambda y,(1-\lambda)a+\lambda b\big)\\
&=h\big((1-\lambda)(x,a)+\lambda(y,b)\big)\\
&\ge M_p(f(x,a),g(y,b),\lambda) \tag{由式~\eqref{eq:2-6}}\\
&=M_p(f_a(x),g_b(y),\lambda)
\end{align*}
则根据 $n-1$ 维的 BBL 不等式,有
\[
\int_{\mathbb{R}^{n-1}}h_c(z) \,\mathrm{d}z\geq M_{\frac{p}{(n-1)p+1}}\left(\int_{\mathbb{R}^{n-1}}f_a(x) \,\mathrm{d}x,\int_{\mathbb{R}^{n-1}}g_b(y) \,\mathrm{d}y,\lambda\right)
\]
再由一维的 BBL 不等式,
\begin{align*}
&\int_{\mathbb{R}}\int_{\mathbb{R}^{n-1}}h(z,c) \,\mathrm{d}z\mathrm{d}c\geq M_{\frac{p}{np+1}}\left(\int_{\mathbb{R}}\int_{\mathbb{R}^{n-1}}f(x,a) \,\mathrm{d}x\mathrm{d}a,\int_{\mathbb{R}}\int_{\mathbb{R}^{n-1}}g(y,b) \,\mathrm{d}y\mathrm{d}b,\lambda\right)\\
\implies& \int_{\mathbb{R}^n}h(z)\,\mathrm{d}z\ge M_{\frac{p}{np+1}}\left(\int_{\mathbb{R}^n}f(x)\,\mathrm{d}x,\int_{\mathbb{R}^n}g(y)\,\mathrm{d}y,\lambda\right)
\end{align*}
最后一步用了 Fubini 定理。
\end{proof}


\subsection{最优传输的视角}

由于与 PL 不等式的情形非常类似,我们只再简略地说明一下。

\begin{proof}[BBL 不等式的 OT 证明]
    
如前所述,排除掉 $p=0$ 以及 $F$ 或 $G$ 为 0 的情况,构造概率测度
\[
\mathrm{d}\mu(x):=\frac{f(x)}{F}\mathrm{d}x,\quad \mathrm{d}\nu(x):=\frac{g(x)}{G}\mathrm{d}x
\]

设 $\mu$ 到 $\nu$ 的最优传输为 $T$,根据 Brenier 定理,存在凸函数 $\varphi$ 满足
\[
\begin{cases}
\nabla \varphi &=T\\
\mathrm{det}\big(\nabla^2\varphi\big)&=\frac{f(x)}{F}\frac{G}{g(T(x))}
\end{cases}
\]

再记 $z_\lambda:=(1-\lambda)x+\lambda T(x)$,$\Sigma=\mathrm{diag}(s_1,\ldots,s_n)$ 是由 $\nabla^2\varphi$ 的特征值组成的对角阵,则有
\begin{align*}
\mathrm{d}z_\lambda&=\mathrm{det}\big((1-\lambda)\mathbf{I}_n+\lambda \nabla^2\varphi\big)\,\mathrm{d}x\\
&=\mathrm{det}\big((1-\lambda)\mathbf{I}_n+\lambda \Sigma\big)\,\mathrm{d}x\\
&=\prod_{i=1}^n\big((1-\lambda)+\lambda s_i\big)\,\mathrm{d}x
\end{align*}
取对数,利用 $t\mapsto \log(1-\lambda +\lambda e^t)$ 是凸函数,有
\begin{align*}
\frac1n\sum_{i=1}^n\log\big(1-\lambda+\lambda s_i\big)&=\frac1n\sum_{i=1}^n\log\big(1-\lambda+\lambda e^{\log s_i}\big)\\
&\ge \log\left(1-\lambda+\lambda \exp\left(\frac1n\sum_{i=1}^n \log s_i\right)\right) \tag{Jensen}\\
&=\log\left(1-\lambda+\lambda\left(\prod_{i=1}^n s_i\right)^{1/n}\right)\\
&=\log\Big((1-\lambda)+\lambda\mathrm{det}\big(\nabla^2\varphi\big)^{1/n}\Big)
\end{align*}
这表明
\begin{align*}
\prod_{i=1}^n\big((1-\lambda)+\lambda s_i\big)&\ge \Big((1-\lambda)+\lambda\mathrm{det}\big(\nabla^2\varphi\big)^{1/n}\Big)^n\\
&=M_{1/n}\big(1,\mathrm{det}\big(\nabla^2\varphi\big),\lambda\big)\\
&=M_{1/n}\left(1,\frac{f(x)}{F}\frac{G}{g(T(x))},\lambda\right) \label{eq:2-7}\tag{2-7}
\end{align*}
于是我们直接得出
\begin{align*}
\int_{\mathbb{R}^n}h(x)\,\mathrm{d}x&\ge\int_{\mathbb{R}^n}h\big((1-\lambda)x+\lambda T(x)\big)\,\mathrm{d}z_\lambda \tag{换元}\\
&\ge\int_{\mathbb{R}^n}M_p(f(x),g(T(x)),\lambda)\,\mathrm{d}z_\lambda \tag{由式~\eqref{eq:2-6}}\\
&\ge \int_{\mathbb{R}^n}M_p(f(x),g(T(x)),\lambda)M_{1/n}\left(1,\frac{f(x)}{F}\frac{G}{g(T(x))},\lambda\right)\,\mathrm{d}x \tag{由式~\eqref{eq:2-7}}\\
&\ge\int_{\mathbb{R}^n}M_{q}\left(f(x),f(x)\frac{GF},\lambda\right)\,\mathrm{d}x \tag{由引理~\ref{lem:power_mean_inequality}}\\
&=M_q\left(1,\frac{GF},\lambda\right)\underbrace{\int_{\mathbb{R}^n}f(x)\,\mathrm{d} x}_{F}\\
&=M_q(F,G,\lambda)\\
&=M_q\left(\int_{\mathbb{R}^n}f(x)\,\mathrm{d}x,\int_{\mathbb{R}^n}g(x)\,\mathrm{d}x,\lambda\right)
\end{align*}
其中 $q=\frac{p}{np+1}$。\end{proof}

 

\chapter{Steiner 对称化}\label{chapter:3}

\emph{Steiner 对称化}(Steiner symmetrization)是一个非常强有力的证明凸几何中不等式工具,尤其是,若不等式在单位球情况下达到最大或最小值时,很可能对称化的技巧可以用来解决。

\section{定义}

Steiner 对称化准确来说做的事情是:给定凸体,限制其在某个方向上对称。这具体要如何做到呢?设 $K\subset \mathbb{R}^n$ 是一个凸体,$u\in\mathbb{S}^{n-1}$ 是给定的方向,定义其正交补空间为
\[
u^\perp :=\big\{x\in\mathbb{R}^n:\langle u,x\rangle=0\big\}
\]
这可以视作是空间 $\mathbb{R}^{n-1}$。

将凸体 $K$ 投影到 $u^\perp$ 中,记作 $P_u K$。此时,对任意 $x\in P_u K$,我们都可以沿着 $u$ 方向画一条线 $\{x+tu, t\in\mathbb{R}\}$,这条线会与 $K$ 相交,交集是一条线段,或称``\emph{弦}(chord)"。将这条弦的上端点对应 $t$ 值记作 $f(x)$、下端点对应 $t$ 值记作 $g(x)$\footnote{交集也可能是一个点,此时视作退化的弦,$f(x)=g(x)$。},见图~\ref{fig:steiner_sym_1}。

\begin{figure}[htbp]
\centering
\includegraphics[width=0.7\textwidth]{figures/steiner_symmetration.pdf}
\caption{将凸体 $K$ 投影到平面 $u^\perp$,再将之表示成由 $u$ 方向的弦组成的集合。对 $x\in P_uK$,$f(x)$ 和 $g(x)$ 分别表示弦的上端点和下端点对应的 $t$ 值。}
\label{fig:steiner_sym_1}
\end{figure}

\begin{equation}
K=\Big\{x+tu:g(x)\le t\le f(x):x\in P_uK\Big\} \label{eq:3-1}\tag{3-1}
\end{equation}

$K$ 关于 $u$ 的 \emph{Steiner 对称化}定义为:对每条弦进行平移,使得 $x$ 变成这条弦的中点,即
\begin{equation}
S_uK:=\left\{x+tu:  |t|\le \frac{f(x)-g(x)}{2}, x\in P_uK\right\} \label{eq:3-2}\tag{3-2}
\end{equation}
此时,$S_uK$ 就是关于 $u^\perp$ 对称的。

% \begin{figure}[htbp]
% \centering
% \includegraphics[width=0.7\textwidth]{https://pic1.zhimg.com/v2-fa6e00b10fb874b198635cbdb196d484_720w.jpg?source=d16d100b}
% \caption{Coupier, D., \& Davydov, Y. (2014). Random symmetrizations of convex bodies. Advances in Applied Probability, 46(3), 603-621.}
% \end{figure}
\begin{figure}[!ht] 
    \centering
\includegraphics[width=5cm,height=6.5cm]{figures/steiner_sym_2.pdf} 
% \caption{\label{fig:Steiner} {\small \textit{Steiner symmetrization with direction $u$. The dotted lines represent the sliding of orthogonal segments along $u$.}}}
\caption{沿方向 $u$ 进行 Steiner 对称化。虚线表示沿 $u$ 方向的弦。\footnotemark}\label{fig:steiner_sym_2} 
\end{figure}
\footnotetext{图片来自 Coupier, D., \& Davydov, Y. (2014). \textit{Random symmetrizations of convex bodies}. Advances in Applied Probability, 46(3), 603-621.}
\hfill\break
\begin{remark}
    Steiner 对称化不仅可以用于凸体,也可以用于非凸的集合,不过这个时候直线 $x+tu$ 与 $K$ 的交集可能会很复杂(例如,一条弦可能会被分成好几段),需要将弦长 $f-g$ 定义为 $V_{1}(K\cap \{x+tu,t\in\mathbb{R}\})$。我们文中不涉及这种情况。
\end{remark}

\section{性质}

下面设 $K,L\subset\mathbb{R}^n$ 是凸体、$u\in\mathbb{S}^{n-1}$,我们给出关于 Steiner 对称化的一些重要性质。

\begin{proposition}[保体积]\label{prop:steiner_volume}
经过 Steiner 对称化之后体积不变,即 $V_n(K)=V_n(S_uK)$。
\end{proposition}

\begin{proof}
只需注意到对任意的 $x\in P_uK$,直线 $x+tu$ 来截 $K$ 和 $S_uK$ 的弦长是一样的,都是 $f(x)-g(x)$:
\[
V_n(S_uK)=\int_{P_u}(f-g)\mathrm{d}x=V_n(K)
\]
\end{proof}

\begin{proposition}[保包含关系]\label{prop:steiner_contain}
设 $K\subseteq L$,那么 $S_uK\subseteq S_u L$。
\end{proposition}

\begin{proof}
显然。
\end{proof}

\begin{proposition}[保凸体]\label{prop:steiner_convex}
$S_uK$ 仍是凸体。
\end{proposition}

\begin{proof}
任取两点 $x,y\in S_uK$,考虑二者所在直线截 $K$ 得到的弦,取两弦之并的凸包
\[
T:=\mathrm{Conv}\Big(\big(\{x+tu, t\in\mathbb{R}\}\cap K\big)\cup\big(\{y+tu, t\in\mathbb{R}\}\cap K\big)\Big)
\]
这个 $T$ 是两弦所夹的凸的梯形(trapezoid)。由于 $K$ 自身是凸的,则两弦的凸包仍位于 $K$ 中,则 $T\subset K$;可见梯形 $T$ 在对称化之后,$S_uT$ 也是一个凸的梯形,且包含于 $S_uK$ 内(命题~\ref{prop:steiner_contain}),见图~\ref{fig:steiner_sym_3}。故
\[
\{(1-\lambda)x+\lambda y:0\le\lambda\le 1\}\subseteq S_uT\subseteq S_uK
\]
因此 $S_uK$ 是凸集。易见 $S_uK$ 紧。
\end{proof}
\begin{figure}[htbp]
\centering
\includegraphics[width=0.5\textwidth]{figures/steiner_sym_3.png}
\caption{用梯形的凸性证明 $S_uK$ 是凸集。\footnotemark}
\end{figure}
\label{fig:steiner_sym_3}
\footnotetext{原图来自 Gruber, P. M. (2007). \textit{Convex and discrete geometry}. Berlin, Heidelberg: Springer Berlin Heidelberg. 的 Fig. 9.2 (p.170)。修改了一下符号。}

\begin{proposition}[对 Minkowski 和的单调性]\label{prop:steiner_monotone}
$S_u(K+L)\supseteq S_uK+S_uL$。
\end{proposition}

\begin{proof}
由于 $u$ 的值不重要,我们可以不失一般性地假设 $u=e_n$(第 $n$ 个坐标向量),则 $u^\perp=\mathbb{R}^{n-1}$。前面的~\eqref{eq:3-1}、\eqref{eq:3-2} 又可写作
\begin{equation}
\begin{aligned}
K&=\big\{(x,t):g(x)\le t\le f(x),x\in \mathbb{R}^{n-1}\big\}\\
S_uK&=\left\{(x,t):|t|\le \frac{f(x)-g(x)}2,x\in \mathbb{R}^{n-1}\right\} 
\end{aligned}
\label{eq:3-3}\tag{3-3}
\end{equation}

记 $K_x:=K\cap \{(x,t),t\in\mathbb{R}\}$ 是 $x$ 沿 $u$ 方向直线截 $K$ 的弦,$x\in \mathbb{R}^{n-1}$。
首先,对任意的 $z\in \mathbb{R}^{n-1}$,有
\begin{align*}
(K+L)_z&=(K+L)\cap \{(z,t):t\in \mathbb{R}\}\\
&=\Big\{(x+y,r+s): \text{ 存在 } (x,r)\in K,\,(y,s)\in L,\\
&\qquad \text{ 使得 } x+y=z \Big\}\\
&=\bigcup_{x+y=z}(K_x+L_y)
\end{align*}
于是 $(K+L)_{x+y}\supseteq (K_x+L_y)$,$\forall x,y\in \mathbb{R}^{n-1}$。

\eqref{eq:3-3} 式表明 $(S_uK)_x=\left\{(x,r):  |r|\le V_1(K_x)/2 \right\}$。可以注意到,对任意的 $(x,r)\in S_uK$、$(y,s)\in S_uL$,有
\begin{align*}
|r+s|&\le |r|+|s|\\
&\le\frac{V_1(K_x)}{2}+\frac{V_1(L_y)}{2}\\
&=\frac{V_1(K_x+L_y)}{2}\\
&\le \frac{V_1((K+L)_{x+y})}{2}
\end{align*}
(第三行是因为 $K_x$ 和 $L_y$ 都是线段,故 $V_1(K_x+L_y)=V_1(K_x)+V_1(L_y)$,回忆一下例~\ref{ex:line_segment}。)

于是对任意 $z \in \mathbb{R}^{n-1}$,
\begin{align*}
(S_u(K+L))_z&=\big\{(z,t):|t|\le V_1((K+L)_z)/2\big\}\\
&=\bigcup_{x+y=z}\Big\{(x+y,r+s):|r+s|\le V_1((K+L)_{x+y})/2\Big\}\\
&\supseteq \bigcup_{x+y=z}\Big\{(x,r)+(y,s):(x,r)\in S_uK,(y,s)\in S_uL\Big\}\\
&=\bigcup_{x+y=z}\Big((S_uK)_x+(S_uL)_y\Big)\\
&=(S_u(K)+S_u(L))_z
\end{align*}
这就证明了 $S_u(K+L)\supseteq S_uK+S_uL$。
\end{proof}

可见,Steiner 对称化的另一个作用是,它把复杂的集合问题转化成了比较简单的线段问题。

\begin{proposition}[缩放变换]\label{prop:steiner_scaling}
对任意 $\lambda \ge 0$,$S_u(\lambda K)=\lambda S_uK$。
\end{proposition}

\begin{proof}
显然。
\end{proof}

关于 Steiner 对称化的性质,也可以参考 Gruber, P. M. (2007). \textit{Convex and discrete geometry}. Berlin, Heidelberg: Springer Berlin Heidelberg. 的 Proposition 9.1 (pp 169-171)。

\section{Gross 定理}

\emph{Gross 定理},或者叫 \emph{Sphericity Theorem of Gross},是一个关于 Steiner 对称化的至关重要的定理。通俗地说:\emph{给定某个凸体 $K$,反复对其进行合适的 Steiner 对称化操作,那么在极限情形下,我们总能将 $K$ 变成一个球}。

我们首先进行一些准备。首先定义球的概念:
\begin{definition}[球和椭球]\label{def:ball}
    设 $x\in\mathbb{R}^n$、$r>0$,我们定义``\emph{球}(balls)"或者说``欧氏球"是 $B_n(x,r):=\{y\in\mathbb{R}^n:\|x-y\|_2\le r\}$。单位球记作 $B_n:=B_n(0,1)$,它在所有方向上都对称,因而天然贴合对称化的目标。设 $T$ 是 $\mathbb{R}^n$ 上的非奇异线性变换,则称 $E= TB_n(x,r)$ 为``\emph{椭球}(ellispoid)"。
\end{definition}

其次我们定义如何衡量两个凸体之间的差距。$\mathbb{R}^n$ 中全部凸体构成的集合上有自然拓扑,它可以视作是由 \emph{Hausdorff 度量}(Hausdorff metric)诱导得出。
\begin{definition}[Hausdorff 距离]\label{def:hausdorff_distance}
定义点 $a\in\mathbb{R}^n$ 到集合 $X\subseteq\mathbb{R}^n$ 的距离为 $d(a,X):=\inf\limits_{x\in X}\|x-a\|_2$,那么凸体 $X,Y$ 之间的 \emph{ Hausdorff 距离} 定义为:
\begin{align*}
d_H(X,Y)&:=\max\Big\{\sup_{x\in X}d(x,y),\sup_{y\in Y}d(X,y)\Big\}\\
&=\inf\Big\{\delta\ge 0:X\subseteq Y+\delta B_n,\,Y\subseteq X+\delta B_n\Big\}
\end{align*}
\end{definition}
Hausdorff 距离可以视作是``凸体 $X$、$Y$ 中一点到另一凸体能达到的最大距离"。若记 $h(X,u)$ 代表凸体 $X$ 的支撑函数,则其还可以等价定义为
\[
d_H(X,Y):=\sup_{u\in \mathbb{S}^{n-1}}|h(X,u)-h(Y,u)|
\]

凸体的体积和表面积在 Hausdorff 距离下都是连续的。

最后还需要 Steiner 对称化的一个性质作为引理,
\begin{proposition}[Steiner 对称化的连续性]\label{prop:steiner_continuous}
设有一列 $\mathbb{R}^n$ 中的凸体 $\{C_j\}_{j=1}^\infty$ 满足 $C_j\to C_0$,那么 $S_u C_j\to S_u C_0$。这里的收敛是在 Hausdorff 度量意义下的收敛。
\end{proposition}

\begin{proof}
不失一般性设原点 $0\in C_0$,上述收敛可以表述为对任意 $\epsilon>0$,都存在充分大的 $j$ 使得 $(1-\epsilon)C_0\subseteq C_j\subseteq (1+\epsilon)C_0$,用性质~\ref{prop:steiner_contain} 和 \ref{prop:steiner_scaling},得出 $(1-\epsilon)S_uC_0\subseteq S_uC_j\subseteq (1+\epsilon)S_uC_0$。
\end{proof}

\begin{theorem}[Gross 定理]\label{thm:gross}
对任意给定的凸体 $K\subset\mathbb{R}^n$,记 $\mathscr{H}$ 代表一切通过对 $K$ 进行有限次 Steiner 对称化操作能够得到的集合,则 $\mathscr{H}$ 中存在一列凸体 $\{K_j\}_{j=1}^\infty$ 能收敛到一个同体积的球,即
\[
K_j\to \left(\frac{V_n(K)}{V_n(B_n)}\right)^{1/n}B_n,\qquad j\to\infty
\]
这里的收敛是在 Hausdorff 度量意义下的收敛。
\end{theorem}

\begin{proof}
我们要用到一个引理,称作 \emph{Blaschke 选择定理}:设 $\mathbb{R}^n$ 中有一列凸体一致有界\footnote{均被某个有界集包含。},则其存在子列在 Hausdorff 度量意义下收敛,其极限是一个(可能退化\emph{i.e.}, 内点为空)的凸体。
\footnote{Blaschke 选择定理是一个非常有用的工具。该定理可以由泛函分析中的 Arzelà–Ascoli 定理推出。证明可以参考 Schneider, R. (2014). \textit{Convex bodies: the Brunn-Minkowski theory}. Cambridge University Press. 的定理 1.8.7 (p. 48);抑或是 Gruber, P. M. (2007). \textit{Convex and discrete geometry}. Berlin, Heidelberg: Springer Berlin Heidelberg. 的 Theorem 6.3. (pp. 85-88)。}

根据性质~\ref{prop:steiner_convex},$\mathscr{H}$ 中的集合都是凸体。记 $\rho(K):=\inf\{r:K\subseteq B_n(0,r)\}$(这个函数也是对 Hausdorff 度量连续),$\sigma:=\inf\limits_{X\in\mathscr{H}}\rho(X)$,则存在一列凸体 $\{C_j\}_{j=1}^\infty\subseteq\mathscr{H}$ 使得 $\rho(C_j)\to \sigma$。由于 $K\subseteq \rho(K)B_n$,根据性质 2,$C_j\subseteq \rho(K)B_n$,因此该序列一致有界。我们假设 $\{C_j\}_{j=1}^\infty$ 在 Hausdorff 度量下有极限 $C_0$,如若不然,依 Blaschke 选择定理,可以用其收敛子列替代之;$C_0$ 显然一定是未退化的凸体,因为 $\rho(C_0)=\sigma$。

接下来我们宣称 $C_0=\sigma B_n$。用反证法,如若不然,必有 $C_0$ 真包含于 $B_n$,则存在一个``球冠\footnote{即球与某个半空间的交。}"与 $C_0$ 不交。读者可以想象:一个球冠只覆盖球面的一部分,但是我们总可以进行有限多次镜像操作让这个球冠覆盖整个平面;这些镜像操作是关于某个通过球心的超平面而言的,设这些超平面的法向量是 $v_1,\ldots,v_k$。
考虑在 $v_1,\ldots,v_k$ 方向上调整 $C_0$,从而使它``挤"进 $B_n$ 内 (见图~\ref{fig:gross}),记
\[
D_0:=S_{v_k}S_{v_{k-1}}\cdots S_{v_1}C_0
\]
于是 $\rho (D_0)<\sigma$。

% \begin{figure}[htbp]
% \centering
% \includegraphics[width=0.7\textwidth]{https://picx.zhimg.com/v2-ff65ae904a5e8185901b60b2fcfb7fe1_720w.jpg?source=d16d100b}
% \caption{示意图,用 mathcha 绘制}
% \end{figure}
\begin{figure}
    \centering 
    \includegraphics[width=0.7\textwidth]{figures/gross.pdf}
    \caption{通过 Steiner 对称化操作可以将 $C_0$ 挤进球内。}
    \label{fig:gross}
\end{figure}

这时不难看出矛盾:如果我们对 $\{C_j\}_{j=1}^\infty$ 也都进行同样的挤压操作,记 $D_j:=S_{v_k}S_{v_{k-1}}\cdots S_{v_1}C_j$,显见 $D_j\in\mathscr{H}$,故由 $\sigma$ 的定义知 $\rho(D_j)\ge\sigma$;但是根据性质 6,$D_j\to D_0$,故应有 $\rho (D_0)\ge\sigma$,矛盾!这就证明了 $C_0=\sigma B_n$。
由于 Steiner 对称化保体积(性质~\ref{prop:steiner_volume}),故 $V_n(\sigma B_n)=V_n(K)$,即
\[
\sigma =\left(\frac{V_n(K)}{V_n(B_n)}\right)^{1/n}
\]
证毕。
\end{proof}
\hfill\break
\begin{corollary}\label{col:double_gross}
设 $K,L\subset\mathbb{R}^n$ 是凸体,则存在一列单位方向 $\{u_j\}_{j=1}^\infty$,$u_j\in\mathbb{S}^{n-1},\;\forall j$,通过 Steiner 对称化可以同时让二者收敛到欧氏球,即
\begin{align*}
K_j&:=S_{u_j}\cdots S_{u_1}K\to\left(\frac{V_n(K)}{V_n(B_n)}\right)^{1/n}B_n\\
L_j&:=S_{u_j}\cdots S_{u_1}L\to\left(\frac{V_n(K)}{V_n(B_n)}\right)^{1/n}B_n
\end{align*}
这里的收敛是在 Hausdorff 度量意义下的收敛。
\end{corollary}

\begin{proof}
任取 $\epsilon>0$,则根据定理~\ref{thm:gross},存在充分大的 $j_1\in\mathbb Z_+$ 和单位向量 $u_1,\ldots,u_{j_1}$,使得
\[
S_{u_{j_1}}\cdots S_{u_1}K\subseteq (1+\epsilon)\left(\frac{V_n(K)}{V_n(B_n)}\right)^{1/n}B_n
\]
接着,再次应用定理~\ref{thm:gross},可知存在充分大的 $j_2\ge j_1$ 和单位向量 $u_{j_1+1},\ldots,u_{j_2}$,使得
\[
S_{u_{j_2}}\cdots S_{u_{j_1+1}}(S_{u_{j_1}}\cdots S_{u_1}L)\subseteq (1+\epsilon)\left(\frac{V_n(L)}{V_n(B_n)}\right)^{1/n}B_n
\]
而根据 Steiner 对称化的性质~\ref{prop:steiner_contain},对 $K$ 也仍然满足
\[
S_{u_{j_2}}\cdots S_{u_{j_1+1}}(S_{u_{j_1}}\cdots S_{u_1}K)\subseteq (1+\epsilon)\left(\frac{V_n(K)}{V_n(B_n)}\right)^{1/n}B_n
\]
用 $\epsilon/2$ 代替 $\epsilon$,用得到的新凸体 $S_{u_{j_2}}\cdots S_{u_1}K$、$S_{u_{j_2}}\cdots S_{u_1}L$ 替代 $K$ 和 $L$,可以继续应用上述步骤,考虑极限情形即证。
\end{proof}

\begin{remark}
    上面定理~\ref{thm:gross} 和推论~\ref{col:double_gross} 的证明来自 Gruber, P. M. (2007). \textit{Convex and discrete geometry}. Berlin, Heidelberg: Springer Berlin Heidelberg. 的 Theorem 9.1 和 Corollary 9.1。
\end{remark}

\section{牛刀小试:证明 BM 不等式}

回顾一下前面的性质~\ref{prop:steiner_volume} 和~\ref{prop:steiner_monotone},我们可以得出:对任意 $u\in \mathbb S^{n-1}$,
\begin{equation}
V_n(S_uK+S_uL)\le V_n(S_u(K+L))=V_n(K+L) \label{eq:3-4}\tag{3-4}
\end{equation}
不过 $V_n(S_uK+S_uL)$ 要怎么计算?

这时定理~\ref{thm:gross} 就派上用场了。只要通过一系列 Steiner 对称化操作,让 $K$ 和 $L$ 收敛到一个球,不久好办了吗?根据推论~\ref{col:double_gross} ,存在一列单位向量 $\{u_j\}_{j=1}^\infty$,使得 $K_j:=S_{u_j}\cdots S_{u_1}K$ 和 $L_j:=S_{u_j}\cdots S_{u_1}L$ 都收敛到欧氏球。根据式~\eqref{eq:3-4},
\begin{align*}
V_n(K+L)\ge V_n(K_1+L_1)\ge V_n(K_2+L_2)\ge\cdots\ge V_n(K_j+L_j)\ge\cdots
\end{align*}
只要一取极限
\begin{align*}
V_n(K+L)&\ge\lim_{j\to\infty} V_n(K_j+L_j)\\
&=V_n\Big(\lim_{j\to\infty} (K_j+L_j)\Big) \tag{Hausdorff 度量对体积连续}\\
&=V_n\left(\left(\left(\frac{V_n(K)}{V_n(B_n)}\right)^{1/n}+\left(\frac{V_n(L)}{V_n(B_n)}\right)^{1/n}\right)B_n\right) \tag{定理~\ref{thm:gross}}\\
&=\left(\left(\frac{V_n(K)}{V_n(B_n)}\right)^{1/n}+\left(\frac{V_n(L)}{V_n(B_n)}\right)^{1/n}\right)^{n}V_n(B_n)\\
&=\left(V_n(K)^{1/n}+V_n(L)^{1/n}\right)^{n}
\end{align*}
根据定理~\ref{thm:BM-equivalence},这正是 Brunn-Minkowski 不等式的等价形式。这样我们就利用 Steiner 对称化轻松证明了 BM 不等式。

根据上面的步骤,可以得出利用 Steiner 对称化来解决问题的一般步骤,

\begin{enumerate}
\item Steiner 对称化可以保持体积(或其它的不变量);
\item 用定理~\ref{thm:gross},存在一列 Steiner 对称化操作,可以把问题里的凸体变成欧氏球;
\item 某个凸体的泛函 $F(K)$ 对 Steiner 对称化有单调性,并且该泛函对 Hausdorff 度量连续,取极限就能得到该泛函的不等式。
\end{enumerate}

我们后面将用这个思路来证明等周不等式、Blaschke–Santaló 不等式和仿射等周不等式。

\hfill\break

\begin{remark}
Steiner 对称化实际上是一种所谓``\emph{纤维对称化}(fiber symmetrization)" 的特殊情形。Steiner 对称化是用一维的``弦"来截凸体;而``纤维对称化"中可能是用高维的空间来截凸体,所得的交集称作``纤维",同样能达到对称化的效果。可以参考一些文章,如 Bianchi, G., Gardner, R. J., \& Gronchi, P. (2017). \textit{Symmetrization in geometry}.  Advances in Mathematics, 306, 51-88。课程主讲的叶德平老师和其合作者们有一些工作利用了纤维对称化来证明不等式,亦可参阅之:

\begin{itemize}
\item Haddad, J., Langharst, D., Putterman, E., Roysdon, M., \& Ye, D. (2025). \textit{Affine isoperimetric inequalities for higher-order projection and centroid bodies}: J. Haddad et al. Mathematische Annalen, 1-49.
\item Zhou, X., Ye, D., \& Zhang, Z. (2025). \textit{The $m$-th order Orlicz projection bodies}. arXiv preprint arXiv:2501.07565.
\end{itemize}
\end{remark}


\chapter{混合体积与变分公式}\label{chapter:4}

\section{表面积与混合体积}

回忆一下我们最开始的例~\ref{ex:minkowski_sum}, 以及图~\ref{fig:Minkowski_sum}。
设有边长为 $l$ 的正方形 $K=[-l/2,l/2]\times[-l/2,l/2]$、圆盘 $\epsilon B_2=\{(x,y):x^2+y^2\le\epsilon^2\}$, 显见
\begin{align*}
V_n(K+\epsilon B_2)&=V_n(K)+4l\epsilon+V_n(\epsilon B_2)\\
&=l^2+4l\epsilon+\pi\epsilon^2
\end{align*}
于是存在极限
\begin{align*}
\lim_{\epsilon\to 0}\frac{V_n(K+\epsilon B_2)-V_n(K)}{\epsilon}&=\lim_{\epsilon\to 0^+}\frac{4l\epsilon+\pi\epsilon^2}{\epsilon}\\
&=4l
\end{align*}
正是正方形的周长。

这个现象可以推广:对于一般的 $K\subset\mathbb{R}^n$, 我们也可以把它与 $\epsilon B_n$ 相加,然后把原本的 $K$ 给挖掉,见图~\ref{fig:eps_extension};
剩下的部分可以近似看成一个能够铺展开成``长方形", 其高是 $\epsilon$, 底则可以看作是 $K$ 的``周长 (perimeter)"——只不过这个``周长"是 $n-1$ 维的,所以叫做``表面积"更加合适。
 
\begin{figure}[h]
    \centering
    \includegraphics[width=0.7\textwidth]{figures/surface_1.pdf}
    \caption{$K + \epsilon B_n$ 等同于 $K$ 的边界向外扩展出 $\epsilon$, 因而可用 $\epsilon\to 0^+$ 时的极限来定义 $K$ 的表面积。}
    \label{fig:eps_extension}
\end{figure}

因此可以说
\begin{align*}
& V_n(K+\epsilon B_n)-V_n(K)\approx K\text{ 的表面积}\times \epsilon\\
\implies & \lim_{\epsilon\to 0^+}\frac{V_n(K+\epsilon B_n)-V_n(K)}{\epsilon}=K\text{ 的表面积}
\end{align*}

\begin{definition}[表面积]\label{def:surface_area}
设 $K\subset\mathbb{R}^n$ 是凸体,则定义其\emph{表面积} (surface area) 为
\[
S(K):=\lim_{\epsilon\to 0^+}\frac{V_n(K+\epsilon B_n)-V_n(K)}{\epsilon}
\]
\end{definition}

\begin{corollary}
设 $\lambda>0$, 那么 $S(\lambda K)=\lambda^{n-1}S(K)$。
\end{corollary}

\begin{proof}
\begin{align*}
S(\lambda K) &=\lim_{\epsilon\to0^+}\frac{V_n(\lambda K+\epsilon B_n)-V_n(\lambda K)}{\epsilon}\\
&=\lim_{\epsilon\to0^+}\frac{\lambda^{n-1}\big(V_n(K+\tfrac{\epsilon}{\lambda}B_n)-V_n(K)\big)}{\frac \epsilon \lambda}\\
&=\lambda^{n-1}S(K)
\end{align*}
\end{proof}

\begin{remark}
    可以证明,对于凸体,定义~\ref{def:surface_area} 中的极限总是存在的,并且正是凸体在测度意义上的表面积,即 $\mathcal{H}^{n-1}(\partial K)$。
    其中 $\mathcal{H}^{n-1}$ 代表 $n-1$ 维 Hausdorff 测度,$\partial K$ 代表 $K$ 的边界。
    更一般地,这个定义还可以再推广到一类表面 ``足够良好" 的紧集,有这样的定理\footnotemark
    \begin{theorem*}
        对紧集$K\subset\mathbb R^n$, 若其内部非空且边界是\emph{可求长的} (rectifiable), 那么 
        \[S(K)=\lim_{\epsilon\to 0^+}\frac{V_n(K+\epsilon B_n)-V_n(K)}{\epsilon}=\mathcal{H}^{n-1}(\partial K)\]
    \end{theorem*}
    有时,$S(K)$ 也称 \emph{Minkowski content}。
\end{remark}
\footnotetext{
    该定理的叙述,笔者参考的是 Böröczky, K. J., Figalli, A., \& Ramos, J. P. (2025).\emph{ Isoperimetric inequalities, Brunn-Minkowski theory and Minkowski type Monge-Ampère equations on the sphere} . \url{https://users.renyi.hu/~carlos/Brunn-Minkowski-Book-2024-02-05.pdf} 一书的 Theorem 4.1.4 (p. 102)。
    不过对此叙述最详尽的专著应属 H. Federer (1969). \textit{Geometric measure theory}. Springer-Verlag.
    % 关于 Hausdorff 测度的定义和性质,可以参考例如 Mattila, P. (1995). \textit{Geometry of sets and measures in Euclidean spaces: Fractals and rectifiability}. Cambridge University Press。
}

在表面积的定义中,球 $\epsilon B_n$ 是个很特殊的对象,它是``各项同性"的,能保证 $K+\epsilon B_n$ 总沿 $K$ 的边界扩展出 $\epsilon$。
我们不禁会想:如果把``球"换成一个一般的凸体,会得到什么样的结果?显然,此时边界各处拓展出去的距离就不一样了:
如果 $K,L$ 是两个一般的凸体,那么在 $K$ 的边界上, $K+\epsilon L$ 关于表面法方向 $u\in\mathbb{S}^{n-1}$ 至多能扩展出去 $\epsilon h_L(u)$, 其中 $h_L(u)$ 代表凸体 $L$ 的支撑函数。再把 $K$ 挖掉,剩下的部分展开后就不能用长方形近似了,因为它的``高"在不停变化,是一个锯齿形,如图~\ref{fig:surface_mix}。
 
\begin{figure}[hb]
    \centering
    \includegraphics[width=0.5\textwidth]{figures/surface_2.png}
    \caption{将定义~\ref{def:surface_area} 中的球换成任意的凸体 $L$ , $K$ 边界向外扩展出去的距离就不再总是相同的。
    底部蓝色线代表 $K$ 的表面,灰色代表`` $K+\epsilon B_n$ 和 $K+\epsilon L$ 扩展出去的部分"}
    \label{fig:surface_mix}
\end{figure}

我们可以猜测,当 $\epsilon\to 0$ 时, $(K+\epsilon L)\setminus K$ 的体积应该可以写作积分的形式
\[
V_n(K+\epsilon L)-V_n(K)\approx \int_{\mathbb{S}^{n-1}}\epsilon h_L(u)\mathrm{d}A(u)
\]
即,
\begin{equation}
\lim_{\epsilon \to 0^+}\frac{V_n(K+\epsilon L)-V_n(L)}{\epsilon}=\int_{\mathbb{S}^{n-1}} h_L(u)\mathrm{d}A(u) \label{eq:4-1}\tag{4-1}
\end{equation}
这个 $A(u)$ 代表上图右下锯齿形的``底边", 表述某种和 $K$ 的表面积有关的测度\footnote{因为当 $L=B_n$ 时, $h_L(u)\equiv 1$, 根据定义应该有 $\displaystyle \int_{\mathbb{S}^{n-1}}\mathrm{d}A(u)=S(K)$}。具体如何定义 $A(u)$ 暂且不表,我们后文中会给出上述等式的严格论述。

\begin{definition}[混合体积]\label{def:mix_volume}
设 $K,L\subset\mathbb{R}^n$ 是凸体,定义 $K$ 和 $L$ 的\emph{第一混合体积} (first mixed volume) 为
\[
V_1(K,L)=\frac1n\lim_{\epsilon\to 0^+}\frac{V_n(K+\epsilon L)-V_n(K)}{\epsilon}
\]
\end{definition}

根据定义, $K$ 的表面积可以写作 $S(K)=nV_1(K,B_n)$。

\section{Minkowski 第一不等式}

\begin{theorem}[Minkowski 第一不等式]\label{thm:minkowski}
设 $K,L\subset\mathbb{R}^n$ 是凸体,那么成立不等式
\[
V_1(K,L)\ge V_n(K)^{\frac{n-1}{n}}V_n(L)^{\frac 1n}
\]
不等号成立当且仅当 $K,L$ 位似,即存在 $c>0$ 和 $x_0\in\mathbb{R}^n$ 使得 $L=cK+x_0$。
\end{theorem}

\subsection{特例:等周不等式和 Urysohn 不等式}

在证明该定理之前,我们先看看这个定理能推导出什么结论。

\begin{corollary}[等周不等式]
注意到根据定义~\ref{def:surface_area}, 单位球的表面积和体积满足关系
\begin{align*}
S(B_n)&=nV_1(B_n,B_n)\\
&=\lim_{\epsilon \to 0^+}\frac{V_n((1+\epsilon)B_n)-V_n(B_n)}{\epsilon}\\
&=\left(\lim_{\epsilon \to 0^+}\frac{(1+\epsilon)^n-1}{\epsilon}\right)V_n(B_n)\\
&=nV_n(B_n)
\end{align*}
在定理~\ref{thm:minkowski} 中取 $L=B_n$ 为单位球,那么
\begin{align*}
S(K)=nV_1(K,B_n)&\ge n V_n(K)^{\frac{n-1}n}V_n(B_n)^{\frac 1n}\\
&=\color{DarkBlue}{nV_n(B_n)}\left(\frac{V_n(K)}{V_n(B_n)}\right)^{\frac{n-1}n}\\
&=\color{DarkBlue}{S(B_n)}\left(\frac{V_n(K)}{V_n(B_n)}\right)^{\frac{n-1}n}
\end{align*}
得出
\[
\left(\frac{S(K)}{S(B_n)}\right)^{\frac1{n-1}}\ge\left(\frac{V_n(K)}{V_n(B_n)}\right)^{\frac1{n}}
\]
取等当且仅当 $K$ 是一个球。
这个不等式表明了:表面积一定的凸体,其体积在该凸体为球时取得最大值;体积一定的凸体,表面积在该凸体为球时取得最小值。这便是著名的\emph{等周不等式} (Isoperimetric inequality)。
\end{corollary}

\begin{corollary}[Urysohn 不等式]
如果在定理~\ref{thm:minkowski}  中取 $K=B_n$ 为单位球,又会怎样?有
\begin{align*}
V_1(B_n,L)&\ge V_n(B_n)^{\frac{n-1}{n}}V_n(L)^{\frac1n}\\
&=V_n(B_n)\left(\frac{V_n(L)}{V_n(B_n)}\right)^{\frac 1n}
\end{align*}
即
\[
\frac{V_1(B_n,L)}{V_n(B_n)} \ge \left(\frac{V_n(L)}{V_n(B_n)}\right)^{\frac 1n}
\]
取等当且仅当 $L$ 是一个球。

上面的不等式左侧可以改写作
\begin{align*}
\frac{V_1(B_n,L)}{V_n(B_n)}&=\frac1{\color{DarkBlue}{nV_n(B_n)}}\int_{\mathbb{S}^{n-1}}h_L(u)\mathrm{d}\sigma(u)\\
&=\frac1{\color{DarkBlue}{S(B_n)}}\int_{\mathbb{S}^{n-1}}h_L(u)\mathrm{d}\sigma(u)\\
&:=W(L)
\end{align*}
其中 $\sigma$ 代表球面 $\mathbb{S}^{n-1}$ 上的面积测度。上式所定义的 $W(L)$ 称作凸体 $L$ 的\emph{均宽} (mean-width), 则有
\[
W(L)\ge\left(\frac{V_n(L)}{V_n(B_n)}\right)^{1/n}
\]
这被称作 \emph{Urysohn 不等式}。
\end{corollary}

\begin{remark}
上面两个不等式都可以用第~\ref{chapter:3} 章中介绍的 Steiner 对称化的技巧证明——回忆一下,我们曾在说过,如果不等式在球的情况下达到最大或最小值时,很可能就能利用 Steiner 对称化的技巧解决。
以等周不等式为例,我们只需证明 Steiner 对称化的另一性质:
\begin{proposition}
设 $K\subset\mathbb{R}^n$ 是凸体,则有 $S(S_uK)\le S(K)$。
\end{proposition}
\begin{proof}
用命题~\ref{prop:steiner_monotone} 和~\ref{prop:steiner_scaling}, 对 $\epsilon >0$,
\[
S_u(K+\epsilon B_n)\supseteq S_u K+\epsilon S_u(B_n)=S_u K+\epsilon B_n
\]
因而再结合体积不变 (命题~\ref{prop:steiner_volume}),
\begin{align*}
\frac{V_n(S_uK+\epsilon B_n)-V_n(S_uK)}{\epsilon} \le \frac{V_n(S_u(K+\epsilon B_n))-V_n(S_uK)}{\epsilon}=\frac{V_n(K+\epsilon B_n)-V_n(K)}{\epsilon}
\end{align*}
\end{proof}

该性质告诉我们表面积泛函 $S(K)$ 在 Steiner 对称化操作下是单调递减的,由 \hyperref[thm:gross]{Gross 定理} 以及表面积在 Hausdorff 度量下连续,立即可以证明等周不等式。
至于如何用 Steiner 对称化证明 Urysohn 不等式,我们留给读者 :-)
\end{remark}

\subsection{Minkowski 第一不等式的证明}

接下来我们用 BM 不等式来证明 Minkowski 第一不等式。在此之前,先证明 BM 不等式的一个推论:

\begin{lemma}\label{lem:concave}
设 $K,L\subset\mathbb{R}^n$ 是凸体,那么 $f(t):=V_n\big((1-t)K+tL\big)^{1/n}$,  $0\le t\le 1$ 是凹函数。
\end{lemma}

\begin{proof}
对任意 $0\le t_1,t_2\le 1$ 和 $0\le \lambda\le1$,
\begin{align*}
&f\big((1-\lambda)t_1+\lambda t_2\big)\\
&=V_n\Big((1-(1-\lambda)t_1-\lambda t_2)K+((1-\lambda)t_1+\lambda t_2)L\Big)^{1/n}\\
&=V_n\Big((1-\lambda)\big((1-t_1)K+t_1L\big)+\lambda\big((1-t_2)K+t_2L\big)\Big)^{1/n}\tag{定理~\ref{thm:BM-convex}}\\
&\ge (1-\lambda)V_n\big((1-t_1)K+t_1L\big)^{1/n}+\lambda V_n\big((1-t_2)K+t_2L\big)^{1/n}\\
&=(1-\lambda)f(t_1)+\lambda f(t_2)
\end{align*}
其中第三行用了 BM 不等式。
\end{proof}

\begin{remark}
事实上, $V_n\big((1-t)K+tL\big)$ 可以写作关于 $t$ 的 $n$ 次多项式。
更一般地,对于凸体 $K_1,\ldots,K_m\subset\mathbb{R}^n$ 和实数 $\lambda_1,\ldots,\lambda_m\ge 0$,  $V_n\left(\lambda_1K_1+\cdots+\lambda_mK_m\right)$ 可以写作 $\lambda_1,\ldots,\lambda_m$ 的 $n$ 次齐次多项式。
这个结论一般会写成
\[V_n(K+\lambda L)=\sum_{i=0}^n c_i \lambda^i\]
其中 $c_i$ 是某些与 $K,L$ 有关的非负常数。特别地,当 $L=B_n$ 时, 该式称作 \emph{Steiner 公式}。
详细的理论可以参考 Gruber, P. M. (2007). \textit{Convex and discrete geometry}. Berlin, Heidelberg: Springer Berlin Heidelberg 的 Theorem 6.5 等。
\end{remark}

\begin{proof}[Minkowski 第一不等式的证明]
第一混合体积 $V_1(K,L)$ 的定义式不太好用,为此,我们考虑改写其形式
\begin{align*}
V_1(K,L)&=\frac1n\lim_{\epsilon\to 0^+}\frac{V_n(K+\epsilon L)-V_n(K)}{\epsilon}\\
&=\frac1n\lim_{t\to 0^+}\frac{V_n\left(K+\frac{t}{1-t} L\right)-V_n(K)}{\frac{t}{1-t}} \tag{令 $\epsilon=\frac t{1-t}$}\\
&=\frac1n\lim_{t\to 0^+}\frac{1-t}{t}\bigg(\bigg(\frac 1{1-t}\bigg)^nV_n\big((1-t)K+tL\big)-V_n(K)\bigg)\\
&=\frac1n\lim_{t\to 0^+}\underbrace{\frac1{(1-t)^{n-1}}}_{\to 1}\frac1{t}\bigg(V_n\big((1-t)K+tL\big)-(1-t)^nV_n(K)\bigg)\\
&=\frac1n\lim_{t\to 0^+}\frac{V_n\big((1-t)K+tL\big)-V_n(K)}{t}+\frac1n V_n(K)\underbrace{\lim_{t\to 0^+}\frac{1-(1-t)^n}{t}}_{=n}\\
&=\frac1n\lim_{t\to 0^+}\frac{V_n\big((1-t)K+tL\big)-V_n(K)}{t}+V_n(K)
\end{align*}
因而,若记 $f(t):= V_n\big((1-t)K+tL\big) ^{1/n}$,  $0\le t\le1$, 则上面的结果可写作
\begin{align*}
V_1(K,L)-V_n(K)&=\frac1n\lim_{t\to 0^+}\frac{V_n\big((1-t)K+tL\big)-V_n(K)}{t}\\
&=\frac1n \frac{\mathrm{d}f^n}{\mathrm{d}t}(0^+)\\
&=f^{n-1}(0)f'(0^+)
\end{align*}

我们知道 $f^{n-1}(0)=V_n(K)^{\frac{n-1}{n}}$, 于是,要证 Minkowski 第一不等式即 $V_1(K,L)\ge V_n(K)^{\frac{n-1}{n}}V_n(L)^{\frac 1n}$ 成立,只需证明
\begin{align*}
f'(0^+)&=\frac{V_1(K,L)-V_n(K)}{V_n(K)^{\frac{n-1}n}}\\
&\ge \frac{V_n(K)^{\frac{n-1}n} V(L)^{\frac1n}-V_n(K)}{V_n(K)^{\frac{n-1}n}}\\
&=V_n(L)^{\frac1n}-V_n(K)^{\frac1n}\\
&=f(1)-f(0)
\end{align*}

而根据引理~\ref{lem:concave},
\begin{align*}
& f(t)=f((1-t)\cdot 0+t\cdot 1)\ge (1-t)f(0)+t f(1),\;\forall\, 0<t<1\\
\implies&\frac{f(t)-f(0)}{t}\ge f(1)-f(0),\;\forall\, 0<t<1\\
\overset{t\to0^+}{\implies}& f'(0^+) \ge f(1)-f(0)
\end{align*}
这就证明了 Minkowski 第一不等式。

下证取等条件。一方面 (充分性), 若存在 $c>0$ 和 $x_0\in\mathbb{R}^n$ 使得 $L=cK+x_0$, 则
\begin{align*}
& V_n(K+\epsilon L)=V_n(K+\epsilon cK+\epsilon x_0)=(1+\epsilon c)^nV_n(K)\\
\implies & V_1(K,L)=\frac1n\lim_{\epsilon\to 0^+}\frac{ V_n(K+\epsilon L)-V_n(K)}{\epsilon}\\
& \quad =\frac1n V_n(K)\lim_{\epsilon\to 0^+}\frac{(1+\epsilon c)^n-1}{\epsilon}\\
& \quad =cV_n(K)\\
& \quad =V_n(K)^{\frac{n-1}n}(c^n V_n(K))^{\frac1n}\\
& \quad =V_n(K)^{\frac{n-1}n}V_n(L)^{\frac1n}
\end{align*}
即 Minkowski 第一不等式取等。

反过来 (必要性), 若取等,则 $f'(0^+)=f(1)-f(0)$ 需成立,由引理~\ref{lem:concave},  $f$ 必须是一个线性函数,即 $f(t)=(1-t)f(0)+tf(1),\;\forall\, 0\le t\le 1$, 这正是 BM 不等式取等。故 Minkowski 第一不等式取等蕴含 BM 不等式的取等条件,即 $K,L$ 位似。
\end{proof}

\section{混合体积的变分公式}

接下来我们回到~\eqref{eq:4-1}式,来解决这样一个问题: $V_1(K,L)$ 的积分形式应该是什么?或者说,要使~\eqref{eq:4-1}式: $\displaystyle V_1(K,L)=\int_{\mathbb{S}^{n-1}} h_L(u)\mathrm{d}A(u) $ 成立,这个 $A(u)$ 到底应该是什么东西?

\subsection{准备工作}

为此,我们得先补充一些定义,

\begin{definition}[支撑函数和超平面]
设 $K\subset\mathbb{R}^n$ 是凸体, $u\in\mathbb{S}^{n-1}$, 则称 $h_K(u):=\sup\limits_{x\in K} \langle x,u\rangle$ 是 $K$ 的\emph{支撑函数} (support function),  $H_K(u):=\{x\in\mathbb{R}^n:\langle x,u\rangle=h_K(u)\}$ 是 $K$ 的一个\emph{支撑超平面} (support plane)。
\end{definition}

\begin{definition}[法向量]
设 $K\subset\mathbb{R}^n$ 是凸体, $\partial K$ 是其边界,若对 $x\in\partial K$, 方向向量 $u\in\mathbb{S}^{n-1}$ 满足 $\langle x,u\rangle=h_K(u)$, 即 $u$ 是过 $x$ 点的支撑超平面的法向量,则称 $u$ 是曲面 $\partial K$ 在 $x$ 处的\emph{法向量} (normal vector)。
\end{definition}

我们上述定义的法向量不一定唯一:在某个 $x\in\partial K$ 处可能存在多个法向量。如下左图所示,在左上方的尖点处会存在多个法向量 (事实上这些法向量构成一个闭锥,称作\emph{法锥} normal cone), 见~\ref{fig:normal_radial} 左图。

同时,给定法向量 $u$, 其对应的 $x$ 也可能不唯一 (例如可能 $K$ 的一个``面"都有相同的法向量) 。

\begin{figure}[h]
    \centering
    \includegraphics[width=0.8\textwidth]{figures/normal_radial.pdf}
    \caption{左图:凸体 $K$ 上某些点可能存在多个法向量 (如尖点处);给定法向量 $u$, 其对应的边界点也可能不唯一 (如平面部分)。
    右图:从原点出发,沿方向 $v$ 走到边界 $\partial K$ 上的点 $r_K(v)$, 走过距离为 $\rho_K(v)$。}
\label{fig:normal_radial}
\end{figure}

\begin{definition}[Gauss 映射和其逆]
设 $x\in\partial K$, 定义集合 $\nu_K(x)$: $u\in\nu_K(x)$ 当且仅当 $\langle x,u\rangle =h_K(u)$, 即 $u$ 是 $x$ 处的法向量;称 $\nu_K$ 是 \emph{Gauss 映射}, 它将 $\partial K$ 上的点映射到 $\mathbb{S}^{n-1}$ 上的点。
反过来,对于给定的 $u\in\mathbb{S}^{n-1}$, 定义集合 $\nu_K^{-1}(u)$: $x\in \nu_K^{-1}(u)$ 当且仅当 $\langle x,u\rangle =h_K(u)$, 称其为 \emph{Gauss 逆映射}, 它将 $\mathbb{S}^{n-1}$ 上的点映射到  $\partial K$。
\end{definition}

这里我们把 $\nu_K$ 和 $\nu_K^{-1}$ 都定义为集合,原因正如我们刚刚所说,在某些点处它们可能不唯一;不过实际上,这种点都是零测的 (相对于后面讨论的球面/表面积测度)。换句话说,这两个映射是``几乎处处"单值的,我们无需担心多值造成的影响。

\begin{remark}
简而言之,对于任意一个凸体 $K$, 我们总能将其边界 $\partial K$ 划分成有限个部分,使得每个部分 (在合适坐标系下) 都是一个凸函数。\emph{Aleksandrov 定理}告诉我们凸函数几乎处处可微,因而我们可以说, $\partial K$ 上几乎处处每一个点存在唯一的支撑超平面以及法向量 (这种点又叫做正则点), 这说明我们可以把 $\nu_K$ 视作是几乎处处单值的。

反过来也一样, $h_K$ 是一个凸函数,因而几乎处处可微。凸几何中有这样的结论: $h_K(u)$ 在 $u$ 处可微当且仅当支撑超平面 $H_K(u)$ 与 $K$ 的交集 (称作支撑集) 仅有一个点 , 即 $H_K(u)\cap K=\{x\}$ \footnotemark, 这说明我们可以把 $\nu_K^{-1}$ 视作是几乎处处单值的。
% 由于那些性质不好的、造成映射多值的点也不会产生影响,我们简单起见,
后文中均将 $\nu_K(u)$ 和 $\nu^{-1}_K(x)$ 视作是唯一定义的。
\end{remark}
\footnotetext{见 Schneider, R. (2013). \textit{Convex bodies: the Brunn–Minkowski theory} (Vol. 151). Cambridge university press. 的 Corollary 1.7.3, p.47}

\begin{definition}[半径函数]\label{def:radial_func}
设 $0\in K$, 定义 $K$ 的\emph{半径函数} (radial function) $\rho_K:\mathbb{S}^{n-1}\to [0,\infty)$ 为从原点出发,在 $K$ 内沿 $v$ 方向的最大距离,
\[
\rho_K(v):=\sup\{r>0:rv\in K\}
\]
定义 $r_K(v):=\rho_K(v)v\in\partial K$, 这是一个 $\mathbb{S}^{n-1}\to\partial K $ 的函数。
\end{definition}

上述定义亦见于~\ref{fig:normal_radial} 右图。本章接下来均假设原点 $0\in K$。

\begin{definition}[半径 Gauss 映射及其逆]\label{def:radial_gauss_map}
记  $\alpha_K:=\nu_k\circ r_K$ 为\emph{半径 Gauss 映射}。

记 $r_K^{-1}:\partial K\to\mathbb{S}^{n-1}$ 为
\[
r_K^{-1}(x):=\frac{x}{\|x\|},\quad x\ne 0
\]

\emph{半径 Gauss 逆映射} 定义为 $\alpha_K^{-1}:=(\nu_k\circ r_K)^{-1}=r_K^{-1}\circ \nu_K^{-1}$。
\end{definition}

通俗地说,从原点出发,向 $u$ 方向一直走到 $K$ 的边界,在边界上这个点处的法向量就是 $\alpha_K(u)$。反过来,如果 $u$ 是边界上某个点的法向量,那么将这个点归一化,所得方向向量就是 $\alpha_K^{-1}(u)$。 $\alpha_K$ 和 $\alpha_K^{-1}$ 都是 $\mathbb{S}^{n-1}\to\mathbb{S}^{n-1}$ 的映射。

如前对 Gauss 映射和其逆的讨论,我们可以将 $\alpha_K(u)$ 和 $\alpha_K^{-1}(u)$ 视作是几乎处处单值的。

\subsection{关于凸体的体积}

首先,可以将凸体 $K\subset\mathbb{R}^n$ 的体积写作积分的形式,并应用极坐标变换,有
\begin{align*}
V_n(K)&=\int_K\mathrm{d}x=\int_{\mathbb{S}^{n-1}}\int_0^{\rho_K(v)}r^{n-1}\mathrm{d}r\mathrm{d}v\\
&=\frac1n\int_{\mathbb{S}^{n-1}}(\rho_K(v))^n\mathrm{d}v \label{eq:4-2}\tag{4-2}
\end{align*}

另一方面,记 $\displaystyle\mathrm{div}(x)=\sum_{i=1}^n\frac{\partial x_i}{\partial x_i}=n$ 代表 $x$ 的散度, $\mathcal H^{n-1}(x)$ 是 $\partial K$ 上的 $n-1$ 维 Hausdorff 测度 (边界上的表面积微元) , 我们还能将体积写成
\begin{align*}
V_n(K)&=\int_K\mathrm{d}x=\frac1n\int_Kn\,\mathrm{d}x\\
&=\frac1n\int_K\mathrm{div}(x)\,\mathrm{d}x\\
&=\frac1n\int_{\partial K}\langle x,\nu_K(x)\rangle\,\mathrm{d}\mathcal H^{n-1}(x) \tag{散度定理}\\
&=\frac1n\int_{\mathbb{S}^{n-1}}h_K(u)\mathrm{d}S(K,u) \label{eq:4-3}\tag{4-3}
\end{align*}

这里最后一行用 $u=\nu_K(x)$ 换元后,Gauss 映射把 $\mathcal{H}^{n-1}(x)$ 推送到了一个新的测度 $\mathrm{d}S(K,u)$, 可以算出来是
\[
S(K,U):=\int_{\{x\in\partial K:x\in\nu^{-1}(u),u\in U\}}\mathrm{d}\mathcal H^{n-1}=\mathcal H^{n-1}(\nu_K^{-1}(U))
\]
其中 $U$ 是球面 $\mathbb{S}^{n-1}$ 中的 Borel 集。

我们称 $\mathrm{d}S(K,u)$ 是 $K$ 的\emph{表面积测度}, 由上可知,它由Gauss映射和 $\partial K$ 上的 $n-1$ 维 Hausdorff 测度诱导而来。什么意思呢?请看图~\ref{fig:gauss_map} 中的例子。

\begin{figure}[h]
\centering
\includegraphics[width=0.8\textwidth]{figures/gauss_map.pdf}
\caption{Gauss 逆映射 $\nu_K^{-1}$ 的示意图。对于方向 $u_1$, 我们发现 $K$ 中有一个``面"即蓝色线段,上面每一个点的法向量都是 $u_1$, 于是这条蓝色线段就是 $\nu^{-1}_K(u_1)$, 其长度就是 $S(K,u_1)$;对于圆周上的一段橙色弧 $U$, 对应到 $K$ 的边界上,也能找到一段橙色部分,其中点的法向量在 $U$ 中,这段橙色部分的长度就是 $S(K,U)$。}
\label{fig:gauss_map}
\end{figure}

式~\eqref{eq:4-2}和式~\eqref{eq:4-3}为我们带来了两个测度:

\begin{itemize} 
\item (用半径函数) $\frac1n\rho_K^n (v)\mathrm{d}v$, 其中 $\mathrm{d}v$ 代表球面测度;
\item (用支撑函数) $\frac1n h_K(u)\mathrm{d}S(K,u)$。
\end{itemize}


\begin{figure} 
\centering
\includegraphics[width=0.7\textwidth]{figures/measure_comp.pdf}
\caption{两个不同测度的比较:$\frac1n\rho_K^n (v)\mathrm{d}v$ 计算扇形面积, $\frac1n h_K(u)\mathrm{d}S(K,u)$ 计算三角形面积。}
\label{fig:measure_comp}
\end{figure}
% \hfill\break

举个二维的例子,见图~\ref{fig:measure_comp}, 可以这样去理解:设 $u$ 是 $\partial K$ 上某一点的法向量,此点附近面积微元与原点所围成的部分能近似为一个扇形 (高维情况下则是``锥") , 上面第一个测度相当于用扇形面积来近似这部分面积;同时这部分也能视作一个三角形,第二个测度相当于用`` $\frac1n$×底面积×高"来近似这部分面积—— $h_K(u)$ 就是高,而 $S(K,u)$ 就是那个底面积。
一个重要的观察是,这两个测度可以通过半径 Gauss 映射和其逆映射联系起来:

\begin{corollary}\label{col:map_diagram}
若作变换 $u=\alpha_K(v)$, 则可将测度 $\frac1n\rho_K^n \mathrm{d}v$ 变为 $\frac1n h_K\mathrm{d}S(K,u)$;反之若作变换 $v=\alpha_K^{-1}(u)$, 则可将测度 $\frac1n h_K\mathrm{d}S(K,u)$ 变回 $\frac1n\rho_K^n \mathrm{d}v$。 
\end{corollary}
\begin{figure}[h]
\centering
\includegraphics[width=0.6\textwidth]{./figures/map_diagram.pdf}
\caption{推论~\ref{col:map_diagram} 的示意图: $\nu_K$、$r_K$ 和 $\alpha_K$ 三者的关系。}
\end{figure}   
后面定理~\ref{thm:var_wulff} 的证明中我们会用到这个代换。

\subsection{推论:Minkowski 第一不等式等价于 BM 不等式}

接下来我们终于能够回答一开始的问题了,答案是这样的

\begin{theorem}[混合体积的变分公式]\label{thm:var}
设 $K,L\subset\mathbb{R}^n$ 是凸体,有
\begin{equation}
V_1(K,L)=\frac1n\int_{\mathbb{S}^{n-1}}h_L(u)\mathrm{d}S(K,u) \label{eq:4-4}\tag{4-4}
\end{equation}
\end{theorem}

通俗地说,这个式子说的是:``第一混合体积" $V_1(K,L)$ 可以视作是用 $K$ 做``底面积", 用 $L$ 做``高"算出来的体积。值得注意的是,测度 $\mathrm{d}S(K,u)$ 是仅依赖于 $K$ 的,而 $L$ 则是任意的。

证明放在最后,我们先看它的一个推论。

\begin{corollary}\label{cor:minkowski-bm}
Minkowski 第一不等式等价于 Brunn-Minkowski 不等式。
\end{corollary}

\begin{proof}
后者推前者已证 (定理~\ref{thm:minkowski} ), 现用 Minkowski 第一不等式推 BM 不等式,首先,我们知道集合的 Minkowski 和总对应着支撑函数的和:
\begin{align*}
h_K(u)+h_L(u)&=\sup_{x\in K}\langle x,u\rangle+\sup_{y\in L}\langle y,u\rangle\\
&=\sup\big\{\langle x+y,u\rangle:x\in K,y\in L\}\\
&=\sup\big\{\langle z,u\rangle:z\in K+L\}\\
&=h_{K+L}(u)
\end{align*}
于是由式~\eqref{eq:4-4}有
\begin{align*}
&V_1(K+L,K)+V_1(K+L,L)\\
=&\frac1n\int_{\mathbb S^{n-1}}h_K(u)\mathrm{d}S(K+L,u)+\frac1n\int_{\mathbb S^{n-1}}h_L(u)\mathrm{d}S(K+L,u)\\
=&\frac1n\int_{\mathbb S^{n-1}}h_{K+L}(u)\mathrm{d}S(K+L,u)\\
=&V_n(K+L)
\end{align*}
而根据 Minkowski 第一不等式有
\begin{align*}
V_1(K+L,K)&\ge V_n(K+L)^{\frac{n-1}{n}}V_n(K)\\
V_1(K+L,L)&\ge V_n(K+L)^{\frac{n-1}{n}}V_n(L) \tag{$*$}
\end{align*}
两式相加,立即得到
\[
V_n(K+L)=V_1(K+L,K)+V_1(K+L,L)\ge V_n(K+L)^{\frac{n-1}{n}}\big(V_n(K)^{\frac1n}+V_n(L)^{\frac1n}\big)
\]
正是 BM 不等式的等价形式 (定理~\ref{thm:BM-equivalence})。等号成立当且仅当 $(*)$ 两式都取等,即 $K$ 位似于 $K+L$ 且 $L$ 位似于 $K+L$, 亦即, $K$ 与 $L$ 位似。
\end{proof}

\subsection{Wulff 形}

相比于~\eqref{eq:4-4}, 我们将证明一个更一般的结论。不过在这之前,还要介绍一个概念。

回忆一下,我们在第一篇文章中就曾介绍过:可以用半空间的交来构造凸集
\[
K=\bigcap_{u\in\mathbb{S}^{n-1}}\{x\in\mathbb{R}^n:\langle x,u\rangle \le h_K(u)\}
\]

这种思想可以推广到函数$f:\mathbb{S}^{n-1}\to (0,\infty)$, 只需将上式中的 $h_K(u)$ 替换成 $f(u)$, 即可得到它的 \emph{Wulff 形} (Wulff shape)
\begin{definition}[Wulff 形]\label{def:wulff}
对连续正值函数 $f:\mathbb{S}^{n-1}\to (0,\infty)$ , 定义其 \emph{Wulff 形} 为
\[
[f]:=\bigcap_{u\in\mathbb{S}^{n-1}}\{x\in\mathbb{R}^n:\langle x,u\rangle \le f(u)\}
\]
    
\end{definition}

图~\ref{fig:wulff} 是一个 Wulff 形的形象例子。
 
\begin{figure}
    \centering 
    \includegraphics[width=1.0\textwidth]{./figures/wulff/Wulff.png}
    \caption{$f(\theta) = 1+0.45 \cos( 3\theta) + 0.15 \cos\theta$ 的 Wulff 形 (左) 和图像 (右), 五张图分别对应 $\theta=0,\frac{\pi}{3},\frac{2\pi}{3}, \pi, \frac{4\pi}{3}, \frac{5\pi}{3}$ 的情形。}
    \label{fig:wulff}
\end{figure}


我们总结一下 Wulff 形的性质:

\begin{proposition}\label{prop:wulff_convex}
$[f]$ 是一个凸体。
\end{proposition}

\begin{proof}
由于 $[f]$ 可以表示为下半空间的交,因而是一个闭凸集;进一步,由于 $f$ 连续,因而其在紧集上有界,故 $[f]$ 是凸体。
\end{proof}

\begin{proposition}\label{prop:wulff_origin}
由于 $f>0$, 可以注意到 $0\in\mathrm{int}[f]$ 也是成立的。
\end{proposition}

\begin{proposition}\label{prop:wulff_support}
记 Wulff 形 $[f]$ 的支撑函数为 $h_{[f]}$, 那么在 $\mathbb{S}^{n-1}$ 上有 $h_{[f]}\le f$。
\end{proposition}
\begin{remark}
    这个其实从图~\ref{fig:wulff} 的例子里很容易看出来,由于 $[f]$ 是用下半空间 $H_{u,f(u)}^-:=\{x:\langle x,u\rangle \le f(u)\}$ 的交定义的,显然生成的凸体之支撑函数必然不超过 $f$。同时有这样的结论: $[f]$ 是最大的凸体使得其支撑函数满足 $h_{[f]}\le f$。
\end{remark}

\begin{proposition}\label{prop:wulff_measure}
对 Wulff 形的表面测度 $\mathrm{d}S([f],u)$,  $h_{[f]}=f$ 几乎处处成立。
\end{proposition}

\begin{proof}
首先,对于任意 $x\in\partial [f]$, 它必须位于至少一个半空间 $H_{u,f(u)}^-$ 的边界上 (否则的话 $x$ 必为内点,不会在 $\partial [f]$) , 即,至少存在一个 $u\in\mathbb{S}^{n-1}$ 使得 $\langle x,u\rangle =f(u)$;如我们之前所说,在凸体的边界上,相对于其表面积测度,Gauss 映射几乎处处为单值,于是 $\partial [f]$ 上几乎所有的点都有唯一的法向量——对这些点有 $\langle x,u\rangle =h_{[f]}(u)$。这样一来,对几乎所有的 $u$,  $h_{[f]}(u)=\langle x,u\rangle=f(u)$。
\end{proof}

\begin{proposition}\label{prop:wulff_eq}
若 $K$ 是凸体且 $0\in\mathrm{int}(K)$, 则 $[h_K]=K$。
\end{proposition}

最后,注意到
\begin{align*}
[f]&=\bigcap_{u\in\mathbb{S}^{n-1}}\big\{y\in\mathbb{R}^n:\langle y,u\rangle\le f(u)\big\}\\
&=\bigcap_{u\in\mathbb{S}^{n-1}}\left\{y\in\mathbb{R}^n:\left\langle y,\frac{u}{f(u)}\right\rangle\le 1\right\}\\
&=\left\{y\in\mathbb{R}^n:\text{ 对任意 }u\in\mathbb{S}^{n-1}\text{ 有 }\langle y,z\rangle \le 1,z:=\frac{u}{f(u)}\right\}
\end{align*}
如果我们定义函数 $g:\mathbb{S}^{n-1}\to (0,\infty)$ 的\emph{凸包} (convex hull) 为
\[
\langle g\rangle:=\mathrm{Conv}\big(\{g(u)u:u\in\mathbb{S}^{n-1}\}\big)
\]
那么有
\begin{align*}
[f]&=\left\{y\in\mathbb{R}^n:\langle y,z\rangle\le 1,\forall z\in\left\langle\frac1f\right\rangle\right\}\\
&=\left\langle\frac1f\right\rangle^\circ
\end{align*}
即: 

\begin{proposition}\label{prop:wulff_polar} 
$f$ 的 Wulff 形是 $\langle 1/f\rangle$ 的极集。
\end{proposition}

后文中性质~\ref{prop:wulff_polar} 需要配合如下两个引理使用,

\begin{lemma}[双极定理 (Bipolar theorem) 的一个特例]\label{lem:bipolar}
设 $K\subset\mathbb{R}^n$ 是凸体, $0\in\mathrm{int}(K)$, 那么 $K^{\circ\circ}=K$。(其证明已在第~\ref{chapter:1}章中提及过。)
\end{lemma}

\begin{lemma}\label{lem:support_radial}
设 $K\subset\mathbb{R}^n$ 是凸体,$0\in\mathrm{int}(K)$, 那么 $h_{K^\circ}=\frac1{\rho_K}$、 $\rho_{K^\circ}=\frac1{h_K}$。
\end{lemma}

\begin{proof}
只需用定义,
\begin{align*}
\rho_{K^\circ}(v)&=\inf\{r>0:rv\in K^\circ\}\\
&= \inf\{r>0:\langle x,rv\rangle\le 1,\forall x\in K\}\\
&=1/\big(\sup\{r^{-1}>0:\langle x,v\rangle\le r^{-1},\forall x\in K\}\big)\\
&=1/\Big(\sup_{x\in K}\langle x,v\rangle\Big)\\
&=1/h_K(x)
\end{align*}
根据引理~\ref{lem:bipolar}, 我们可以用 $K^\circ$ 取代上式中的 $K$, 得到 $\rho_K=\rho_{K^{\circ\circ}}=1/h_{K^\circ}$。
\end{proof}

\subsection{变分公式的证明}
我们本节证明一个更加一般的结论
\begin{theorem}\label{thm:var_wulff}
设 $f:\mathbb{S}^{n-1}\to(0,\infty)$ 是一个正的连续函数, $K\subset\mathbb{R}^n$ 为凸体,且 $0\in\mathrm{int}(K)$。记
\[
f_\epsilon:=h_K+\epsilon f,\quad \epsilon >0
\]
那么
\begin{equation}
\frac1n\lim_{\epsilon\to 0^+}\frac{V_n([f_\epsilon])-V_n(K)}{\epsilon}=\frac1n\int_{\mathbb{S}^{n-1}}f(u)\mathrm{d}S(K,u) \label{eq:4-5}\tag{4-5}
\end{equation}
\end{theorem}

\begin{remark}
\begin{itemize}
\item 定理~\ref{thm:var} 中,不失一般性设 $0\in\mathrm{int}(K)$ 且 $0\in\mathrm{int}(L)$, 则变分公式~\eqref{eq:4-4} 是定理~\ref{thm:var_wulff} 式~\eqref{eq:4-5}在取 $f=h_L$ 时的特例。
\item 由于 $0\in\mathrm{int}K$,  $h_K(u)>0$ 总成立,我们实际上能把 $f$ 的值域放宽至 $\mathbb{R}$——因为总能选取 $\epsilon>0$ 足够小使得 $f_\epsilon>0$ 成立;实际上可以证明定理的结论此时也是成立的。
\end{itemize}
\end{remark}

\begin{proof}
为了叙述清晰,我们将证明分为 6 步。
\begin{enumerate}[label=\textbf{第\chinese*步}:, leftmargin=4em]
\item 由于 $[f_\epsilon]$ 是凸体且 $0\in\mathrm{int}[f_\epsilon]$, 应有 $\rho_{[f_\epsilon]}(v)>0$ 对一切 $v\in\mathbb{S}^{n-1}$ 都成立, 且有
\[
V_n([f_\epsilon])=\frac1n\int_{\mathbb{S}^{n-1}}\rho^n_{[f_\epsilon]}(v)\mathrm{d}v
\]

另一方面,由 Wulff 形性质~\ref{prop:wulff_polar} 和引理~\ref{lem:support_radial}, 有 $\rho_{[f_\epsilon]}=\rho_{\langle 1/f_\epsilon\rangle^\circ}=1/h_{\langle 1/f_\epsilon\rangle}$, 带入上式,得到
\[
V_n([f_\epsilon])=\frac1n\int_{\mathbb{S}^{n-1}}\frac1{h^{n}_{\langle 1/f_\epsilon\rangle} (u)}\mathrm{d}u
\]

欸, $1/f_\epsilon$ 写起来太不方便了,我们不妨把它记作 $g_\epsilon:=1/f_\epsilon$, 这样一来就有  $\displaystyle V_n([f_\epsilon])=\frac1n\int_{\mathbb{S}^{n-1}}\frac1{h^n_{\langle g_\epsilon\rangle}(u)}\mathrm{d}u$,  且
\begin{align*}
&g_0=\frac1{f_0}=\frac1{h_K}=\rho_{K^\circ}\\
\implies \; &\langle g_0\rangle=K^\circ\\
\implies \;& V_n(K)=\int_{\mathbb{S}^{n-1}}\rho^n_K(v)\mathrm{d}v=\int_{\mathbb{S}^{n-1}}\frac1{h_{\langle g_0\rangle}^n(u)}\mathrm{d}u
\end{align*}

\eqref{eq:4-5} 中左侧的极限就可以改写成
\begin{equation}
\lim_{\epsilon\to0^+}\frac{V_n([f_\epsilon])-V_n(K)}{\epsilon}=\lim_{\epsilon\to 0^+}\frac1n\int_{\mathbb{S}^{n-1}}\frac{h^{-n}_{\langle g_\epsilon\rangle}(u)-h^{-n}_{\langle g_0\rangle}(u)}{\epsilon}\mathrm{d}u \label{eq:4-6}\tag{4-6}
\end{equation}

\item 选定 $u\in\mathbb{S}^{n-1}$, 我们把目光放在  $\frac{h^{-n}_{\langle g_\epsilon\rangle}(u)-h^{-n}_{\langle g_0\rangle}(u)}{\epsilon}$ 上面,先来看一下它在 $\epsilon\to 0^+$ 的极限是什么,后面可以想办法用控制收敛定理交换~\eqref{eq:4-6}中极限与积分。

由于 $g_\epsilon$ 一致收敛于 $g_0$,  $\epsilon\to 0^+$, 所以在 Hausdorff 度量下 $\langle g_\epsilon\rangle\to \langle g_0\rangle$, 可得 $h_{\langle g_\epsilon\rangle}\to h_{\langle g_0\rangle}$ 一致收敛,有
\begin{align*}
&\lim_{\epsilon\to 0^+}\frac{h^{-n}_{\langle g_\epsilon\rangle}(u)-h^{-n}_{\langle g_0\rangle}(u)}\epsilon\\
&=\lim_{\epsilon\to 0^+}\frac{h^{-n}_{\langle g_\epsilon\rangle}(u)-h^{-n}_{\langle g_0\rangle}(u)}{\color{DarkBlue}{h_{\langle g_\epsilon\rangle}(u)-h_{\langle g_0\rangle}(u)}}\frac{\color{DarkBlue}{h_{\langle g_\epsilon\rangle}(u)-h_{\langle g_0\rangle}(u)}}{\epsilon}\\
&=-nh_{\langle g_0\rangle}^{-n-1}(u)\lim_{\epsilon\to 0^+}\frac{h_{\langle g_\epsilon\rangle}(u)-h_{\langle g_0\rangle}(u)}{\epsilon} \label{eq:4-7}\tag{4-7}
\end{align*}

现在问题又变成了算 $\displaystyle \lim_{\epsilon\to 0^+}\frac{h_{\langle g_\epsilon\rangle}(u)-h_{\langle g_0\rangle}(u)}{\epsilon}$。

首先,取定 $\epsilon>0$,
\begin{align*}
h_{\langle g_\epsilon\rangle}(u)&=\max_{x\in\langle g_\epsilon\rangle}\langle x,u\rangle\\
&=\max\Big\{\langle x,u\rangle:x\in\mathrm{Conv}\big(\{g_\epsilon(v)v,v\in\mathbb{S}^{n-1}\}\big)\Big\}\\
&=:\langle x_\epsilon,u\rangle
\end{align*}
第三行是因为连续函数在紧集上一定存在最大值,因而我们记其最大值点为 $x_\epsilon$;并且进一步,我们知道这个最大值必然在 $\langle g_\epsilon\rangle$ 的极点处取得,换言之,必然有某 $v_\epsilon\in\mathbb{S}^{n-1}$ 使得 $x_\epsilon=g_\epsilon(v_\epsilon)v_\epsilon$。总结起来就是:
\begin{equation}
\begin{aligned}
\exists\, v_\epsilon\in\mathbb{S}^{n-1},&\quad h_{\langle g_\epsilon\rangle }(u)=\langle g_\epsilon(v_\epsilon)v_\epsilon,u\rangle\\
\forall\,v\in\mathbb{S}^{n-1},&\quad h_{\langle g_\epsilon\rangle }(u)\ge\langle g_\epsilon(v)v,u\rangle 
\end{aligned}\label{eq:4-8}\tag{4-8}
\end{equation}

对 $h_{\langle g_0\rangle}(u)$ 同理,
\begin{equation}
\begin{aligned}
\exists\, v_0\in\mathbb{S}^{n-1},&\quad h_{\langle g_0\rangle }(u)=\langle g_0(v_0)v_0,u\rangle\\
\forall\,v\in\mathbb{S}^{n-1},&\quad h_{\langle g_0\rangle }(u)\ge\langle g_0(v)v,u\rangle
\end{aligned} \label{eq:4-9}\tag{4-9}
\end{equation}

综合起来可以推出
\begin{align*}
h_{\langle g_\epsilon\rangle}(u)-h_{\langle g_0\rangle}(u)&\le \langle g_\epsilon (v_\epsilon)v_\epsilon,u\rangle- \langle g_0 (v_\epsilon)v_\epsilon,u\rangle=\langle v_\epsilon,u\rangle\big(g_\epsilon (v_\epsilon)-g_0(v_\epsilon)\big)\\
h_{\langle g_\epsilon\rangle}(u)-h_{\langle g_0\rangle}(u)&\ge \langle g_\epsilon (v_0)v_0,u\rangle- \langle g_0 (v_0)v_0,u\rangle=\langle v_0,u\rangle\big(g_\epsilon (v_0)-g_0(v_0)\big)
\end{align*}
故
\begin{equation}
\langle v_0,u\rangle{\color{red}\underbrace{\frac{g_\epsilon(v_0)-g_0(v_0)}{\epsilon}}_{\Large\text{①}}}\le \frac{h^{-n}_{\langle g_\epsilon\rangle}(u)-h^{-n}_{\langle g_0\rangle}(u)}\epsilon\le {\color{blue}\underbrace{\frac{g_\epsilon(v_\epsilon)-g_0(v_\epsilon)}{\epsilon}}_{\Large\text{②}}}{\color{DarkGreen}\underbrace{\vphantom{\frac{}{}}\langle v_\epsilon,u\rangle}_{\Large\text{③}}} \label{eq:4-10}\tag{4-10}
\end{equation}
这里我们把~\eqref{eq:4-10} 按不同颜色分成了三个部分,需要逐个击破。

\item  先来看 $\Large{\color{red}{\text{①}}}$, 这个比较简单,有
\begin{align*}
g_{\epsilon}(v_0)-g_0(v_0) & =\frac{1}{(h_K+\epsilon f)(v_0)}-\frac1{h_K(v_0)} \\
& =\frac{h_K(v_0)-(h_K(v_0)+\epsilon f(v_0))}{h_K(v_0)(h_K(v_0)+\epsilon f(v_0))} \\
& =\frac{-\epsilon f(v_0)}{h_K^{2}(v_0)+\epsilon h_{k}(v_0)f(v_0)}
\end{align*}
故
\[
\lim_{\epsilon\to 0^+}\frac{g_\epsilon(v_0)-g_0(v_0)}{\epsilon}=-\lim_{\epsilon\to 0^+}\frac{f(v_0)}{h_K^2(v_0)+\epsilon h_K(v_0)f(v_0)}=-\frac{f(v_0)}{h_K^2(v_0)}
\]

接下来看 $\Large{\color{DarkGreen}{\text{③}}}$, 理想情况下,我们期望应该有 $\lim\limits_{\epsilon\to 0^+}v_\epsilon=v_0$。这要怎么证明呢?

首先, $g_\epsilon$ 一致收敛于 $g$, 我们从中任取一列 $\{g_j\}_{j=1}^\infty$, 设 $g_j$ 通过式~\eqref{eq:4-8}选取出来 $v_j$。由于 $\mathbb{S}^{n-1}$ (列) 紧,那么其中任一序列必存在收敛子列,不妨设 $\{v_j\}_{j=1}^\infty$ 收敛到某 $v\in\mathbb{S}^{n-1}$ (不然,可以用其收敛子列替代之)  ;则
\[
g_j(v_j)\to g_0(v),\quad j\to\infty
\]
再根据
\begin{align*}
h_{\langle g_j\rangle}(u)&=g_j(v_j)\langle v_j,u\rangle\to g_0(v)\langle v,u\rangle\\
h_{\langle g_j\rangle}(u)&\to h_{\langle g_0\rangle}(u)
\end{align*}
这两个极限应该是相同的,故 $h_{\langle g_0\rangle}(u)=\langle g_0(v)v,u\rangle$。同时,对于几乎处处 $u\in\mathbb{S}^{n-1}$, 该等式不依赖于 $\{v_j\}_{j=1}^\infty$ 的选取,原因正如之前所说, $u$ 的半径逆 Gauss 映射 $\alpha_{\langle g_0\rangle}^{-1}(u)$ 是几乎处处唯一的,有
\begin{align*}
h_{\langle g_0\rangle}(u)=\langle g_0(v),u\rangle &\implies g_0(v)v\in \nu^{-1}_{\langle g_0\rangle}(u) \\
&\implies v=\frac{g_0(v)v}{\|g_0(v)v\|}=\big(r_{\langle g_0\rangle}^{-1}\circ \nu_{\langle g_0\rangle}^{-1}\big)(u)=\alpha_{\langle g_0\rangle}^{-1}(u)
\end{align*}
所以 (对几乎处处 $u$),  $v$ 是独立于我们序列的选取的,有 $\lim\limits_{\epsilon\to 0^+}v_\epsilon=v=\alpha^{-1}_{\langle g_0\rangle}(u)$。

哎!可别忘了,根据式~\eqref{eq:4-9},  $h_{\langle g_0\rangle}(u)=\langle g_0(v_0)v_0,u\rangle\implies v_0=\alpha^{-1}_{\langle g_0\rangle}(u)$, 那只能有 $v_0=v$。这就证明了  $\lim\limits_{\epsilon\to 0^+}v_\epsilon\to v_0$。

最后算 $\Large{\color{blue}{\text{②}}}$。和 $\Large{\color{red}{\text{①}}}$ 类似,有
\begin{align*}
g_{\epsilon}(v_\epsilon)-g_0(v_\epsilon) & =\frac{1}{(h_K+\epsilon f)(v_\epsilon)}-\frac1{h_K(v_\epsilon)} \\
& =\frac{-\epsilon f(v_\epsilon)}{h_K^{2}(v_\epsilon)+\epsilon h_{k}(v_\epsilon)f(v_\epsilon)}
\end{align*}
于是
\[
\lim_{\epsilon\to 0^+}\frac{g_\epsilon(v_\epsilon)-g_0(v_\epsilon)}{\epsilon}=-\lim_{\epsilon\to 0^+}\frac{f(v_\epsilon)}{h_K^2(v_\epsilon)+\epsilon h_K(v_\epsilon)f(v_\epsilon)}=-\frac{f(v_0)}{h_K^2(v_0)}
\]
最后一步用了 $\Large{\color{DarkGreen}{\text{③}}}$ 的结论。

整理一下,如果 $u\in\mathbb{S}^{n-1}$ 满足 $\alpha^{-1}_{\langle g_0\rangle}(u)$ 唯一,则根据式~\eqref{eq:4-9},
\begin{align*}
\lim_{\epsilon\to 0^+}\frac{h^{-n}_{\langle g_\epsilon\rangle}(u)-h^{-n}_{\langle g_0\rangle}(u)}\epsilon&\ge \langle v_0,u\rangle{\color{red}\underbrace{\frac{g_\epsilon(v_0)-g_0(v_0)}{\epsilon}}_{\Large\text{①}}}=\color{red}{-\frac{f(v_0)}{h^2_K(v_0)}}\langle v_0,u\rangle\\
\lim_{\epsilon\to 0^+}\frac{h^{-n}_{\langle g_\epsilon\rangle}(u)-h^{-n}_{\langle g_0\rangle}(u)}\epsilon&\le {\color{blue}\underbrace{\frac{g_\epsilon(v_\epsilon)-g_0(v_\epsilon)}{\epsilon}}_{\Large\text{②}}}\color{DarkGreen}{\underbrace{\vphantom{\frac{}{}}\langle v_\epsilon,u\rangle}_{\Large\text{③}}} =\color{blue}{-\frac{f(v_0)}{h^2_K(v_0)}}\color{DarkGreen}{\langle v_0,u\rangle}
\end{align*}
即,
\begin{equation}
\lim_{\epsilon\to 0^+}\frac{h^{-n}_{\langle g_\epsilon\rangle}(u)-h^{-n}_{\langle g_0\rangle}(u)}\epsilon= -\frac{f(v_0)}{h^2_K(v_0)} \langle v_0,u\rangle \label{eq:4-11}\tag{4-11}
\end{equation}

\item 来研究一下 $v_0$ 和 $u$ 的关系。由于 $v_0=\alpha^{-1}_{\langle g_0\rangle}(u)=\alpha^{-1}_{K^\circ}(u)$,
\begin{align*}
&h_{K^\circ}(u)=h_{\langle g_0\rangle }(u)=\langle u,g_0(v_0)v_0\rangle=\langle u,\rho_{K^\circ}(v_0)v_0\rangle \tag{由~\eqref{eq:4-9}式}\\
\implies& h_K(v_0)=\frac1{\rho_{K^\circ}(v_0)}=\left\langle\frac u{h_{K^\circ}(u)},v_0\right\rangle=\langle \rho_K(u)u,v_0\rangle\\
\implies& v_0 = \alpha_K(u) \label{eq:4-12}\tag{4-12}
\end{align*}
当然,这里我们仍然可以认为对于几乎处处的 $u\in\mathbb{S}^{n-1}$,  $\alpha_K(u)$ 唯一确定,所以写作 $v_0=\alpha_K(u)$ 没问题。
至此,我们终于能够算出第二步想计算的的目标了。整理一下~\eqref{eq:4-7},
\begin{align*}
&\lim_{\epsilon\to 0^+}\frac{h^{-n}_{\langle g_\epsilon\rangle}(u)-h^{-n}_{\langle g_0\rangle}(u)}\epsilon\\
&=-nh_{\langle g_0\rangle}^{-n-1}(u)\lim_{\epsilon\to 0^+}\frac{h_{\langle g_\epsilon\rangle}(u)-h_{\langle g_0\rangle}(u)}{\epsilon} \tag{由~\eqref{eq:4-7}式}\\
&=nh_{K^\circ}^{-n-1}(u) \frac{f(v_0)}{h^2_K(v_0)} \langle v_0,u\rangle \tag{由 ~\eqref{eq:4-11} 式}\\
&=\frac{nf(v_0)\langle v_0,u\rangle \rho_K^{n+1}(u)}{h^2_K(v_0)}\\
&=\frac{nf(v_0)\overbrace{\langle v_0, \rho_K^{n}(u)u\rangle}^{h_K(v_0)} \rho_K^{n}(u)}{h^2_K(v_0)}\\
&=\frac{nf(v_0)\rho_K^n(u)}{h_K(v_0)}\\
&=\frac{nf(\alpha_K(u))\rho_K^n(u)}{h_K(\alpha_K(u))} \tag{由 ~\eqref{eq:4-12} 式} 
\end{align*}
即,对几乎处处的 $u\in\mathbb{S}^{n-1}$,
\[
\lim_{\epsilon\to 0^+}\frac{h^{-n}_{\langle g_\epsilon\rangle}(u)-h^{-n}_{\langle g_0\rangle}(u)}\epsilon=\frac{nf(\alpha_K(u))\rho_K^n(u)}{h_K(\alpha_K(u))}\label{eq:4-13}\tag{4-13}
\]
这是变分公式的证明中最重要的极限。

\item 证明完极限,我们证明 $\frac{h^{-n}_{\langle g_\epsilon\rangle}(u)-h^{-n}_{\langle g_0\rangle}(u)}\epsilon$ 在 $\mathbb{S}^{n-1}$ 上能被控制,以便能用控制收敛定理搞定~\eqref{eq:4-6}式。也就是说,我们证明:存在一个 $M\in(0,\infty)$ 使得
\[
\left|\frac{h^{-n}_{\langle g_\epsilon\rangle}(u)-h^{-n}_{\langle g_0\rangle}(u)}\epsilon\right|\le M,\quad \forall \,u\in\mathbb{S}^{n-1}
\]
这个其实很简单,直接用式 ~\eqref{eq:4-10}, 有
\[
\left|\frac{h^{-n}_{\langle g_\epsilon\rangle}(u)-h^{-n}_{\langle g_0\rangle}(u)}\epsilon\right|\le\max\Bigg\{\left|\frac{g_\epsilon(v_0)-g_0(v_0)}{\epsilon}\right|,\left|\frac{g_\epsilon(v_\epsilon)-g_0(v_\epsilon)}{\epsilon}\right|\Bigg\}
\]
 (根据 Cauchy-Schwarz 不等式, $|\langle v_0,u\rangle|$ 和 $|\langle v_\epsilon,u\rangle|$ 都不超过 1, 这里将之放缩为了 1。) 

而上式右侧处理起来也很方便:
\[
\left|\frac{g_\epsilon(v_\epsilon)-g_0(v_\epsilon)}{\epsilon}\right|=\left|\frac{f(v_\epsilon)}{h_K(\epsilon)(h_K(\epsilon)+\epsilon f(v_\epsilon))}\right|\le \frac{\max\limits_{v\in\mathbb{S}^{n-1}}|f(v)|}{\min\limits_{v\in\mathbb{S}^{n-1}}h_K^2(v)}=:M
\]
同理 $\displaystyle \left|\frac{g_\epsilon(v_0)-g_0(v_0)}{\epsilon}\right|\le M$。

\item 收尾!根据式~\eqref{eq:4-6},
\begin{align*}
&\lim_{\epsilon\to0^+}\frac{V_n([f_\epsilon])-V_n(K)}{\epsilon}\\
&=\lim_{\epsilon\to 0^+}\frac1n\int_{\mathbb{S}^{n-1}}\frac{h^{-n}_{\langle g_\epsilon\rangle}(u)-h^{-n}_{\langle g_0\rangle}(u)}{\epsilon}\mathrm{d}u \tag{由~\eqref{eq:4-6}式}\\
&=\frac1n\int_{\mathbb{S}^{n-1}}\lim_{\epsilon\to 0^+}\frac{h^{-n}_{\langle g_\epsilon\rangle}(u)-h^{-n}_{\langle g_0\rangle}(u)}{\epsilon}\mathrm{d}u \tag{控制收敛定理}\\
&= \frac1n \int_{\mathbb{S}^{n-1}}\frac{nf(\alpha_K(u))}{h_K(\alpha_K(u))}\rho_K^n(u)\mathrm{d}u \tag{由~\eqref{eq:4-13} 式}\\
&=\int_{\mathbb{S}^{n-1}}\frac{f(v)}{h_K(v)}h_K(v)\mathrm{d}S(K,v) \tag{$v=\alpha_K(u)$ 换元}\\
&=\int_{\mathbb{S}^{n-1}}f(v)\mathrm{d}S(K,v)
\end{align*}
得证!其中第四行用到了我们前文推论~\ref{col:map_diagram}。\end{enumerate}
\end{proof}

\begin{remark}
    上述证明来自 Huang, Y., Lutwak, E., Yang, D., \& Zhang, G. (2016). \textit{Geometric measures in the dual Brunn–Minkowski theory and their associated Minkowski problems}. Acta Mathematica, 216(2), 325-388。
\end{remark}
\hfill\break

在凸几何中,混合体积的变分公式~\eqref{eq:4-4}是非常重要的结论,有相当一部分研究对象都可以视作是 $V_1(K,L)$ 的优化问题。包括但不限于

\begin{itemize}
\item Minkowski 问题;
\item Petty 体 (Petty bodies);
\item 著名的 John 椭球 (John ellipsoid, 也称 Löwner–John ellipsoid);
\item 仿射表面积 (affine surface areas)。
\end{itemize}
因此单独一个混合体积的变分公式可以引申出来解决很多问题。
  

\chapter{Minkowski 问题}\label{chapter:5}

\section{引入}

我们回顾一下在第~\ref{chapter:4} 章得到的两个重要定理:Minkowski 第一不等式 (定理~\ref{thm:minkowski})
\[
V_1(K,L)\ge V_n(K)^{\frac{n-1}{n}}V_n(L)^{\frac1n}
\]
其中 $K,L\subset\mathbb{R}^n$ 是凸体,取等当且仅当 $K,L$ 位似;以及混合体积的变分公式 (定理~\ref{thm:var}) 
\[
V_1(K,L)=\frac1n\int_{\mathbb{S}^{n-1}}h_L(u)\mathrm{d}S(K,u)
\]
其中 $h_L(u)$ 是 $L$ 的支撑函数、 $\mathrm{d}S(K,u)$ 是 $K$ 的表面积测度,混合体积 $V_1(K,L):=\frac1n\frac{\mathrm{d}}{\mathrm{d}t}V_n(K+tL)$。

把上面两式结合到一起,可得:
\[
\frac1n\int_{\mathbb{S}^{n-1}}h_L(u)\mathrm{d}S(K,u)\ge V_n(K)^{\frac{n-1}{n}}V_n(L)^{\frac 1n}
\]
我们能从中看出什么呢?

考虑这样的操作,
\begin{itemize}
\item 把 $L$ 按体积归一化,也就是用 $\frac L{V_n(L)^{1/n}}$ 取代 $L$,因而 $h_L$ 也变作 $\frac{h_L}{V_n(L)^{1/n}}$;
\item 把 $K$ 的表面积测度也 (某种程度上) 归一化,考虑测度 $\mathrm{d}\mu:=\frac{\mathrm{d}S(K,u)}{V_n(K)^{\frac{n-1}{n}}}$。
\end{itemize}
那么应有如下不等式总成立
\[
\frac1n\int_{\mathbb{S}^{n-1}}h_L(u)\mathrm{d}\mu(u)\ge 1
\]
取等当且仅当 $K,L$ 位似。

注意到测度 $\mathrm{d}\mu$ 在 $K$ 的位似变化下是不变的 (后文详细解释),因而若固定 $L$,则\emph{如下关于测度 $\mu$ 的方程}
\[
\frac1n\int_{\mathbb{S}^{n-1}}h_L(u)\mathrm{d}\mu(u)= 1
\]
\emph{只有一个解},这个解可以写作 $\mathrm{d}\mu=\mathrm{d}S(L,u)$。

或者,我们也能固定测度 $\mu$ 而变化 $L$,可以这样说:若 $\mu$ 是某个凸体 $L_0\subset\mathbb{R}^n$ 的表面积测度, $V_n(L_0)=1$,那么 $L_0$ 是如下极值问题的解,
\begin{equation}
\inf\left\{\int_{\mathbb{S}^{n-1}}h_L\mathrm{d}\mu: V_n(L)=1,L\subset\mathbb{R}^n\text{ 是凸体 }\right\} \label{eq:5-1}\tag{5-1}
\end{equation}

这表明,表面积测度可以作为凸体本身的``变分"特征——我们在后文中甚至会表明:\emph{凸体其实可以唯一由其表面积测度决定}。于是一个顺理成章的问题就出现了:既然某些测度能刻画凸体,那么如果从反方向出发——给定一个 $\mathbb S^{n-1 }$ 的 Borel 测度 $\mu$——是否也能找到一个凸体 $L_0$,使得它的表面积测度正好是 $\mu$?这就是著名的 \emph{Minkowski 问题}。

\begin{example}
先举一个最简单的例子,在 $\mathbb{R}^2$ 中,Borel 测度 $\mu$ 聚集在"上下左右" $\{u_1,u_2,u_3,u_4\}$ 四个方向上,且对每个 $u_i$ 都有 $\mu(u_i)=1$。

有没有一个凸体 $L\subset\mathbb{R}^2$ 使得 $\mu=S(L,\cdot)$?答案是显然的,取一个边长为 1 的正方形就行了。见图~\ref{fig:square}。
\end{example}
\begin{figure}[h]
\centering
\includegraphics[width=0.3\textwidth]{figures/square.pdf}
\caption{正方形满足表面积测度集中在四个方向上。}\label{fig:square}
\end{figure}

此外还可以引申一些,比如,
\begin{example}\label{ex:petty_body}
    我们能不能把 $L$ 换成它的极集 $L^\circ$?换言之,如下极值问题
\[
\inf\left\{\int_{\mathbb{S}^{n-1}}h_L\mathrm{d}\mu: V_n({\color{blue}L^\circ})=1,L\subset\mathbb{R}^n\text{ 是凸体 }\right\}
\]
又如何呢?答案是,上述问题的解称作 \emph{Petty 体} (Petty Body) \footnote{C. M. Petty, 1974},不过这个不再具体介绍。
\end{example}

\begin{example}
    再比如,由于平移不变性我们让 $0\in\mathrm{int}L$, 根据引理~\ref{lem:support_radial},上述引申出来的问题又等价于
\begin{align*}
&\inf\left\{\int_{\mathbb{S}^{n-1}}h_{{\color{blue}L^\circ}}(u)\mathrm{d}S(K,u): V_n(L)=1,\,0\in\mathrm{int}L,\,L\subset\mathbb{R}^n\text{ 是凸体 }\right\} \\
=\;& \inf\left\{\int_{\mathbb{S}^{n-1}}\frac1{\rho_L(u)}\mathrm{d}S(K,u): V_n(L)=1,\,0\in\mathrm{int}L,\,L\subset\mathbb{R}^n\text{ 是凸体 }\right\}
\end{align*}

这能不能再推广到更一般的集合上?
\[
\inf\left\{\int_{\mathbb{S}^{n-1}}\frac1{\rho_L(u)}\mathrm{d}S(K,u): V_n(L)=1,L\subset\mathbb{R}^n\text{ 是}{\color{red}\text{星形体}}\right\}
\]
上述问题的答案取决于 $K$ 的\emph{仿射表面积} (affine surface area) \footnotemark。
\end{example}
\footnotetext{E. Lutwak, 1991}

这里补充一下所谓星形体的定义
\begin{definition}[星形集和星形体]
非空集合 $S\subset\mathbb{R}^n$ 若满足:存在 $x_0\in S$,使得对任意 $s\in S$, $x_0$ 和 $s$ 所连线段都在 $S$ 内,则称 $S$ 是关于 $x_0$ 的\emph{星形集} (star set)。

若紧集 $S\subset\mathbb{R}^n$ 是关于原点 0 的星形集,且其半径函数 $\rho_S(u):=\sup\{\lambda>0:\lambda u\in S\}$ 是 $\mathbb{S}^{n-1}$ 上的正值连续函数,那么称 $S$ 是一个 (关于原点的) \emph{星形体} (star body)。%见图~\ref{fig:star_body}。
\end{definition}

\begin{figure}[htbp]
\centering
\includegraphics[width=0.3\textwidth]{figures/Star_domain.pdf}
\caption{星形集, 或称 \emph{Star domain} 的示意图;可以看出星形集并不一定是凸集。\footnotemark}
\label{fig:star_body}
\end{figure}
\footnotetext{图片来自 \url{https://en.wikipedia.org/wiki/Star_domain}}

根据定义,任意非空凸集都是星形集,任意 $0\in\mathrm{int}L$ 的凸体 $L\subset\mathbb{R}^n$ 都是关于原点的星形体 (为什么?),因此星形体可以视作凸体的推广。

\section{表面积测度的一些性质}

之前我们没有仔细地考察凸体 $K\subset\mathbb{R}^n$ 的表面积测度 $S(K,\cdot)$ 的性质。为了给后文做准备,先证明一些。

\begin{proposition}\label{prop:surface_trans}
表面积测度是平移不变的,即对任意 $x_0\in\mathbb{R}^n$, $S(K,\cdot)=S(K+x_0,\cdot)$。
\end{proposition}

\begin{proof}
注意到 $h_{K+x_0}(u)=h_K(u)+\langle u,x_0\rangle$,因而 \hyperref[def:wulff]{Wulff 形}在平移下对应有
\[
[h_{K+x_0}]=[h_K+\langle x_0,\cdot\rangle]=[h_K]+x_0
\]
所以 Wulff 形的体积是不变的: $V_n([h_{K+x_0}])=V_n([h_K])$。

根据定理~\ref{thm:var_wulff},对任意 $\mathbb{S}^{n-1}$ 上的正连续函数 $f\in\mathcal{C}^+\big(\mathbb S^{n-1}\big)$,
\begin{align*}
\int_{\mathbb S^{n-1}}f\,\mathrm{d}S(K,u)&=\lim_{\epsilon\to0^+}\frac{V_n([h_K+\epsilon f])-V_n([h_K])}{\epsilon}\\
&=\lim_{\epsilon\to0^+}\frac{V_n([h_{K+x_0}+\epsilon f])-V_n([h_{K+x_0}])}{\epsilon}\\
&=\int_{\mathbb S^{n-1}}f\,\mathrm{d}S(K+x_0,u)
\end{align*}
即,紧空间 $\mathbb{S}^{n-1}$ 上的有限测度 $\mathrm{d}S(K,u)$ 和 $\mathrm{d}S(K+x_0,u)$ 满足对所有 $f\in\mathcal{C}^+\big(\mathbb S^{n-1}\big)$ 都有
\[
\int_{\mathbb S^{n-1}}f\,\mathrm{d}S(K,u)=\int_{\mathbb S^{n-1}}f\,\mathrm{d}S(K+x_0,u)
\]
自然只能有 $S(K,\cdot)=S(K+x_0,\cdot)$。
\end{proof}

\begin{remark}
证明方法不止这一种,读者可以尝试用其它方法证明此性质。
\end{remark}

\begin{proposition}\label{prop:surface_scaling}
设 $\lambda>0$,则表面积测度满足 $S(\lambda K,\cdot)=\lambda^{n-1}S(K,\cdot)$。
\end{proposition}

\begin{proof}
注意到 $h_{\lambda K}(u)=\lambda h_K(u)$,于是 $[h_{\lambda K}]=[\lambda h_K]=\lambda [h_K]$,故
\begin{align*}
\int_{\mathbb S^{n-1}}f\,\mathrm{d}S(\lambda K,u)&=\lim_{\epsilon\to0^+}\frac{V_n([h_{\lambda K}+\epsilon f])-V_n([h_{\lambda K}])}{\epsilon}\\
&=\lim_{\epsilon\to0^+}\lambda^{n-1}\frac{ V_n([h_{K}+\frac\epsilon\lambda f])-V_n([h_{  K}])}{\frac\epsilon\lambda}\\
&=\lambda^{n-1}\int_{\mathbb S^{n-1}} f\,\mathrm{d}(K,u)
\end{align*}
基于命题~\ref{prop:surface_trans} 同样的原因, $S(\lambda K,\cdot)=\lambda^{n-1}S( K,\cdot)$。
\end{proof}

\begin{remark}
性质~\ref{prop:surface_trans}  和~\ref{prop:surface_scaling}说明测度 $\mathrm{d}\mu:=\frac{\mathrm{d}S(K,u)}{V_n(K)^{\frac{n-1}{n}}}$ 在 $K$ 的平移和缩放——即位似变换下确实是不变的。
\end{remark}

\begin{proposition}\label{prop:surface_centroid}
$S(K,\cdot)$ 的重心在原点,即 $\displaystyle \int_{\mathbb{S}^{n-1}}u\,\mathrm{d}S(K,u)=0$。
\end{proposition}

\begin{proof}
用混合体积变分公式 (定理~\ref{thm:var}),对任意 $x_0\in\mathbb{R}$,
\begin{align*}
\int_{\mathbb{S}^{n-1}}h_{L+x_0}(u)\mathrm{d}S(K,u)&=\lim_{\epsilon\to 0^+}\frac{V_n(K+\epsilon(L+x_0)-V_n(K))}{\epsilon}\\
&=\lim_{\epsilon\to 0^+}\frac{V_n(K+\epsilon(L)-V_n(K))}{\epsilon}\\
&= \int_{\mathbb{S}^{n-1}}h_{L }(u)\mathrm{d}S(K,u)
\end{align*}
所以
\begin{align*}
0&=\int_{\mathbb{S}^{n-1}}\Big(h_{L+x_0}(u)-h_L(u)\Big)\,\mathrm{d}S(K,u)\\
&=\int_{\mathbb{S}^{n-1}}\langle x_0,u\rangle\,\mathrm{d}S(K,u)\\
&=\left\langle x_0,\int_{\mathbb{S}^{n-1}}u\,\mathrm{d}S(K,u)\right\rangle
\end{align*}
这对任意 $x_0$ 都成立,自然只能有 $\displaystyle \int_{\mathbb{S}^{n-1}}u\,\mathrm{d}S(K,u) =0$。
\end{proof}

\begin{proposition}\label{prop:surface_great_circle}
$S(K,\cdot)$ 不集中在任何一个\emph{大圆} (great subsphere) 上。
\end{proposition}

\begin{remark}
    这里,所谓``大圆"指的是:取方向 $u\in\mathbb S^{n-1}$,其正交补空间与球面 $\mathbb{S}^{n-1}$ 的交集 $\mathbb{S}^{n-1}\cap u^\perp$,它应该是一个 $n-2$ 维的球面 ( $n\ge 2$)。换言之, $S(K,\cdot)$ 的质量不能全部落在 $\mathbb{R}^{n}$ 的更低维度的子空间中。
\end{remark}
\begin{proof}
反证法,谬设 $S(K,\cdot)$ 集中在某个大圆 $\mathbb{S}^{n-1}\cap u^\perp$ 内,记 $A:=\{v:\langle v,u\rangle>0\}$ 和 $B:= \{v:\langle v,u\rangle<0\}$ 代表两个开半球面,则 $S(K,A)=0$、 $S(K,B)=0$,即 $\mathcal{H}^{n-1}(\nu_K^{-1}(A))=0$、 $\mathcal{H}^{n-1}(\nu_K^{-1}(B))=0$——但是, $\nu^{-1}_K(A)$ 和 $\nu_K^{-1}(B)$ 分别包含了 $\partial K$ 上所有法向量在 $A$、 $B$ 中的点,这表明在 $\partial K$ 中几乎所有的点法向量不在 $A,B$ 中,即与 $u$ 正交。

根据散度定理,
\begin{align*}
V_n(K)&=\int_K\mathrm{d}x=\int_K\mathrm{div}(\langle x,u\rangle u)\,\mathrm{d}x\\
&=\int_{\partial K}\langle x,u\rangle\underbrace{\langle u,\nu_K(x)\rangle}_{\text{几乎处处为 0}}\,\mathrm{d}\mathcal{H}^{n-1}(x)\\
&=0
\end{align*}
与 $K$ 是凸体(内点非空)矛盾!
\end{proof}

\begin{proposition}\label{prop:surface_half}
对任意 $u\in\mathbb{S}^{n-1}$,开半球面 $\{v\in\mathbb{S}^{n-1}:\langle v,u\rangle>0\}$ 的 $S(K,\cdot)$ 测度都是正的。更一般地,任意 $\mathbb{S}^{n-1}$ 上的 Borel 测度 $\mu$ 若满足前面性质~\ref{prop:surface_great_circle}、\ref{prop:surface_half} ,则该性质均成立。
\end{proposition}

\begin{proof}
记集合 $A:=\{v\in\mathbb S^{n-1},\langle v,u\rangle >0\}$、 $B:=\{v\in\mathbb S^{n-1},\langle v,u\rangle <0\}$、 $E:=\{v\in\mathbb S^{n-1},\langle v,u\rangle =0\}$,那么根据命题~\ref{prop:surface_great_circle},
\begin{align*}
0&=\left\langle v,\int_{\mathbb{S}^{n-1}}u\,\mathrm d\mu(u)\right\rangle=\int_{\mathbb{S}^{n-1}}\langle v,u\rangle\,\mathrm d\mu(u) \\
&=\int_A \langle v,u\rangle\,\mathrm d\mu(u)+\int_B\langle v,u\rangle\,\mathrm d\mu(u)+\underbrace{\int_E\langle v,u\rangle\,\mathrm d\mu(u)}_{=0}\\
&=\int_A \langle v,u\rangle\,\mathrm d\mu(u)+\int_B\langle v,u\rangle\,\mathrm d\mu(u)
\end{align*}
上面的两个积分中,第一项的被积分项恒正、第二项的被积分项恒负,若 $\mu(A)=0$ 则第一项积分为 0,于是第二项积分也只能为 0,则 $\mu(B)=0$。这样一来, $\mu$ 的质量就全都集中在了大圆 $E$ 上,与命题~\ref{prop:surface_half}  矛盾!所以一定 $\mu(A)> 0$。
\end{proof}

\begin{lemma}[Hausdorff 测度的上半连续性]\label{lem:hausdorff_semi_cont}
设 $F_0,F_1,F_2,\ldots\subseteq\partial K$ 是紧集,且 $\{F_j\}_{j=1}^\infty$ 在 Hausdorff 度量下收敛: $F_j\to F_0$,那么 $\limsup\limits_{j \to \infty} \mathcal{H}^{n-1}(F_j) \leq \mathcal{H}^{n-1}(F_0)$。
\end{lemma}

\begin{proof}
由于 $F_0$ 紧,它可以表示为递减序列 $\displaystyle N_m := \left( F_0 + \frac{1}{m} B_n \right) \cap \partial K$ 的交集,因此 $\mathcal{H}^{n-1}(F_0) = \lim\limits_{m \to \infty} \mathcal{H}^{n-1}(N_m)$;又每个 $N_m\supset F_0$,对于足够大的 $j$,有 $F_j \subseteq N_m$,于是每个 $m$ 都有 $\limsup\limits_{j \to \infty} \mathcal{H}^{n-1}(F_j) \leq \mathcal{H}^{n-1}(N_m)$,引理得证。
\end{proof}

\begin{proposition}\label{prop:surface_weak_converge}
表面积测度在 Hausdorff 度量下有弱收敛性质,即,设 $K_0,K_1,K_2,\dots\subset\mathbb{R}^n$ 是凸体,且在 Hausdorff 度量下有收敛 $K_i\to K_0$,则测度 $S(K_i,\cdot)$ 弱收敛于 $S(K_0,\cdot)$,亦即,对任意 $f\in\mathcal{C}\big(\mathbb S^{n-1}\big)$ 都有 $\displaystyle\int_{\mathbb S^{n-1}}f\,\mathrm{d}S(K_i,u)\to\int_{\mathbb S^{n-1}}f\,\mathrm{d}S(K,u)$。
\end{proposition}

\begin{proof}
要证 $S(K_i,\cdot)$ 在 $\mathbb{S}^{n-1}$ 上弱收敛到 $S(K_0,\cdot)$,根据 \emph{Portmanteau 定理}\footnote{\url{https://en.wikipedia.org/wiki/Convergence_of_measures\#Weak_convergence_of_measures}},我们只需证明对任意紧集 $F\subseteq\mathbb{S}^{n-1}$ 都有 $ \limsup\limits_{i\to\infty}S(K_i,F)\le S(K_0,F)$。

固定紧集 $F\subseteq\mathbb{S}^{n-1}$,不失一般性假设原点 $0$ 为所有 $K_i,K_0$ 的内点。定义沿射线的边界对应映射 
\[
\psi_i:\partial K_i\to\partial K_0,\qquad \psi_i(x):=r_{K_0}\big(r^{-1}_{K_i}(x)\big)=r_{K_0}\big(x/\|x\|\big)
\]
( $r_K$ 见定义~\ref{def:radial_func})。

不难看出 $\psi_i$ 与其逆在对应的边界上均为\emph{双 Lipschitz},且其 Lipschitz 常数趋于 1:直观地说,$\psi_i$ 仅在固定方向上把 $\partial K_i$ 的在该方向上的点替换为 $\partial K_0$ 上同方向的点,随着 $i\to\infty$,两者的半径之比趋近于 1。

令 $\nu_K:\partial K\to\mathbb{S}^{n-1}$ 表示外法向的 Gauss 映射。对紧集 $F\subseteq\mathbb{S}^{n-1}$,考虑集合 $E_i:=\nu_{K_i}^{-1}(F)\subseteq\partial K_i$ 及其经 $\psi_i$ 的像 $G_i:=\psi_i\big(E_i\big)\subseteq\partial K_0$。

\begin{figure}[htbp]
\centering
\includegraphics[width=0.7\textwidth]{figures/weak_converge_proof.pdf}
\caption{$\psi_i$ 构建起了连接 $\partial K_i$ 子集 $E_i$ 与 $\partial K_0$ 子集 $G_i$ 的桥梁。}
\end{figure}

由紧集的 \emph{Blaschke 选择定理}
\footnote{Blaschke 选择定理不仅对凸体成立,也可以推广到紧集上。对于 $\mathbb{R}^n$ 中的一列一致有界的紧集,一定存在 Hausdorff 度量意义下的收敛子列。},
 $\{G_i\}$ 在 Hausdorff 度量下存在子列收敛到某个紧集 $G_0\subseteq\partial K_0$,不妨设 $G_i\to G_0$。

我们先证明 $G_0\subseteq \nu_{K_0}^{-1}(F)$。任取 $x\in G_0$,存在点列 $y_i\in G_i$ 使 $y_i\to x$。写 $y_i=\psi_i(x_i)$ 且 $x_i\in E_i$。由 $\psi_i$ 的双 Lipschitz 性及常数趋于 1,可知 $x_i\to x$。对每个 $i$ 取 $u_i\in\nu_{K_i}(x_i)$ 作为 $x_i$ 处的法向,即 $K_i\subseteq H^-_{u_i,\langle u_i,x_i\rangle}$ 且 $\langle u_i,x_i\rangle=h_{K_i}(u_i)$。由紧性取子列使 $u_i\to u\in F$。结合 $K_i\to K_0$、$x_i\to x$、$u_i\to u$,传递到极限得到 $K_0\subset H^-_{u,\langle u,x\rangle}$ 且 $\langle u,x\rangle=h_{K_0}(u)$,从而 $x\in \nu_{K_0}^{-1}(F)$。于是 $G_0\subseteq\nu_{K_0}^{-1}(F)$。

将引理 1 应用于 $G_i\to G_0$ 得 
\[
\limsup\limits_{i\to\infty}\mathcal{H}^{n-1}(G_i)\le \mathcal{H}^{n-1}(G_0)\le \mathcal{H}^{n-1}\big(\nu_{K_0}^{-1}(F)\big)
\]
再借助 $\psi_i$ 的双 Lipschitz 性,将上述不等式从像集 $G_i$ 传回原集 $E_i$:
\[
\limsup_{i\to\infty}\mathcal{H}^{n-1}\big(E_i\big)\le \mathcal{H}^{n-1}\big(\nu_{K_0}^{-1}(F)\big)
\]
因面积测度满足 $S(K_i,F)=\mathcal{H}^{n-1}(\nu_{K_i}^{-1}(F))=\mathcal{H}^{n-1}(E_i)$,上式即 
\[
\limsup_{i\to\infty}S(K_i,F)\le S(K_0,F)
\]
\end{proof}

\begin{remark}
    上述证明思路参考了 Gruber, P. M. (2007). \textit{Convex and discrete geometry}. Berlin, Heidelberg: Springer Berlin Heidelberg. 的 Proposition 10.2. (pp. 190-192)。
\end{remark}

\hfill\break

前面这些性质也可以视作是某个 $\mathbb{S}^{n-1}$ 上的测度 $\mu$ 能够作为某凸体的表面积测度 $S(K,\cdot)$ 的必要条件。那么有没有充分条件呢?这就是 Minkowski 问题了。

\section{Minkowski 问题的解决}

{
    \renewcommand{\problemname}{Minkowski 问题}
    \renewcommand{\theprob}{} % 清空编号
\begin{problem}
给定任意一个 $\mathbb{S}^{n-1}$ 上的 Borel 测度,是否能够找到一个凸体 $K\subset\mathbb{R}^n$,使得 $\mu=S(K,\cdot)$?这样的 $K$ 是否唯一?
\end{problem}
}
\subsection{唯一性}

先解决一个比较容易的结论:若满足 Minkowski 问题的 $K$ 存在,则其在至多相差一个平移的意义下一定是唯一的。

\begin{theorem}\label{thm:minkowski_unique}
设对于 $\mathbb{S}^{n-1}$ 上的 Borel 测度 $\mu$,凸体 $K,L\subset\mathbb{R}^n$ 均满足
\begin{align*}
\mu&=S(K,\cdot)\\
\mu&=S(L,\cdot)
\end{align*}
那么 $K,L$ 至多相差一个平移。
\end{theorem}

\begin{proof}
\begin{align*}
\frac1n\int_{\mathbb{S}^{n-1}}h_K\mathrm{d}\mu&= \frac1n\int_{\mathbb{S}^{n-1}}h_K\mathrm{d}S(K,\mu)=V_n(K)\tag{由~\eqref{eq:4-2}}\\
\frac1n\int_{\mathbb{S}^{n-1}}h_K\mathrm{d}\mu&=\frac1n\int_{\mathbb{S}^{n-1}}h_K\mathrm{d}S(L,\mu)=V_1(L,K)\tag{由定理~\ref{thm:minkowski}}
\end{align*} 

得到 $V_n(K)=V_1(L,K)$。

但再根据 Minkowski 第一不等式 (定理~\ref{thm:minkowski}), $V_1(L,K)\ge V_n(L)^{\frac{n-1}n}V_n(K)^{\frac1n}$,带入上式,就得到了 $V_n(K)\ge V_n(L)$。

同理,交换一下 $K$ 和 $L$,我们还能得到 $V_n(L)\ge V_n(K)$,因而只能有 $V_n(K)=V_n(L)$。
此时,Minkowski 第一不等式是取等的,因为
\[
V_1(L,K)=V_n(K)=V_n(K)^{\frac {n-1}n}V_n(L)^{\frac1n}
\]
因而 $K,L$ 位似——但二者体积又相等,自然只能至多相差一个平移了。
\end{proof}

\begin{remark}
    上述定理也可参考 Schneider, R. (2013). \textit{Convex bodies: the Brunn–Minkowski theory} (Vol. 151). Cambridge university press. 的 Theorem 8.1.1,有时也称作 \emph{Aleksandrov-Fenchel-Jessen 定理}。
\end{remark}

\subsection{存在性}

下面的定理阐释了``一个 $\mathbb{S}^{n -1}$ 上的 Borel 测度 $\mu$ 能够作为某凸体的表面积测度"的充分条件(实际上是充要条件)。

\begin{theorem}\label{thm:minkowski_exist}
设 $\mu$ 是 $\mathbb{S}^{n-1}$ 上 的 Borel  测度,若如下两个条件成立:

\begin{enumerate}[label=\textbf{条件 \chinese*}:, leftmargin=5em]
\item 前文的性质~\ref{prop:surface_centroid} 成立,即 $\mu$ 的重心在原点: $\displaystyle\int_{\mathbb{S}^{n-1}}u\,\mathrm d\mu(u)=0$;
\item 前文的性质~\ref{prop:surface_great_circle} 成立, $\mu$ 不能集中在任何一个大圆上。
\end{enumerate}

那么存在一个凸体 $K\subseteq\mathbb{R}^n$ 使得 $\mu=S(K,\cdot)$。
\end{theorem}
\begin{proof}
    
先整理一下证明思路。我们在本文一开始就提出过这个想法:要想解 Minkowski 问题,某种程度下就是要让 Minkowski 第一不等式取到等号(见式~\eqref{eq:5-1})。

另外根据条件 1,积分 $\displaystyle \int_{\mathbb{S}^{n-1}}h_L\,\mathrm{d}\mu$ 是平移不变的:
\begin{align*}
\int_{\mathbb{S}^{n-1}}h_{L+x_0}\,\mathrm{d}\mu&=  \int_{\mathbb{S}^{n-1}}\Big(h_L(u)+\langle x_0,u\rangle\Big)\,\mathrm{d}\mu(u)\\
&= \int_{\mathbb{S}^{n-1}}h_L\,\mathrm{d}\mu+\underbrace{\left\langle x_0, \int_{\mathbb{S}^{n-1}}u\mathrm{d}\mu(u)\right\rangle}_{0}\\
&= \int_{\mathbb{S}^{n-1}}h_L\,\mathrm{d}\mu
\end{align*}
所以问题~\eqref{eq:5-1} 是平移不变的,我们可以不失一般性地把~\eqref{eq:5-1} 改写作
\begin{equation}
\theta:=\inf\left\{ \int_{\mathbb{S}^{n-1}}h_L\,\mathrm{d}\mu:V_n(L)=1,\;L\text{ 是凸体 },\;0\in\mathrm{int}L\right\} \label{eq:5-2}\tag{5-2}
\end{equation}

只要解决了上述问题,Minkowski 问题也就完成一半了。接下来的证明分两步走。

\textbf{第一步} (证明 ~\eqref{eq:5-2} 存在解):

观察 ~\eqref{eq:5-2},首先,显然 ~\eqref{eq:5-2} 是良定义的,因为可行域非空 (如单位体积球就是一个可行的 $L$),则 $\theta$ 一定有界;其次,得益于 $0\in\mathrm{int}L$, $h_L>0$ 总成立,则目标积分也是正值,下确界一定非负。总的来说,有 $0\le \theta<\infty$。

在 ~\eqref{eq:5-2} 的可行域中找一列 $\{L_j\}_{j=1}^\infty$ 使得 $\displaystyle \lim_{i\to\infty}\int_{\mathbb S^{n-1}}h_{L_j}\,\mathrm{d}\mu=\theta$。为了能对这列凸体用 \emph{Blaschke 选择定理},我们需要其一致有界,亦即,存在 $R>0$ 使得 $L_j\subseteq R B_n,\;\forall j$( $B_n$ 代表 $\mathbb{R}^n$ 中的单位球)。

用反证法,记 $\rho_{L_j}$ 代表 $L_j$ 的半径函数,定义 $R_j:=\max\limits_{u\in\mathbb S^{n-1}}\rho_{L_j}(u)$,谬设 $\sup\limits_{j\in\mathbb Z_+} R_j=\infty$——不失一般性地,可以 $\lim\limits_{j\to\infty}R_j=\infty$(不然总可以找到一个子序列替代之)。存在 $u_j\in\mathbb S^{n-1}$ 使得 $R_j=\rho_{L_j}(u_j)$,因 $\mathbb S^{n-1}$(列) 紧, $\{u_j\}_{j=1}^\infty$ 存在收敛子列收敛到 $u_0$,不妨设 $u_j\to u_0$。根据半径函数的定义,从原点 0 到 $R_ju_j$ 所连线段都包含在 $L_j$ 内,记该线段为 $[0,R_ju_j]$,算下该线段的支撑函数,显然
\[
h_{[0,R_ju_j]}(v)=\max_{x\in[0,R_ju_j]}\langle x,v\rangle=\begin{cases} 0&\text{ 若 }\langle v,u_j\rangle\le 0\\ \langle v,R_ju_j\rangle,&\text{ 若 }\langle v,u_j\rangle>0 \end{cases}
\]
因此
\begin{equation}
\int_{\mathbb S^{n-1}}h_{L_j}(v)\,\mathrm d\mu(v)\ge R_j\int_{\{v\in\mathbb{S}^{n-1}:\langle v,u_j\rangle>0\}}\langle v,u_j\rangle \,\mathrm d\mu(v) \label{eq:5-3}\tag{5-3}
\end{equation}

对上式取极限——左侧的极限已知是 $\theta$,而右侧是
\begin{align*}
&\lim_{j\to\infty}R_j\int_{\{v\in\mathbb{S}^{n-1}:\langle v,u_j\rangle> 0\}}\langle v,u_j\rangle \,\mathrm d\mu(v)\\
=\;& \lim_{j\to\infty }R_j\lim_{j\to\infty }\int_{\{v\in\mathbb{S}^{n-1}:\langle v,u_j\rangle> 0\}}\langle v,u_j\rangle\,\mathrm d\mu(v)\\
=\;& \underbrace{\vphantom{\int_{_{_{_{}}}}}\lim_{j\to\infty }R_j}_{\infty}\underbrace{\int_{\{v\in\mathbb{S}^{n-1}:\langle v,u_0\rangle>0\}}\langle v,u_0\rangle\,\mathrm d\mu(v)}_{>0} \tag{控制收敛定理}\\
=\;& \infty
\end{align*}
第三行的积分大于 0 用到了命题~\ref{prop:surface_half}。

于是,对 ~\eqref{eq:5-3} 式一取极限就变成了 $\theta>\infty$,矛盾! $\sup\limits_{j\in\mathbb Z_+} R_j<\infty$ 得证。

根据 \emph{Blaschke 选择定理}, $L_j$ 存在收敛子列收敛到闭凸集 $L_0$——这个 $L_0$ 一定是非退化的凸体,因为体积对 Hausdorff 度量满足连续性,有 $V_n(L_0)=1$。不失一般性设 $0\in\mathrm{int}L_0$(不然可以平移 $L_0$ 使之成立),那么 $L_0$ 就是 ~\eqref{eq:5-2} 的最优解。

\textbf{第二步} (证明 $\mu$ 是表面积测度): 至此我们得到了一个 $L_0$,但要怎么用它呢?首先,利用 Wulff 形构造一个新的问题:
\begin{equation}
\widetilde{\theta\,}:=\inf\left\{\int_{\mathbb S^{n-1}}f\,\mathrm{d}\mu:V_n([f])=1,\;f\in\mathcal{C}^+\big(\mathbb S^{n-1}\big)\right\} \label{eq:5-4}\tag{5-4}
\end{equation}

不难看出, ~\eqref{eq:5-2}、~\eqref{eq:5-4} 最优解的值是一样的,也就是说 $\theta=\widetilde{\theta\,}$,因为

\begin{itemize}
\item 一方面,Wulff 形 $[f]$ 是一个凸体且 $0\in\mathrm{int}[f]$,且根据命题~\ref{prop:wulff_support}, $h_{[f]}\le f$ 在 $\mathbb S^{n-1}$ 上总成立,于是我们有 $\displaystyle \int_{\mathbb S^{n-1}}h_{[f]}\,\mathrm d\mu\le \displaystyle \int_{\mathbb S^{n-1}}f\,\mathrm d\mu $。这得出 $\widetilde{\theta\,}\ge\theta$。
\item 另一方面,其实相比于~\eqref{eq:5-2},~\eqref{eq:5-4} 的范围其实是扩大了,因为对任意满足 $0\in\mathrm{int}L$ 的凸体 $L$ 都有 $L=[h_L]$ (命题~\ref{prop:wulff_eq})。同时 $h_L\in\mathcal{C}^+\big(\mathbb S^{n-1}\big)$ 总是成立的。由于条件放松,有 $\widetilde{\theta\,}\le \theta$。
\end{itemize}

于是 $h_{L_0}$ 就也是~\eqref{eq:5-4}的解。
\hfill\break

下面是一个很巧妙的处理:任取连续函数 $g\in\mathcal C(\mathbb S^{n-1})$,我们考虑一个"二维扰动"函数 $h_{L_0}+tg+s$, $s,t\in\mathbb{R}$(可以让二者充分接近 0,使得 $h_{L_0}+tg+s>0$)。根据变分公式,映射$(s,t)\mapsto V_n([h_{L_0}+tg+s])$ 有梯度
\begin{align*}
\frac{\partial}{\partial t}V_n([h_{L_0}+tg+s])&=\int_{\mathbb{S}^{n-1}}g\,\mathrm{d}S([h_{L_0}+tg+s])\\
\frac{\partial}{\partial s}V_n([h_{L_0}+tg+s]) &=\int_{\mathbb{S}^{n-1}} \mathrm{d}S([h_{L_0}+tg+s])
\end{align*}

该梯度的秩为 1,且依弱收敛性质 (命题~\ref{prop:surface_weak_converge}),该梯度对 $(s,t)$ 是连续的,这也就说明了映射 $(s,t)\mapsto V_n([h_{L_0}+tg+s])$ 是连续可微的。考虑
\[
\mathcal{L}(t,s):=\int_{\mathbb S^{n-1}}(h_{L_0}+tg+s)\mathrm{d}\mu-\lambda\Big( V_n\big([h_{L_0}+tg+s]\big)-1\Big)
\]
\emph{Lagrange 乘子法}告诉我们
\begin{align*}
& \left\{\begin{aligned} \frac{\partial\mathcal{L}}{\partial t }(0,0)&=0\\ \frac{\partial\mathcal{L}}{\partial s}(0,0)&=0 \end{aligned}\right. \\
\implies & \left\{\begin{aligned} \int_{\mathbb S^{n-1}} g\,\mathrm d\mu&=\lambda\left. \int_{\mathbb S^{n-1}} g(u)\,\mathrm{d}S\big([h_{L_0}+s],u\big)\right|_{s=0}=\lambda\int_{\mathbb S^{n-1}} g(u)\,\mathrm{d}S(L_0,u)\\ \int_{\mathbb S^{n-1}} \mathrm d\mu&=\lambda\left. \int_{\mathbb S^{n-1}}  \mathrm{d}S\big([h_{L_0}+tg],u\big)\right|_{t=0}=\lambda\int_{\mathbb S^{n-1}}\mathrm{d}S(L_0,u)  \end{aligned}\right.
\end{align*}
这里第二步用到了体积的变分公式, $\lambda$ 是 Langrange 乘子。来看下最后得到的两个等式:

\begin{itemize}
\item $\displaystyle\int_{\mathbb S^{n-1}} \mathrm d\mu=\lambda\int_{\mathbb S^{n-1}}\mathrm{d}S(L_0,u)$ 推出 $\lambda= \frac{\int_{\mathbb{S}^{n-1}}\mathrm d\mu}{\int_{\mathbb S^{n-1}}\mathrm{d}S(L_0,u) }=\frac{\int_{\mathbb{S}^{n-1}}\mathrm d\mu}{S(L_0)}$ 代表 $\lambda$ 是一个常数,不依赖于 $g$。(这也是为什么要引入 $s$ 这个"扰动"——不然无法说明此点。)
\item $\displaystyle \int_{\mathbb S^{n-1}} g\,\mathrm d\mu=\lambda\int_{\mathbb S^{n-1}} g(u)\,\mathrm{d}S(L_0,u)$ 是对任意 $g\in\mathcal C(\mathbb S^{n-1})$ 都成立的,根据测度的Riesz 表示定理 (也叫 \emph{Riesz–Markov–Kakutani 表示定理}),有
\[
\mu=\lambda S(L_0,\cdot)
\]
\end{itemize}

综合得到 $\mu=\lambda S(L_0,\cdot)= \frac{\int_{\mathbb{S}^{n-1}}\mathrm d\mu}{S(L_0)}S(L_0,\cdot)=S (K,\;\cdot )$,其中 $K:=\left(\frac{\int_{\mathbb{S}^{n-1}}\mathrm d\mu}{S(L_0)}\right)^{\frac1{n-1}}L_0$。这正是我们想要的结论。
\end{proof}

\begin{remark}
\begin{itemize}
\item 根据定理~\ref{thm:minkowski_unique},我们上述证明得到的 $K$ 在至多相差一个平移的条件下是唯一的。
\item 根据之前已经证明的性质~\ref{prop:surface_great_circle}、\ref{prop:surface_half},定理~\ref{thm:minkowski_exist} 的条件是充要的。
\item 该证明来自 Gardner, R. J., Hug, D., Weil, W., Xing, S., \& Ye, D. (2019). \textit{General volumes in the Orlicz–Brunn–Minkowski theory and a related Minkowski problem I.} {Calculus of Variations and Partial Differential Equations}, 58(1), 12.,其思路在解决Minkowski 问题的相关问题上有一定的通用性。
\end{itemize}
\end{remark}

\subsection{应用}

在本文的最后,我们再简单介绍一下 Minkowski 问题的一些应用,不再细述。

\begin{enumerate}
\item Minkowski 问题可以退化成最优传输中 (弱化的) 的 Monge-Ampere 方程——给定 $f:\mathbb S^{n-1}\to[0,\infty)$,找 $h:\mathbb S^{n-1}\to \mathbb{R}$ 使得 $\mathrm{det}(\nabla ^2h+h\mathbf{I}_{n-1})=f$。
\item Sobolev 或者有界变分函数 $f$ 允许通过 Minkowski 问题变成一个凸体,称之为 ``\emph{凸化} (convexification)",这是一个非常强大的工具。例如,可以通过某些凸体的不等式反推出泛函的不等式,如 Sobolev 不等式和 Affine Sobolev 不等式——\href{https://cims.nyu.edu/~gaoyong/}{张高勇}教授有一篇著名工作 Zhang, G. (1999). The affine Sobolev inequality. \textit{Journal of Differential Geometry}, \textit{53}(1), 183-202。
\item 可以建立 \emph{Blaschke-Santaló 不等式} 的证明 (会在第~\ref{chapter:6} 章中给出, 见定理~\ref{thm:bs})。
\end{enumerate}

% \section*{符号表}

% 本文的符号有点多,整理如下:

% \begin{itemize}
% \item $V_n(K)$:凸体 $K$ 的 $n$ 维体积。
% \item $V_1(K,L)$:凸体 $K,L$ 的第一混合体积。
% \item $h_K$:凸体 $K$ 的支撑函数。
% \item $S(K)$:凸体 $K$ 的表面积。
% \item $ S(K,\cdot)$:凸体 $K$ 的表面积测度。
% \item $\mathrm{int}K$: $K$ 的内点。
% \item $\rho_K$:凸体 $K$ 的半径函数,定义为 $\rho_K(u):=\sup\{\lambda:\lambda u\in K\}$。
% \item $r_K(u):=\rho_K(u)u$; $r_K^{-1}(x):=x/\|x\|$。
% \item $\nu_K$、 $\nu_K^{-1}$:凸体 $K$ 的 Gauss 映射及其逆。
% \item $ \mathcal{C}\big(\mathbb S^{n-1}\big)$:全体 $\mathbb S^{n-1}\to\mathbb{R}$ 的连续函数。
% \item $ \mathcal{C}^+\big(\mathbb S^{n-1}\big)$:全体 $\mathbb S^{n-1}\to (0,\infty)$ 的连续函数。
% \item $[f]$:函数 $f\in\mathcal{C}^+\big(\mathbb S^{n-1}\big)$ 的 Wulff 形。
% \item $B_n$: $\mathbb R^n$ 中的单位球。
% \item $\mathcal H^{n-1}$: $n-1$ 维 Hausdorff 测度。
% \end{itemize}
  

\chapter{Blaschke–Santaló 不等式和仿射等周不等式}\label{chapter:6}

 


\section{Blaschke–Santaló 不等式}
Blaschke–Santaló 不等式也是凸分析里的一个十分重要的不等式,它揭示了凸体和其极集的关系,并且可以结合 Brunn-Minkowski 不等式得到一些非常优美的结论。%首先我们补充几个定义。
\subsection{Santaló  点}
\begin{lemma}[Santaló 点]
对于任意凸体 $K\subset\mathbb R^n$,取
\[
(K-x)^\circ=\{y\in\mathbb R^n:\langle y,z-x\rangle\le 1,\;\forall z\in K\}
\]
那么对 $x\in\mathrm{int}K$,存在唯一的点 $x^*$最小化函数 $g(x):= V_n((K-x)^\circ)$,且
\begin{equation}
\int_{\mathbb{S}^{n-1}}\Big(h_K(u)-\langle x^*,u\rangle\Big)^{-(n+1)}u\,\mathrm{d}u=0 \label{eq:6-1}\tag{6-1}
\end{equation}
称点 $x^*$为 \emph{Santaló 点}。特别地, $x^*=0$当且仅当 $K$的极集 $K^\circ$重心在原点。
\end{lemma}

\begin{proof}
由于对 $x\in\mathrm{int}K$有 $h_{K-x}(u)=h_K(u)-\langle x,u\rangle$,用式 ~\eqref{eq:4-2}和引理~\ref{lem:support_radial},
\begin{align*}
g(x)&=V_n((K-x)^\circ)=\frac1n\int_{\mathbb{S}^{n-1}}\rho_{(K-x)^\circ}^n(u)\mathrm{d}u\\
&=\frac1n\int_{\mathbb{S}^{n-1}}h_{K-x}^{-n}(u)\mathrm{d}u=\int_{\mathbb{S}^{n-1}}\big(h_K(u)-\langle x,u\rangle\big)^{-n}\mathrm{d}u
\end{align*}
注意到
\begin{enumerate}
\item $g(x)$这对 $x$是严格凸的 (因为 $t\mapsto t^{-n}$严格凸);
\item 当 $x$趋近于 $K$的边界 $\partial K$时,存在 $u$使得 $h_K(u)-\langle x,u\rangle\to 0^+$,故 $g(x)\to\infty$。
\end{enumerate}
所以必存在一个唯一的 $x^*\in\mathrm{int}K$最小化 $g(x)$。由梯度 $\nabla g(x^*)=0$即可得到式 \eqref{eq:6-1}。

特别地,在 \eqref{eq:6-1}中取 $x^*=0$,有
\begin{align*}
K^\circ\text{ 的重心 }&=\int_{K^\circ}x\,\mathrm{d}x\\
&=\int_{\mathbb S^{n-1}}\int_0^{\rho_{K^\circ}(u)}ru\cdot r^{n-1}\,\mathrm{d}r\mathrm{d}u\\
&=\frac1{n+1}\int_{\mathbb S^{n-1}}\rho_{K^\circ}^{n+1}(u)u\,\mathrm du\\
&=\frac1{n+1}\int_{\mathbb S^{n-1}}h^{-(n+1)}_{K^\circ}(u)u\,\mathrm du\\
&=0 \tag{由式~\eqref{eq:6-1}}
\end{align*}
\end{proof}

\begin{remark}
\begin{itemize}
\item 注意式~\eqref{eq:6-1}的被积分项是向量值函数,右侧的 0 也代表全 0 的 $n$维向量。
\item 该引理参考了 Böröczky, K. J., Figalli, A., \& Ramos, J. P. (2025). \textit{Isoperimetric inequalities, Brunn-Minkowski theory and Minkowski type Monge-Ampère equations on the sphere}.的 Lemma 6.5.1 (p. 211)。
\end{itemize}
\end{remark}

\subsection{BS 不等式的证明}

\begin{theorem}[Blaschke–Santaló 不等式]\label{thm:bs}
设 $K\subset\mathbb R^n$是一个凸体,且其 Santaló 点 在原点,即
\begin{equation}
\int_{\mathbb{S}^{n-1}}h_K^{-(n+1)}(u)u\, \mathrm{d}u=0 \label{eq:6-2}\tag{6-2}
\end{equation}
那么成立不等式
\begin{equation}
V_n(K)V_n(K^\circ)\le\big(V_n(B_n)\big)^2 \label{eq:6-3}\tag{6-3}
\end{equation}
取等当且仅当 $K$是一个中心在原点的椭球。
\end{theorem}

\begin{proof}
由于 $K$有界且 $0\in\mathrm{int}K$,支撑函数 $0<h_K<\infty$恒成立,我们定义一个新的测度
\[
\mathrm{d}\mu:=h_K^{-(n+1)}(u)\,\mathrm{d}u
\]
这是一个球面 $\mathbb {S}^{n-1}$上的、非零、有限的 Borel 测度,而且根据式 \eqref{eq:6-2}有 $\mu$的重心在原点:
\[
\int_{\mathbb{S}^{n-1}}u\,\mathrm{d}\mu(u)=0
\]
同时,由于 $h_K$在球面上恒大于 0,所以 $\mu$肯定不集中在任意一个大圆上。至此, $\mu$满足了 Minkowski 问题的全部条件 (定理~\ref{thm:minkowski_exist}),可以找到凸体 $\Lambda K$使得
\begin{equation}
\mathrm{d}\mu(u)=h_K^{-(n+1)}(u)\,\mathrm{d}u=\mathrm{d}S(\Lambda K,u) \label{eq:6-4}\tag{6-4}
\end{equation}
有的文献中称 $\Lambda K$为"曲率体 (curvature body)",因为上式同时表明 $\Lambda K$的曲率函数满足,$f_{\Lambda K}:=\frac{\mathrm{d}S(\Lambda K,u)}{\mathrm du}=h_K^{-(n+1)}$。

$\Lambda K$在至多相差一个平移的意义下是唯一的,且还有
\begin{align*}
&f_{\Lambda K}(u)=h_K^{-(n+1)}(u)\\
\implies \;&V_n(K^\circ)=\frac1n\int_{\mathbb{S}^{n-1}}\rho_{K^\circ}^n(u)\,\mathrm{d}u=\frac1n\int_{\mathbb S^{n-1}}h_K^{-n}(u)\,\mathrm du= \frac1n\int_{\mathbb S^{n-1}}f_{\Lambda K}^{\frac n{n+1}}(u)\,\mathrm du \label{eq:6-5}\tag{6-5}\\
\implies\;& nV_n(K^\circ)=\int_{\mathbb S^{n-1}}f_{\Lambda K}^{\frac n{n+1}}(u)\,\mathrm du=:as(\Lambda K) \label{eq:6-6}\tag{6-6}
\end{align*}
这里补充一下定义,

\begin{definition}[曲率函数]
设 $K\subset \mathbb R^n$是凸体,且其表面积测度相对于球面测度绝对连续,即 $\mathrm{d}S(K,u)\ll \mathrm du$,那么称二者的 Radon-Nikodym 导数为 $K$的\emph{曲率函数} (curvature function),记作 $f_K$。
\end{definition}

\begin{definition}[仿射表面积]
设 $K\subset \mathbb R^n$是凸体且存在曲率函数 $f_K$,那么定义 $K$的\emph{仿射表面积} (affine surface area) 为
\[
as(K):=\int_{\mathbb S^{n-1}}f_K^{\frac n{n+1}}(u)\,\mathrm du
\]
\end{definition}

继续定理~\ref{thm:bs}的证明,回顾一下式~\eqref{eq:6-5},还能发现
\begin{align*}
V_n(K^\circ)&=\frac1n\int_{\mathbb S^{n-1}}h_K^{-n}(u)\,\mathrm du \tag{由式~\eqref{eq:6-5}}\\
&=\frac1n\int_{\mathbb S^{n-1}}h_K(u) h_K^{-(n+1)}(u)\,\mathrm du \\
&=\frac1n\int_{\mathbb S^{n-1}}h_K(u)\,\mathrm{d}S(\Lambda K,u) \tag{由式~\eqref{eq:6-4}}\\
&=V_1(\Lambda K, K) \tag{定理~\ref{thm:var}}\\
&\ge V_n(\Lambda K)^{\frac{n-1}n}V_n(K)^{\frac1n} \tag{定理~\ref{thm:minkowski}}
\end{align*}
即
\begin{align*}
&V_n(K^\circ)^n\ge V_n(\Lambda K)^{n-1}V_n(K)\\
\implies\;&V_n(K^\circ)^{n+1}\ge V_n(\Lambda K)^{n-1}V_n(K)V_n(K^\circ)\\
\implies\;&\left(\frac{as(\Lambda K)}{n}\right)^{n+1}\ge V_n(\Lambda K)^{n-1}V_n(K)V_n(K^\circ) \tag{由式~\eqref{eq:6-6}}
\end{align*}
我们发现自己找到了欲证目标 \eqref{eq:6-3}的一个上界,
\begin{equation}
\frac{as(\Lambda K)^{n+1}}{n^{n+1}V_n(\Lambda K)^{n-1}}\ge V_n(K)V_n(K^\circ) \label{eq:6-7}\tag{6-7}
\end{equation}
如果能再证明这个上界不超过 $V_n(B_n)^2$的话,就证明出来了 BS 不等式。

实际上可以证明(会在后文中介绍),
\begin{equation}
\frac{as(\Lambda K)^{n+1}}{n^{n+1}V_n(\Lambda K)^{n-1}}\le \frac{as(B_n)^{n+1}}{n^{n+1}V_n(B_n)^{n-1}} \label{eq:6-8}\tag{6-8}
\end{equation}
等号成立当且仅当 $\Lambda K$是椭球;而这正是我们想要的上界:
\begin{align*}
\frac{as(B_n)^{n+1}}{n^{n+1}V_n(B_n)^{n-1}}&=\frac{\displaystyle\left(\int_{\mathbb{S}^{n-1}}f_{B_n}^{\frac n{n+1}}(u)\,\mathrm{d}u\right)^{n+1}}{n^{n+1}V_n(B_2)^{n-1}}\\
&=\frac{\displaystyle \left(\int_{\mathbb{S}^{n-1}}\mathrm{d}u\right)^{n+1}}{n^{n+1}V_n(B_n)^{n-1}} \tag{$f_{B_n}\equiv 1$}\\
&=\frac{S(\mathbb S^{n-1})^{n+1}}{n^{n+1}V_n(B_n)^{n-1}}\\
&=\frac{(nV_n(B_n))^{n+1}}{n^{n+1}V_n(B_n)^{n-1}}\\
&=V_n(B_n)^2 \label{eq:6-9}\tag{6-9}
\end{align*}
结合 \eqref{eq:6-7}、 \eqref{eq:6-8}、 \eqref{eq:6-9}立即得到 BS 不等式 $V_n(K)V_n(K^\circ)\le V_n(B_n)^2$。

下面简单叙述一下取等条件,显见对任意的非奇异线性变换 $T$, 
\begin{align*}
(TK)^\circ&=\big\{y\in\mathbb{R}^n:\langle y,Tx\rangle\le 1,\,\forall x\in K\big\}\\
&=\big\{y\in\mathbb{R}^n:\big\langle T^\top y,x\big\rangle\le 1,\,\forall x\in K\big\}\\
&=\big(T^\top\big)^{-1}\big\{z\in\mathbb{R}^n:\langle z,x\rangle\le 1,\,\forall x\in K\big\} \tag{令 $z:=T^{\top}y$}\\
&=\big(T^\top\big)^{-1}K^\circ
\end{align*}
故
\begin{align*}
V_n(TK)V_n((TK)^\circ)&=V_n(TK)V_n\left(\big(T\big)^{-1}K^\circ\right)\\
&=\mathrm{det}(T)V_n(K)\cdot\frac1{\det(T)}V_n(K)\\
&=V_n(K)V_n(K^\circ)
\end{align*}
即, $V_n(K)V_n(K^\circ)$作为凸体 $K$的泛函在 $T$作用下是不变的。又显然在 $K=B_n$时 BS 不等式取等,根据椭球的定义,对任意中心在原点的椭球都有 BS 不等式取等。

另一方面,若 BS 不等式取等, \eqref{eq:6-8}就应取等, $\Lambda K$是一个椭球;又由于证明过程中用了 Minkowski 第一不等式 (定理~\ref{thm:minkowski}),其取等条件是 $K$和 $\Lambda K$位似——所以 $K$也得是一个椭球。根据引理 1, $K^\circ$重心在原点,自然 $K$也应该是一个中心在原点的椭球。
\end{proof}

\begin{remark}
\begin{itemize}
\item 回忆一下,椭球的定义是:设 $B_n(x,r)$是一个球,则对其进行非奇异线性变换后 $TB_n(x,r)$就是椭球 (见定义~\ref{def:ball});因而这里所说的``椭球"本身都是包含球的。
\item $V_n(K)V_n(K^\circ)$有时也称作\emph{体积积} (volume product)。
\item BS 不等式最早由 W. Blaschke 于 1920s 给出了 $n=2,3$情况下的证明,L. A. Santaló 则在 1949 年给出了一般维度的证明\footnotemark,思路和前面给出的证明基本相同;这个证明本质上是融合了两个重要结论:Minkowski 第一不等式,以及式 \eqref{eq:6-8}——它叫做\emph{仿射等周不等式} (affine isoperimetric inequality)。
\item 早期,仿射表面积 $as(K)$的概念来自于微分几何,是只定义在足够光滑的凸体上的。尽管,我们直觉上可以用光滑凸体来逼近非光滑凸体,从而表明与仿射表面积相关的不等式在一般凸体下也成立,但这样的话等号成立条件可能就失效了。鉴于上述原因,对仿射等周不等式和  BS 不等式的取等条件讨论直至 1985 年才由 C. M. Petty 给出\footnotemark。
\end{itemize}
\end{remark}
% \footnotetext{}
\footnotetext{Santaló, L. A. (1949). \textit{Un invariante afin para los cuerpos convexos del espacio de n dimensiones}. Portugaliae mathematica, 8, 155-161.}
\footnotetext{Petty, C. M. (1985). \textit{Affine isoperimetric problems}. Annals of the New York Academy of Sciences, 440(1), 113-127.}
\subsection{另一种证明}

下面考虑一种特殊情况: $K$是关于原点\emph{中心对称} (origin-symmetric) 的凸体,即 $K=-K$,此时可以利用 BM 不等式 + Steiner 对称化比较容易地给出 BS 不等式的证明。

回忆一下用 Steiner 对称化证明不等式的一般步骤(见第~\ref{chapter:3} 章),我们得先找到一个在 Hausdorff 度量下连续的泛函,让它在 Steiner 对称化操作下满足单调性。在这里,需要的是
\begin{equation}
V_n(K^\circ)\le V_n((S_uK)^\circ) \label{eq:6-10}\tag{6-10}
\end{equation}
只要上式成立,就可以根据 Gross 定理 (定理~\ref{thm:gross}),设凸体 $K$在一列单位方向$\{u_j\}_{j=1}^\infty$下依 Hausdorff 度量收敛至球 $\left(\frac{V_n(K)}{V_n(B_n)}\right)^{\frac1n}B_n$,并记 $K_j:=S_{u_j}S_{u_{j-1}}\cdots S_{u_2}S_{u_1}K$,则
\begin{align*}
V_n(K)V_n(K^\circ) \le V_n(K_1)V_n(K_1^\circ) \le V_n(K_2)V_n(K_2^\circ)\le \cdots\le V_n(K_j)V_n(K_j^\circ)\le\cdots
\end{align*}
取极限,
\begin{align*}
V_n(K)V_n(K^\circ)&\le \lim_{j\to\infty}V_n(K_j)V_n(K_j^\circ)\\
&= V_n(K)\lim_{j\to\infty}V_n(K_j^\circ) \tag{Steiner 对称化不改变体积}\\
&=V_n(K)V_n\left(\lim_{j\to\infty}K_j^\circ\right) \tag{体积对 Hausdorff 度量连续}\\
&=V_n(K)V_n\left(\left(\lim_{j\to\infty}K_j\right)^\circ\right)\\
&=V_n(K)V_n\left(\left(\frac{V_n(K)}{V_n(B_n)}\right)^{-\frac1n}B_n\right)\\
&=V_n(B_n)^2
\end{align*}
这就得到了 BS 不等式。上面第四行来自于这样的事实:

\begin{lemma}
设 $K_0,K_1,K_2,\dots\subset\mathbb{R}^n$是凸体,且在 Hausdorff 度量下 $K_j\to K_0$;如果 $0\in\mathrm{int}K_j $对任意 $j=1,2,\ldots$和 $j=0$都成立,则依 Hausdorff 度量收敛可以与取极集交换,即
\[
\lim_{j\to\infty}K_j^\circ=\left(\lim_{j\to\infty}K\right)^\circ=K_0^\circ
\]
\end{lemma}

\begin{proof}
$0\in\mathrm{int}K_j $保证了半径函数 $0<\rho_{K_j}<\infty$,且 $K_j\to K_0$意味着 $\rho_{K_j}\to\rho_{K_0}$一致收敛(因为 Hausdorff 度量的定义蕴含 $|\rho_{K_j}(u)-\rho_{K_0}(u)|\le d_H(K_j,K_0),\;\forall u\in\mathbb{S}^{n-1}$)。

设 $r_0:=\min\limits_{u\in\mathbb{S}^{n-1}}\rho_{K_0}(u)>0$,则对任意取定的 $0<\epsilon<1$和充分大的 $j$有 $ \rho_{K_j}\ge (1-\epsilon)r_0$。故
\begin{align*}
d_H(K_j^\circ,K_0^\circ)&=\sup\limits_{u\in\mathbb S^{n-1}}\big|h_{K_j^\circ}(u)-h_{K_0^\circ}(u)\big|\\
&=\sup\limits_{u\in\mathbb S^{n-1}}\left|\frac1{\rho_{K_j}(u)}-\frac1{\rho_{K_0}(u)}\right|\\
&\le\frac1{(r_0(1-\epsilon))^2}\sup_{u\in\mathbb S^{n-1}}\big|\rho_{K_j}(u)-\rho_{K_0}(u)\big|\\
&\to 0
\end{align*}
\end{proof}

可以看出,只要证明了 \eqref{eq:6-10},就可以立即得到 BS 不等式成立。再写作引理

\begin{lemma}\label{lem:ball}
设 $K\subset\mathbb R^n$是凸体且关于原点中心对称,那么对任意固定的 $u\in\mathbb S^{n-1}$有 $V_n(K^\circ)\le V_n((S_uK)^\circ)$,取等当且仅当每个平行于 $u^\perp$的超平面截 $K^\circ$得到的集合都是中心对称的。
\footnotemark
\end{lemma}
\footnotetext{Keith M. Ball, 1986}

\begin{proof}
由于 $u$具体方向不重要,不失一般性设 $u=e_n$是第 $n$个坐标向量,则 $u^\perp=\mathbb R^{n-1}$。于是我们可以按前 $n-1$维+第 $n$维来分解 $S_uK$(回忆一下式~\eqref{eq:3-3}):
\begin{align*}
S_uK&=\left\{(x,t):x\in\mathbb R^{n-1},|t|\le\frac{f(x)-g(x)}{2}\right\}\\
&=\left\{\left(x,\frac{t-s}{2}\right):(x,s)\in K, (x,t)\in K\right\}
\end{align*}
这里 $(x,t)$代表一个 $n$维向量, $x\in\mathbb R^{n-1}$、 $t\in\mathbb R$。

则 Steiner 对称化的极集为
\begin{align*}
(S_u K)^\circ&=\left\{(y,r)\in\mathbb R^n:\left\langle(y,r),\left(x,\frac{t-s}{2}\right)\right\rangle\le1,\;\forall\, (x,t)\in K,(x,s)\in K\right\}\\
&=\left\{(y,r)\in\mathbb R^n:\langle y,x\rangle+\frac{t-s}{2}r\le 1,\;\forall\, (x,t)\in K,(x,s)\in K\right\}
\end{align*}
对任意凸体 $L$,定义它在 $u=e_n$方向的截面为 $L_\tau:=\{z\in\mathbb{R}^{n-1}:(z,\tau)\in L\}$(可以视作用与 $u^\perp$平行的超平面截 $L$所获得的一系列切片)。那么对任意的 $r,s>0$,取定 $x\in K^\circ_r$、 $y\in K^\circ_{-s}$,则对任意 $\tau\in\mathbb R^n$和 $z\in K_\tau$有
\begin{align*}
(x,r)\in K^\circ&\implies \langle x,z\rangle+r\tau\le 1\\
(y,-s)\in K^\circ&\implies \langle y,z\rangle-s\tau\le 1
\end{align*}

若令 $w:=\frac{s}{r+s}x+\frac{r}{r+s}y$ (见图~\ref{fig:BS_proof}),那么上面两个关系可以推出:对 $(z,\tau_1)\in K$和 $(z,\tau_2)\in K$有
\begin{align*}
&\langle w,z\rangle+\frac{rs}{r+s}(\tau_1-\tau_2)\\
&=\frac s{r+s}\langle x,z\rangle+\frac r{r+s}\langle y,z\rangle+\frac{s}{r+s}r\tau_1-\frac{r}{r+s}s\tau_2\\
&=\frac{s}{r+s}\Big(\underbrace{\langle x,z\rangle+r\tau_1}_{\le 1}\Big)+\frac{r}{r+s}\Big(\underbrace{\langle x,z\rangle-s\tau_2}_{\le 1}\Big)\\
&\le \frac s{r+s}+\frac r{s+r}=1
\end{align*}
即 $\displaystyle\left\langle\left(w,\frac{2rs}{r+s}\right),\left(z,\frac{\tau_1-\tau_2}{2}\right)\right\rangle\le 1$;又由于 $(z,\tau_1)\in K$、 $(z,\tau_2)\in K$,有 $\displaystyle \left(z,\frac{\tau_1-\tau_2}{2}\right)\in S_uK$,故 $\displaystyle\left(w,\frac{2rs}{r+s}\right)\in (S_uK)^\circ$,或者说 $w\in (S_uK)^{\circ}_{\frac{2rs}{r+s}}$。

\begin{figure}[htbp]
\centering
\includegraphics[width=0.7\textwidth]{figures/BS_proof.pdf}
\caption{引理~\ref{lem:ball} 证明的示意图:在$K^\circ$ 的两个截面$K^\circ_{-s}$ 和 $K^\circ_{t}$ 上分别选取点$x$ 和 $y$;$w$ 为二者的插值。}
\label{fig:BS_proof}
\end{figure}

我们上面的分析揭示了这样的一个关系:
\[
\frac s{r+s}K^\circ_r +\frac r{r+s}K^\circ_{-s}\subseteq (S_uK)^\circ _{\frac{2rs}{r+s}}
\]
用 BM 不等式,有
\begin{align*}
V_{n-1}\left( (S_uK)^\circ _{\frac{2rs}{r+s}}\right)&\ge V_{n-1}\left(\frac s{r+s}K^\circ_r +\frac r{r+s}K^\circ_{-s}\right)\\
&\ge V_{n-1}(K_r^\circ)^{\frac s{r+s}}V_{n-1}(K^\circ_s)^{\frac r{r+s}} \tag{BM 不等式}
\end{align*}
上述分析对一般的凸体 $K$都成立,不过要是继续分析下去就会变得非常麻烦,故此我们只考虑更加简单的情形: $K$关于原点中心对称。这个条件的好处在于:由于 $K=-K$,有 $K^\circ=-K^\circ$,即 $K^\circ$也是关于原点中心对称的,可以直接在上式取 $s=r$,即
\begin{align*}
V_{n-1}((S_u K)^\circ_r)&\ge V_{n-1}(K^\circ_r)^{\frac12}V_{n-1}(K^\circ_{-r})^{\frac12}\\
&=V_{n-1}(K_r^\circ) \tag{由中心对称}
\end{align*}
直接对 $r$积分就得到了想要的结论:
\[
V_n((S_uK)^\circ)=2\int_0^\infty V_{n-1}((S_uK)^\circ_r)\,\mathrm dr\ge 2\int_0^\infty V_{n-1}(K^\circ_r)\,\mathrm dr=V_n(K^\circ)
\]
下面简单讨论一下取等条件,要使得等号成立,BM 不等式那一步就应取等,即 $K_r^\circ$和 $K_{-r}^\circ=-K_r^\circ$位似——二者只能相差一个平移,所以 $K^\circ_r$一定存在一个对称中心,这对所有 $r>0$都成立。
\end{proof}

\begin{remark}
\begin{itemize}
\item 上面的引理也可以参考 Böröczky, K. J., Figalli, A., \& Ramos, J. P. (2025). \textit{Isoperimetric inequalities, Brunn-Minkowski theory and Minkowski type Monge-Ampère equations on the sphere}.的 Proposition 6.5.3 (pp. 212-213)。
\item 一个一般的凸体 $C\subseteq\mathbb R^n$是中心对称的,当且仅当存在 $x_0\in C$使得 $C-x_0$是关于原点中心对称的,即 $(C-x_0)=-(C-x_0)$。
\item 有这样的结论 
\begin{theorem*}[Brunn]
给定 $2<m<n$, $K\subset\mathbb R^n$是凸体,若任意一个 $m$维超平面与 $K$的交集都是中心对称的 (可能为空集),那么 $K$一定是一个椭球。
\end{theorem*}
这个定理结合引理~\ref{lem:ball}表明,若 $V_n(K^\circ)= V_n((S_uK)^\circ)$对任意 $u\in\mathbb S^{n-1}$成立,则任意方向和位置的 $n-1$维超平面截 $K^\circ$得到的集合都是中心对称的,所以 $K^\circ$只能是椭球,故而 $K$也只能是椭球。这再次刻画了 BS 不等式的取等条件。
\item 引理~\ref{lem:ball}的思路——对比 $K^\circ$的截面和 $(S_uK)^\circ$的截面,其实可以推广到一般的、不一定中心对称的凸体,这个思路是行得通的,但是分析很复杂,超出了本文的范围。读者可以参阅 Meyer, M., \& Pajor, A. (1990). \textit{On the Blaschke-Santaló inequality}.  Archiv der Mathematik, 55(1), 82-93. 一文的 Lemma 7。
\end{itemize}
\end{remark}

\section{仿射等周不等式}

前文的式~\eqref{eq:6-8}挖了一个大坑,最后还是要填一下。我们需证明下面这样的结论:

\begin{theorem}[仿射等周不等式]\label{thm:as_isoperimetric}
设 $K\subset\mathbb R^n$是凸体,那么
\[
\left(\frac{as(K)}{as(B_n)}\right)^{\frac{n+1}{n-1}}\le \frac{V_n(K)}{V_n(B_n)}
\]
取等当且仅当 $K$是一个椭球。
\end{theorem}

\begin{remark}
根据式 \eqref{eq:6-9},仿射等周不等式的形式也可以改写作
\[
as(K)\le nV_n(B_n)^{\frac2{n+1}}V_n(K)^{\frac{n-1}{n+1}}
\]
\end{remark}

\subsection{仿射表面积}

先重新审视一下仿射表面积的定义。之前的定义是这样的:若凸体 $K\subset\mathbb R^n$存在曲率函数 $f_K\ge 0$使得 $\mathrm{d}S(K,u)=f_K(u)\mathrm{d}u$,那么
\[
as(K):=\int_{\mathbb S^{n-1}}f_K^{\frac n{n+1}}(u)\,\mathrm du
\]
就定义为 $K$的仿射表面积。早期这个概念是来自于微分几何,因此限制了 $K$需满足一定的光滑性。这里的光滑性一般指的是
\begin{itemize}
\item $K$的边界曲面 $\partial K$是一个 $\mathbb {R}^n$中的正则子流形,且二次可微;
\item Gauss 映射 $\nu_K:\partial K\to \mathbb{S}^{n-1}$是一个微分同胚。
\end{itemize}
满足上述条件的凸体集合一般会记为 $C_+^2$。我们不深入探讨这些,简单起见,后文中我们都假设 $K$的表面足够光滑,使得曲率函数 $f_K$存在且非负,并且 $f_K$是连续的。

仿射表面积有其它等价定义,这里也非常简单地介绍一下。

\textbf{仿射表面积等价定义 1}: 设 $L\subset\mathbb R^n$是任意一个星形体, $0\in\mathrm{int}(L)$,则其半径函数 $\rho_L$恒正且连续,有

\begin{align*}
as(K)&=\int_{\mathbb S^{n-1}}f_K^{\frac n{n+1}}\mathrm du\\
&=\int_{\mathbb S^{n-1}}f_K^{\frac n{n+1}}\rho_L^{-\frac n{n+1}}\rho_L^{\frac n{n+1}}\mathrm du\\
&\le \left(\int_{\mathbb S^{n-1}}\left(f_K^{\frac n{n+1}}\rho_L^{-\frac n{n+1}}\right)^{\frac {n+1}n}\mathrm du\right)^{\frac n{n+1}}\left(\int_{\mathbb S^{n-1}}\left(\rho_L^{\frac n{n+1}}\right)^{n+1}\mathrm du\right)^{\frac 1{n+1}} \tag{Hölder 不等式}\\
&=\bigg(\int_{\mathbb S^{n-1}}\rho_L^{-1}\,\underbrace{\vphantom{\int}f_K\mathrm du}_{S(K,u)}\bigg)^{\frac n{n+1}}\bigg(\underbrace{\int_{\mathbb S^{n-1}}\rho_L^{n}(u)\,\mathrm du}_{nV_n(L)}\bigg)^{\frac1{n+1}}\\
&= \bigg(\underbrace{\int_{\mathbb S^{n-1}}\rho_L^{-1}\mathrm dS(K,u)}_{=:\,nV_1(K,L^\circ)}\bigg)^{\frac n{n+1}}\big(nV_n(L)\big)^{\frac1{n+1}}\\
&=nV_1(K,L^\circ)^{\frac n{n+1}}V_n(L)^{\frac1{n+1}}  
\end{align*}
这里形式上定义 $\displaystyle V_1(K,L^\circ):=\int_{\mathbb S^{n-1}}\rho_L^{-1}(u)\,\mathrm{d}S(K,u)$(这避免了定义 $L^\circ$)。
上式对所有星形体 $L$都成立,且 $f_K$连续时,不等式取等当且仅当 $f_K=\rho_L^{n+1}$。

不难注意到 $nV_1(K,L^\circ)^{\frac n{n+1}}V_n(L)^{\frac1{n+1}}$在 $L$的缩放变换下是不变的(幂次正好消掉),于是,当 $f_K$连续时,可以将 $as(K)$写作如下极值问题的解\footnote{E. Lutwak, 1991}:
\begin{align*}
\left(\frac1nas(K)\right)^{\frac {n+1}{n}}&=\inf\Big\{V_1(K,L^\circ)V_n(L)^{\frac1n}:\; 0\in\mathrm{int}L,\;L\text{ 是星形体 }\Big\} \\
&=\inf\Big\{V_1(K,L^\circ):V_n(L)=1,\; 0\in\mathrm{int}L,\;L\text{ 是星形体 }\Big\} \label{eq:6-11}\tag{6-11}
\end{align*}
(这个式子在第~\ref{chapter:5} 章的例~\ref{ex:petty_body} 中就引入过,这里算是回收了伏笔 :-)


\textbf{仿射表面积等价定义 2}: 由于之前的定义对 $K$有光滑性限制,人们后来得出了一个推广的仿射表面积定义,是这样的:对一般的凸体 $K\subset\mathbb R^n$,
\[
as(K):=\int_{\partial K}\kappa(x)^{\frac1{n+1}}\mathrm{d}\mathcal{H}^{n-1}(x)
\]
这里的 $\kappa(x)$是\emph{Gauss 曲率} (Gauss curvature),
\begin{itemize}
\item 由于 $K$的表面 (某种程度上) 是几乎处处二次可微的\footnote{Alexandrov 定理}, $\kappa(x)$在 $\partial K$上也几乎处处有定义。
\item 当 $K$表面光滑,使得 $f_K$存在且恒正时,有 $\displaystyle\kappa(x)=\frac1{f_K(\nu_K(x))}$。可以验证该定义与之前给出的定义相容:
\begin{align*}
\int_{\partial K}  f_K(\nu_K(x))^{-\frac1{n+1}}\mathrm{d}\mathcal{H}^{n-1}(x)&=\int_{\mathbb S^{n-1}}f_K(u)^{-\frac1{n+1}}\underbrace{\vphantom{\int}\mathrm{d}S(K,u)}_{f_K(u)\mathrm{d}u} \tag{换元 $u=\nu_K(x)$}\\
&=\int_{\mathbb{S}^{n-1}}f_K(u)^{\frac n{n+1}}(u)\,\mathrm du
\end{align*}
\end{itemize}

最后我们研究 $as(K)$在 $K$的变换下相应的结论。显然, $as(K)$在平移变换下是不变的,因为 $S(K,\cdot)$平移不变,自然 $f_K$和 $as(K)$也应如此。此外,我们后文中还会用到一个结论,

\begin{lemma}\label{lem:as_scaling}
设 $\lambda>0$,则 $as(\lambda K)=\lambda^{\frac{n(n+1)}{n+1}}as(K)$。
\end{lemma}

\begin{proof}
最简单的方法是用 \eqref{eq:6-11}。由 $S(\lambda K,\cdot)=\lambda^{n-1}S(K,\cdot)$ (命题~\ref{prop:surface_scaling}),故
\[
V_1(\lambda K,L^\circ):=\int_{\mathbb S^{n-1}}\rho_L^{-1}(u)\mathrm dS(\lambda K,u)=\lambda^{n-1}\int_{\mathbb S^{n-1}}\rho_L^{-1}(u)\mathrm dS(K,u)
\]
因此,根据 \eqref{eq:6-11},
\begin{align*}
\left(\frac1nas(\lambda K)\right)^{\frac {n+1}{n}}&=\lambda^{n-1}\left(\frac1nas(K)\right)^{\frac {n+1}{n}}\\
\implies as(\lambda K)&=\lambda^{\frac{n(n-1)}{n+1}}as(K)
\end{align*}
\end{proof}

\begin{remark}
    为什么这个名字叫``仿射"表面积呢?因为实际上有一个更一般的仿射不变性的结论:设 $T:\mathbb R^n\to \mathbb{R}^n$是一个仿射变换,那么 $as(TK)=\mathrm{det}(T)^{\frac{n-1}{n+1}}as(K)$。特别地,对任意的 $T\in\mathrm{SL}(n)$,都有 $as(TK)=as(K)$。读者可以尝试自证。
\end{remark}

\subsection{仿射等周不等式的证明}

主要思路就是用 Steiner 对称化,分三步走。

\textbf{第一步},先证明仿射表面积在 Steiner 对称化下是单调的,

\begin{lemma}\label{lem:as_monotone}
设凸体 $K\subset \mathbb{R}^n$,那么 $as(K)\le as(S_nK)$对任意 $u\in\mathbb{S}^{n-1}$成立。
\end{lemma}

\begin{proof}
回忆一下,在 Steiner 对称化中,我们会将 $K$投影至 $u^\perp$,得到 $P_uK$;随之,对任意 $x\in P_uK$,作 $u$方向过 $x$的直线,与 $K$交于两点;最后平移两点之间的线段使得 $x$位于其中点,达到对称化的效果。不失一般性,设 $u=e_n$是第 $n$个坐标向量,则 $u^\perp=\mathbb{R}^{n-1}$, 见图~\ref{fig:steiner_2}。则 (式~\eqref{eq:3-3}):
\begin{align*}
K&=\big\{(x,t):g(x)\le t\le f(x),x\in \mathbb{R}^{n-1}\big\}\\
S_uK&=\left\{(x,t):|t|\le \frac{f(x)-g(x)}2,x\in \mathbb{R}^{n-1}\right\}
\end{align*}

\begin{figure}[htbp]
\centering
\includegraphics[width=0.5\textwidth]{figures/steiner_2.png}
\caption{Steiner 对称化的示意图。}\label{fig:steiner_2}
\end{figure}

可以看出, $(x,f(x))\in\partial K$、 $(x,g(x))\in\partial K$,且 $f$是凹函数、 $g$是凸函数,因而几乎处处二次可微,我们可以认为二者包含了 $K$的边界信息。例如,将\emph{上图} (upper graph) $\{(x,f(x)):x\in\mathrm{int}(P_uK)\}$视作曲面,不难得出
\begin{itemize}
\item $(x,f(x))$处的法向量为 $\displaystyle\frac{(-\nabla f(x),1)}{\sqrt{1+\|\nabla f\|^2}}$;
\item $(x,f(x))$处的 Gauss 曲率为 $\kappa (z)=\frac{\mathrm{det}(-\nabla^2 f)}{\left(\sqrt{1+\|\nabla f\|^2}\right)^{n+1}}$\footnote{读者可以算下第一和第二基本型矩阵的行列式,从而得到上式;注意凸体的 Gauss 曲率一定是非负的。};
\item 以及,该曲面上的 Hausdorff 测度满足 $\mathrm{d}\mathcal{H}^{n-1}(z)=\sqrt{1+\|\nabla f(x)\|^2}\,\mathrm{d}x$\footnote{这个是曲面的第一基本型矩阵的行列式决定的。}
\end{itemize}

好了,把这几个公式用起来,得到
\begin{align*}
\mathrm{as}(K)&=\int_{\partial K}\kappa(z)^{\frac1{n+1}}\mathrm{d}\mathcal H^{n-1}(z)\\
&=\int_{\{(x,f(x)):x\in\mathrm{int}(P_uK)\}}\kappa(z)^{\frac1{n+1}}\mathrm{d}\mathcal H^{n-1}(z)+\int_{\{(x,g(x)):x\in\mathrm{int}(P_uK)\}}\kappa(z)^{\frac1{n+1}}\mathrm{d}\mathcal H^{n-1}(z)\\
&=\int_{\mathrm{int}(P_uK)}\left(\frac{\mathrm{det}(-\nabla^2f(x))}{(\sqrt{1+\|\nabla f(x)\|^2})^{n+1}}\right)^{\frac1{n+1}}\sqrt{1+\|\nabla f(x)\|^2}\mathrm{d}x\\
&\quad + \int_{\mathrm{int}(P_uK)}\left(\frac{\mathrm{det}(\nabla^2g(x))}{(\sqrt{1+\|\nabla g(x)\|^2})^{n+1}}\right)^{\frac1{n+1}}\sqrt{1+\|\nabla g(x)\|^2}\mathrm{d}x\\
&=\int_{\mathrm{int}(P_uK)}\left(\mathrm{det}(-\nabla^2f(x))^{\frac1{n+1}}+\mathrm{det}(\nabla^2g(x))^{\frac1{n+1}}\right)\mathrm{d}x \label{eq:6-12}\tag{6-12}
\end{align*}
完全类似地,用 $\frac{f-g}2$替换上式的 $f$、 $-\frac{f-g}2$替换上式的 $g$,有
\begin{align*}
as(S_uK)&=\int_{\mathrm{int}(P_uK)}2\mathrm{det}\left(\frac{\nabla^2g(x)-\nabla^2f(x)}{2}\right) ^{\frac1{n+1}}\mathrm{d}x \label{eq:6-13}\tag{6-13}
\end{align*}

我们用一个矩阵的不等式来比较上面两个式子 

\begin{lemma}[Minkowski 行列式不等式]\label{lem:minkowski_det}
对任意半正定对称 $n\times n$矩阵 $A,B$,有
\[
\mathrm{det}(A+B)^{\frac1n}\ge\mathrm{det}(A)^{\frac1n}+\mathrm{det}(B)^{\frac1n}
\]
等号成立当且仅当 $A=B$。\footnotemark
\end{lemma}
\footnotetext{这个某种程度上其实可以算是矩阵的 BM 不等式。读者可以参考 \url{https://mathoverflow.net/q/65430}等。}


\begin{corollary}\label{col:matrix_ineq}
对任意半正定对称 $(n-1)\times (n-1)$矩阵 $A,B$,
\[
\left(\mathrm{det}\left(\frac{A+B}{2}\right)\right)^{\frac1{n+1}}\ge \frac12\mathrm{det}(A)^{\frac1{n+1}}+\frac12\mathrm{det}(B)^{\frac1{n+1}}
\]
\end{corollary}

\begin{proof}
设 $X$是 $(n-1)\times (n-1)$矩阵,引理~\ref{lem:minkowski_det} 表明函数 $F(X):=\mathrm{det}(X)^{\frac1{n-1}}$是凹函数;再结合 $\varphi(x):=x^{\frac{n-1}{n+1}}$是凹函数且单调递增,二者复合 $\varphi\circ F$就也是凹函数 (见 Boyd 凸优化的 3.2.4 节),用 Jensen 不等式立得。
\end{proof}

把 $-\nabla ^2f$和 $\nabla^2g$代入推论~\ref{col:matrix_ineq} 里,立即可以推出
\[
2\mathrm{det}\left(\frac{\nabla^2g(x)-\nabla^2f(x)}{2}\right)^{\frac1{n+1}}\ge \mathrm{det}(-\nabla^2f(x))^{\frac1{n+1}}+\mathrm{det}(\nabla^2g(x))^{\frac1{n+1}}
\]
因此 \eqref{eq:6-13}式 $\geq$ \eqref{eq:6-12}式,正是我们想要的结论。
\end{proof}

\textbf{第二步}: 证明仿射表面积是上半连续的。其实根据式 \eqref{eq:6-11}——仿射表面积可以表示成下确界的形式,上半连续是自然成立的:

\begin{lemma}\label{lem:as_semi_cont}
若一列凸体 $K_j$依 Hausdorff 度量收敛到 $K$,记作 $K_j\to K$,那么
\[
\limsup_{j\to\infty}as(K_j)\le as(K)
\]
\end{lemma}

\textbf{第三步}: 准备就绪,用 Steiner 对称化!设 $K$在一列单位方向$\{u_j\}_{j=1}^\infty$下依 Hausdorff 度量收敛至球 (定理~\ref{thm:gross}),记 $K_j=S_{u_j}S_{u_{j-1}}\cdots S_{u_2}S_{u_1}K$,那么根据引理~\ref{lem:as_monotone},
\[
as(K)\le as(K_1)\le \cdots\le as(K_j)\le \cdots
\]
取极限
\begin{align*}
as(K)&\le \lim_{j\to\infty}as(K_j)\\
&\le as\left(\lim_{j\to\infty}K_j\right) \tag{引理~\ref{lem:as_semi_cont}}\\
&=as\left(\left(\frac{V_n(K)}{V_n(B_n)}\right)^{\frac1n}B_n\right)\\
&=\left(\frac{V_n(K)}{V_n(B_n)}\right)^{\frac{n-1}{n+1}}as(B_n) \tag{引理~\ref{lem:as_scaling}}
\end{align*}
这正是定理~\ref{thm:as_isoperimetric} 叙述的仿射等周不等式。

如前述,定理~\ref{thm:as_isoperimetric} 的等号刻画条件证明起来比较复杂,从略。

\subsection{与 BS 不等式的等价性}

回顾一下本章证明的两个结论:
\begin{itemize}
\item BS 不等式 (定理~\ref{thm:bs}): $V_n(K)V_n(K^\circ)\le V_n(B_n)^2$;
\item 仿射等周不等式 (定理~\ref{thm:as_isoperimetric}): $as(K)\le nV_n(B_n)^{\frac2{n+1}}V_n(K)^{\frac{n-1}{n+1}}$;
\end{itemize}
在证明定理~\ref{thm:bs}的过程中用到了仿射等周不等式(在式 \eqref{eq:6-8}),因而后者是蕴含前者的。更进一步,可以证明它们俩实际上等价。下面我们用 BS 不等式推一下仿射等周不等式。

其实这个特别容易。由平移不变性,我们把 $K$的 Santaló 点平移到原点。仍用~\eqref{eq:6-11},注意到取 $L=K^\circ$时 $\displaystyle V_n(K,L^\circ)=\int_{\mathbb{S}^{n-1}}h_K(u)\,\mathrm{d}S(K,u)=V_n(K)$,则
\begin{align*}
as(K)&=\inf\Big\{nV_1(K,L^\circ)^{\frac n{n+1}}V_n(L)^{\frac1{n+1}}:0\in\mathrm{int}(L),\;L\text{ 是星形体}\Big\}\\
&\le nV_n(K)^{\frac n{n+1}}V_n(K^\circ)^{\frac1{n+1}} \tag{取 $L=K^\circ$}\\
&=n\big(V_n(K)V_n(K^\circ)\big)^{\frac1{n+1}}V_n(K)^{\frac {n-1}{n+1}}\\
&\le nV_n(B_n)^{\frac2{n+1}}V_n(K)^{\frac {n-1}{n+1}} \tag{由 BS 不等式}
\end{align*}
正是仿射等周不等式,其取等仅当 BS 不等式取等,即 $K$是一个椭球。
  

\end{document}
